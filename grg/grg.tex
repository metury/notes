\documentclass{report}

\usepackage{geometry}
\geometry{a4paper,total={170mm,257mm},left=20mm,top=20mm}
\usepackage[utf8]{inputenc}
\usepackage{amsmath}
\usepackage{amsfonts}
\usepackage{amsthm}
\usepackage{amssymb}
\usepackage{bm}
\usepackage{graphicx}
\usepackage{paralist}
\usepackage[dvipsnames]{xcolor}
\usepackage{caption}
\usepackage{subcaption}
\usepackage{hyperref}
\hypersetup{urlbordercolor=ForestGreen, linkbordercolor=RoyalPurple}
\usepackage{tikz}
\usetikzlibrary{positioning}
\usetikzlibrary{intersections}
\usepackage{algpseudocode}
\usepackage{algorithm}
\usepackage{titling}
\usepackage{pgfplots}
\usepackage{fontawesome5}
\usepackage{luacode}
\usepackage{tcolorbox}

\newtcolorbox{mybox}[1]{colback=SeaGreen!20!white,
	colframe=SeaGreen!70!black,
	colbacktitle=SeaGreen!90!white,
	fonttitle=\bfseries,
	title={#1.},
	center title}

\usepackage{babel}

\title{Geometric Representations of Graphs}
\author{Tomáš Turek\thanks{Many parts of text are taken from the handouts made by Jan Kratochvíl. I also add some of my notes from the lectures and some pictures. Second part is made from my notes from the course lectured by Vít Jelínek. \INFO}}
\titlepicture[width=6in]{res/filament}
\date{\today}

\DeclareMathOperator{\cn}{\textsf{cn}}
\newcommand{\cyc}[2]{\textsf{Cyc}_{#1}(#2)}
\newcommand{\rr}[1]{\overrightarrow{r}^{(#1)}}
\newcommand{\cros}{\textsf{cr}}
\newcommand{\pcros}{\textsf{pcr}}

\usepackage[dvipsnames]{xcolor}
\usepackage{hyperref}

\colorlet{myblue}{RoyalBlue}
\colorlet{myred}{WildStrawberry}
\colorlet{myorange}{Melon}
\colorlet{mygreen}{OliveGreen}
\colorlet{myviolet}{RoyalPurple}
\colorlet{myyellow}{Goldenrod}
\hypersetup{urlbordercolor=Green, linkbordercolor=Blue}
\lighttheme

\begin{document}
	\maketitle	
	\tableofcontents
	\part{Geometric Representations of Graphs I}
	\chapter{Definitions}

\section{Simple hypergraphs}

\begin{defn}
	Hypergraph is a tuple $(X, \M)$ where $\M \subseteq \mathcal{P}(X)$. Or generally just a set system.
\end{defn}

\begin{defn}
	A simple hypergraph (linear, $k$-graph) is $(X,\M)$ if $M_1 \neq M_2 \in \M \Rightarrow |M_1 \cap M_2| \leq 1$.
\end{defn}

\begin{example}
	Lets see some of the easier examples of simple hypergraphs.
	
	\begin{itemize}
		\item Graphs themselves are simple hypergraphs. Where $(X, \M)$ and $\M \subseteq \binom{X}{2}$.
		\item Or generally $k$-graphs, where $\M \subseteq \binom{X}{k}$.
		\item A well known Fano plane, see picture \ref{fano-plane}.
		\item Lets have a set of points $A$ and define $X = \binom{A}{2}$; which are edges in $A$ and $\M = \{\binom{T}{2} | |T| = 3, T \subseteq A\}$; which are triangles in $A$. This is also simple.
	\end{itemize}
\end{example}


\begin{defn}
	A chromatic number of such hypergraphs is defined in a following way:
	
	$$
	\chi (X,\M) := \min \{k | \exists \bigcup_{i=1}^k X_i = X \text{ and no } X_i \text{ contains } M \in \M\}.
	$$
\end{defn}

\noindent In other words: At least two "colors" for each $M \in \M$. And by Ramsey theory we may state that $\forall k \ \exists X: \chi(X,\M) > k$.

\begin{example}
	For fixed $k \in \N$ we have $k$ committees, each of them has $k$ members and they are meeting in a room with $k$ seats. Any two committees are disjoint. Can someone sit at the same place? And how many of them? -- This was stated by Erd\H os, Faber and Lovász in 1972.
\end{example}

\begin{thm}[Kuhn, Osthus, Kang, Kelly, Methuku, 2023]
	Showed that the previous example is true for large $k$.
\end{thm}

And some different formulation is by using simple hypergraphs. Lets have simple hypergraf $(X,\M)$ where $|\M| = k$ and line chromatic number $\leq k$. That is coloring the edges instead. If they meet they have to be distinct.

\begin{prop}
	$\chi_l (K_{2k}) = 2k-1$ for $k \in \N$.
\end{prop}

\begin{proof}[Sketch of proof]
	Lets draw the graph, so the vertices are on a circle. Then take the edges across in the same direction and one from the inside of the circle to the boundary and color them. Then rotate and color once again, until colored.
\end{proof}

\section{Dual hypergraphs}

Lets now define a dual hypergraphs, which may not be so intuitive at a first glance. Lets see a picture \ref{dual-hg} showing the incidence graph for $(X,\M)$. Then the dual is obtained by switching the parts of $(X, \M)$ and $(\M',X')$. Lets denote the dual of $(X,\M)$ as $d(X,\M)$.

\begin{figure}[!ht]\centering
	\begin{subfigure}{.45\textwidth}
		\begin{tikzpicture}
			\draw (0,0) to (6,0);
			\draw (0,3) to (6,3);
			\node (X) at (6.5,0) {$\M$};
			\node (M) at (6.5,3) {$X$};
			\draw (2,0) -- (1,3) node[midway, below, left] {$\in$};
			\draw (2,0) -- (2,3);
			\draw (2,0) -- (3,3);
			\draw (2,0) -- (4,3);
			\node (M) at (2,-.2) {$M$};
			\node (x1) at (1,3.2) {$x_1$};
			\node (x2) at (2,3.2) {$x_2$};
			\node (x3) at (3,3.2) {$x_3$};
			\node (x4) at (4,3.2) {$x_4$};
		\end{tikzpicture}
	\end{subfigure}
	\begin{subfigure}{.45\textwidth}
		\begin{tikzpicture}
			\draw (0,0) to (6,0);
			\draw (0,3) to (6,3);
			\node (X) at (6.5,0) {$X'$};
			\node (M) at (6.5,3) {$\M'$};
			\draw (2,0) -- (1,3) node[midway, below, left] {$\in$};
			\draw (2,0) -- (2,3);
			\draw (2,0) -- (3,3);
			\draw (2,0) -- (4,3);
			\node (M) at (2,-.2) {$x$};
			\node (x1) at (1,3.2) {$M_1$};
			\node (x2) at (2,3.2) {$M_2$};
			\node (x3) at (3,3.2) {$M_3$};
			\node (x4) at (4,3.2) {$M_4$};
		\end{tikzpicture}
	\end{subfigure}
	\caption{Diagram for the dual hypergraph.}
	\label{dual-hg}
\end{figure}

\begin{lemma}
	$(X,\M)$ is simple \ifft $d(X,\M)$ is simple.
\end{lemma}

\begin{proof}[Proof by picture]
	For the proof see the picture \ref{simple-dual-hg}. This $C_4$ like structure happens if it is not simple and hence when we flip the diagram, obtaining the dual, the diagram does not change.
	
	\begin{figure}[!ht]\centering
		\begin{tikzpicture}
			\draw (0,0) to (6,0);
			\draw (0,3) to (6,3);
			\node (X) at (6.5,0) {$\M$};
			\node (M) at (6.5,3) {$X$};
			\draw (2,0) -- (2,3);
			\draw (4,0) -- (4,3);
			\draw (2,0) -- (4,3);
			\draw (4,0) -- (2,3);
			\node (M1) at (2,-.2) {$M_1$};
			\node (M2) at (4,-.2) {$M_2$};
			\node (x1) at (2,3.2) {$x_1$};
			\node (x2) at (4,3.2) {$x_2$};
		\end{tikzpicture}
		\caption{Simple dual graphs proof.}
		\label{simple-dual-hg}
	\end{figure}
\end{proof}

Now lets denote $A(X,\M)$ as an incidence matrix of a given hypergraph, then the dual has incidence matrix $A(d(X,\M)) = A^T(X,\M)$. Lets consider $(X, \M)$ a $k$-uniform hypergraph. Can we somehow bound the size of $|\M|$? We may establish trivial bounds as $0 \leq |\M| \leq \binom{X}{k}$. We will further make better bound. lets see the picture \ref{bounds}. We will be using double counting, for which we notice that for some $\binom{X}{2}$ we have at least one $M \in \M$.

\begin{figure}[!ht]\centering
	\begin{tikzpicture}
		\draw (0,0) to (6,0);
		\draw (0,3) to (6,3);
		\node (X) at (6.5,0) {$\M$};
		\node (M) at (6.5,3) {$\binom{X}{2}$};
		\draw (4,0) -- (4,3);
		\draw (4,0) -- (3,3);
		\draw (4,0) -- (5,3);
		\node (M) at (4,-.2) {$M$};
		\node (Xo2) at (1,3.3) {$\binom{X}{2}$};
		\draw (1,3) -- (2,0);
		
		\node (x2) at (4,3.3) {$\binom{k}{2}$};
	\end{tikzpicture}
	\caption{Providing better bound.}
	\label{bounds}
\end{figure}

\noindent Hence we may compute the following.

$$
|\M| \cdot \binom{k}{2} = \sum_{M \in \M} \binom{|M|}{2} \leq \binom{|X|}{2}
$$

\noindent Therefore we can obtain the bound.

$$
|\M| \leq \frac{\binom{|X|}{2}}{\binom{k}{2}}
$$

\noindent See that this bound is actually tight. For $k=2$ we can consider a graph $K_n$ and for $k=3$ we may look at Fano plane. If the equality hold we call it \textit{Steiner system}. Or in other words it is true if $\forall x \neq y \in X \ \exists! M \in \M$ such that $\{x,y\} \subseteq M$. For $k = 3$ we call this \textit{Steiner triple system} or STS for short (one can be seen as Fano plane and the other as another seen on picture \ref{k=3}). This is particularly used in experiments and mainly in agriculture. Usually this is then denoted as \textit{BIBD} which stands for balanced incomplete block design.

\begin{figure}[!ht]\centering
	\begin{tikzpicture}[scale=.6, l/.style = {line width=2pt},
		n/.style = {draw, circle, fill, inner sep=1.5pt},
		o/.style = {myorange},
		r/.style = {myred},
		b/.style = {myblue},
		g/.style = {mygreen},
		p/.style = {myviolet}]
		\foreach \j in {1,...,3} {
			\foreach \i in {1,...,3} {
				\node[n] (\j-\i) at (2*\j,2*\i) {};
			}
		}
		
		\draw[l,o] (1-1) to (1-2) to (1-3);
		\draw[l,o] (2-1) to (2-2) to (2-3);
		\draw[l,o] (3-1) to (3-2) to (3-3);
		
		\draw[l,b] (1-1) to (2-1) to (3-1);
		\draw[l,b] (1-2) to (2-2) to (3-2);
		\draw[l,b] (1-3) to (2-3) to (3-3);
		
		\draw[l,g] (1-1) to (2-2) to (3-3);
		\draw[l,g] (1-2) to (2-3) to[out=20,in=180] (6.5,6.5) to[out=0, in=0] (3-1);
		\draw[l,g] (1-3) to[out=200,in=180] (1.5,1.5) to[out=0,in=200] (2-1) to (3-2);
		
		\draw[l,r] (1-1) to[out=330,in=250] (6.5,1.5) to[out=70,in=330] (3-2) to (2-3);
		\draw[l,r] (3-3) to[out=120,in=0] (1.5,6.5) to[out=180,in=150] (1-2) to (2-1);
		\draw[l,r] (1-3) to (2-2) to (3-1);
	\end{tikzpicture}
	\caption{Another Steiner triple system.}
	\label{k=3}
\end{figure}

We may say that for STS to exists it must hold that both $n-1$ and $\frac{\binom{n}{2}}{\binom{3}{2}}$ must be integers. So it only exists if $n$ is either $6k+1$ or $6k+3$.

\begin{thm}
	Steiner triple system exists \ifft $n$ is either $6k+1$ or $6k+3$.
\end{thm}

\begin{proof}
	We will be showing how it can be generated. That is from two STS we create a new one. First we can observe that if both $(X_1, \M_1)$ and $(X_2, \M_2)$ are STS then also $(X_1 \times X_2, \M)$ is STS, where $\M$ can be viewed from a picture \ref{M-def} or from an algebraic view. That is we have algebra of $(X, \M)$ $\mathbb{A}_{(X,\M)}$ then it is $\mathbb{A}_{(X_1,\M_1)} \times \mathbb{A}_{(X_2,\M_2)}$.
	
	\begin{figure}[!ht]\centering
		\begin{tikzpicture}[thick]
			\draw (1,1) -- (7,1);
			\draw (1,1) -- (1,7);
			\node at (6.6, 0.6) {$(X_2, \M_2)$};
			\node at (0, 6.6) {$(X_1, \M_1)$};
			\node[myblue] at (0.6, 2) {$x_1$};
			\node[myblue] at (0.6, 3) {$y_1$};
			\node[myblue] at (0.6, 5) {$z_1$};
			\node[mygreen] at (2, 0.6) {$x_2$};
			\node[mygreen] at (4, 0.6) {$y_2$};
			\node[mygreen] at (5, 0.6) {$z_2$};
			\node[myorange] at (2, 2) (x) {$x$};
			\node[myorange] at (4, 3) (y) {$y$};
			\node[myorange] at (5, 5) (z) {$z$};
			\draw (2,1) -- (x);
			\draw (1,2) -- (x);
			\draw (4,1) -- (y);
			\draw (1,3) -- (y);
			\draw (5,1) -- (z);
			\draw (1,5) -- (z);
		\end{tikzpicture}
		\caption{Definition of $\M$, where $\{x,y,z\} \in \M$.}
		\label{M-def}
	\end{figure}
\end{proof}

\section{Introducing BIBD and integrality conditions}

\begin{defn}[BIBD]
	Hypergraph $(X, \M)$ is BIBD with parameters $(v,k,\lambda,t)$ if $|X| = v$ $\M \subseteq \binom{X}{k}$ and $\forall x_1, \dots x_t \in \binom{X}{t}$ we have that $|\{M \in \M | \{x_1, \dots, x_t\} \subseteq M\}| = \lambda$. 
\end{defn}

With the similar arguments we may see that $|\M| = \frac{\binom{|X|}{t}}{\binom{k}{t}} \cdot \lambda$ this holds and hence it has to be an integer. Now the question is whether there actually exists BIBD with given values $(n,k,\lambda,t)$? There is pretty simple observation that if we take $A \subseteq X$ of size $|A| = a$ it must be true that $\frac{\binom{n - a}{t - a}}{\binom{k - a}{t - a}} \cdot \lambda$ must be integer for the number of such $M$'s containing $A$. From these properties we establish the integer constraints for the existence of BIBD.

\begin{prop}[BIBD necessary constraints]
	If we have BIBD $(n,k,\lambda,t)$ everything has to hold. Firstly the size of $\M$
	
	$$
	|\M| = \frac{\binom{|X|}{t}}{\binom{k}{t}} \cdot \lambda
	$$
	
	\noindent and also the integrality of the following fractions.
	
	$$
	\frac{\binom{n - a}{t - a}}{\binom{k - a}{t - a}} \cdot \lambda \ \text{for } a = 1, \dots, t.
	$$
\end{prop}

Now recall that we have shown how two STS can create a new STS. Now we will show a similar proposition.

\begin{prop}
	If there exists BIBD $(v,3,2,1)$ then also BIBD $(2v +1,3,2,1)$ exists.
\end{prop}

\begin{proof}
	The proof is by a picture \ref{new-sts-twice}. Firstly lets have STS and duplicate it. Then also add new vertex. We will create new $M$'s so that the properties of STS are still satisfied. Which also includes the newly created vertex.
	
	\begin{figure}[!ht]\centering
		\begin{tikzpicture}[main/.style = {circle, draw, fill, inner sep=1.5pt}]
			\node[main] at (8,-2) (c) {};
			\foreach \x in {1,...,7} {
				\node[main] at (2*\x, 1) (1-\x) {};
				\node[main] at (2*\x, 3) (2-\x) {};
				\path[draw, line width=1.5pt, myorange] (c) -- (1-\x) -- (2-\x);
			}
			\draw[myblue, fill, fill opacity=0.5] (1.5,3.5) rectangle (6.5,2.5);
			\node[myblue] at (1, 3) {$M_1$};
			\draw[mygreen, fill, fill opacity=0.5] (1.5,1.5) rectangle (6.5,0.5);
			\node[mygreen] at (1, 1) {$M_2$};
		\end{tikzpicture}
		\caption{Newly created larger STS, where for $M_1$ and $M_2$ we create all possible $M$ so that at least one element is from $M_1$ and $M_2$ and in total we have 3 elements.}
		\label{new-sts-twice}
	\end{figure}
\end{proof}

\begin{thm}[R. Wilson, R. Chatouri]
	$\forall k, \lambda, t =2$ for every large $n$ the integrality conditions are sufficient for the existence of BIBD.
\end{thm}
	\include{grg-i/02-interval, perm, fun graphs}
	\chapter{Interval filament graphs}

\begin{defn}
	Given a half-plane $\pi$ with a border-line $l$, an \textbf{interval filament} in $\pi$ is a simple curve with endpoints on $l$, its interior lying in $\pi$ and within the stripe determined by lines perpendicular to $l$ and passing through the endpoints of the curve. The class of \textbf{interval filament graphs} is $\text{IFA} = \mathcal{IG}\{$interval filaments in a half-plane$\}$.
\end{defn}

\begin{figure}[!ht]\centering
	\begin{tikzpicture}
		\draw[color = gray] (2,0) -- (10,0);
		\node at (3,.3) {$l$};
		\draw[color=black, thick] (5,.01) edge (10,.01);
		\draw[dotted] (5,0) edge (5, 5);
		\draw[dotted] (10,0) edge (10,5);
		
		\draw[color=myred, thick] (5, .01) .. controls (10,5) and (4,3) .. (7, 4);
		\draw[color=myred, thick] (7, 4) .. controls (11,1) and (1,1) .. (10, .01);
		
		\fill[color=Red] (7,4) circle[radius=.35pt];
	\end{tikzpicture}
	\caption{An illustration to the definition. The name “interval-filament” comes from the fact that the curve “lives above” the interval that is determined by the endpoints of the filament on the boundary line $l$.}
\end{figure}

\begin{comm}
	If the base intervals of two interval-filaments overlap (i.e., they are not disjoint, but none of them is a subinterval of the other one), the filaments necessarily cross each other and the corresponding vertices in the intersection graph are adjacent (cf. the blue and red filaments in Fig. \ref{filaments}). On the other hand, if one of the base intervals is included in the other one, their filaments may or may not be disjoint (cf. the blue and yellow filaments for the disjoint case, and the red and green filaments for the non-disjoint one, both in Fig. \ref{filaments}).
\end{comm}

\begin{figure}[!ht]\centering
	\begin{tikzpicture}
		\draw[color = gray] (0,0) -- (10,0);
		\node at (1,.3) {$l$};
		
		\draw[color=black, thick] (5,.01) edge (10,.01);
		\draw[color=black, thick] (7,.03) edge (9,.03);
		\draw[color=black, thick] (2,.03) edge (6,.03);
		\draw[color=black, thick] (3.5,.01) edge (4.5,.01);
		
		\draw[color=myred, thick] (5, .01) .. controls (10,5) and (4,3) .. (7, 4);
		\draw[color=myred, thick] (7, 4) .. controls (11,1) and (1,1) .. (10, .01);
		\fill[color=myred] (7,4) circle[radius=.35pt];
		
		\draw[color=myyellow, thick] (3.5, .01) .. controls (4,3) and (4,.5) .. (4.5, .01);
		\draw[color=myblue, thick] (2, .03) .. controls (6,6) and (2,2) .. (6, .03);
		\draw[color=mygreen, thick] (7, .03) .. controls (7.5,4) and (8,1) .. (9, .03);
	\end{tikzpicture}
	\caption{An illustration to the possible relative positions of interval-filaments.}
	\label{filaments}
\end{figure}

\begin{observ}
	$\text{IFA}$ contains $\text{INT}$, $\text{CHOR}$, $\text{FUN}$, $\text{CIR}$, $\text{CA}$, and $\text{PC}$.
\end{observ}

\begin{defn}[class $\mathcal{A}$-mixed]
	Let $\mathcal{A}$ be a graph class. A graph $G = (V, E)$ belongs to the class $\mathcal{A}$-mixed if its edge set allows a partition $E = E_1 \cup E_2$ and a transitive orientation $\overrightarrow{E_2}$ of $E_2$ such that for any three vertices $u, v, w \in V(G)$, $uv \in \overrightarrow{E_2}$ and $vw \in E_1$ imply $uw \in E_1$.
\end{defn}

\begin{thm}
	co-IFA = (co-INT)-mixed.
\end{thm}

\begin{proof}
	"$\subseteq$": Let $G = (V, E) \in \text{IFA}$ and let $\{f_u : u \in V\}$ be an interval-filament representation of $G$, with $f_u$ being a filament with its base interval $I_u$ lying on a common line $l$. Then its complement $-G$ is in co-IFA, and our goal is to show that $-G \in$ (co-INT)-mixed. Define
	
	$$
	E_1 = \{uv : I_u \cap I_v = \emptyset\}
	$$
	
	and
	
	$$
	E_2 = \{uv : ((I_u \subseteq I_v ) \lor (I_v \subseteq I_u )) \land (f_u \cap f_v = \emptyset)\}.
	$$
	
	Then
	$$
	\overrightarrow{E_2} = \{uv : (I_u \subseteq I_v ) \land (f_u \cap f_v = \emptyset)\}
	$$
	
	is a transitive orientation of $E_2$, $E_1 \cup E_2 = \binom{V}{2} \setminus E$, $(V, E_1) \in$ co-INT and $E_1$ and $\overrightarrow{E_2}$ satisfy the mixing property. Hence $-G \in$ (co-INT)-mixed.
	
	"$\supseteq$": Consider a graph in (co-INT)-mixed and denote its complement by $G = (V, E)$. Our goal is to show that $-G \in$ co-IFA, or equivalently that $G \in$ IFA. Since $-G$ is (co-INT)-mixed, the non-edges of $G$ can be partitioned into disjoint sets $E_1$, $E_2$ so that $G_1 = (V, E_1) \in$ co-INT, and $E_2$ allows a transitive orientation $\overrightarrow{E_2}$ such that $E_1$ and $\overrightarrow{E_2}$ satisfy the mixing property. Fix this partition $E_1$, $E_2$ and the transitive orientation $\overrightarrow{E_2}$.
	
	Consider an interval intersection representation $\mathcal{R} = \{I_u = [l_u , r_u] : u \in V\}$ of $-G_1$ . We will only consider representations in which all end-points of the intervals are different points on the base line. For two vertices $u, v \in V$, there are three possibilities for the relative position of the intervals $I_u$, $I_v$ -- the intervals are disjoint (when $r_u < l_v$ or $r_v < l_u$), or in inclusion (when $l_u < l_v < r_v < r_u$ or $l_v <
	l_u < r_u < r_v$), or overlapping (when $l_u < l_v < r_u < r_v$ or $l_v < l_u < r_v < r_u$). Given a representation	$\mathcal{R}$, we set $\mathcal{R}_{\text{disjoint}} = \{uv : I_u \text{ and } I_v \text{ are disjoint}\}$, $\mathcal{R}_{\text{inclusion}} = \{uv : I_u \text{ and } I_v \text{ are in inclusion}\}$ and $\mathcal{R}_{\text{overlap}} = \{uv : I_u \text{ and } I_v \text{ are overlapping}\}$. Observe that $\mathcal{R}$ is an interval intersection representation of $-G_1$ if and only if $\mathcal{R}_{\text{disjoint}} = E_1$. In such a case, $\mathcal{R}_{\text{inclusion}} \cup \mathcal{R}_{\text{overlap}} = E \cup E_2$.
	
	\begin{figure}[!ht]\centering
		\begin{tikzpicture}
			%% Lines
			\draw[line width = 3, color=lightgray] (0,0) -- (10,0);
			% Disjoint
			\draw[line width = 3, color=black] (1,.10) -- (1.7,.10);
			\draw[line width = 3, color=black] (2,.10) -- (2.8,.10);
			% In inclusion
			\draw[line width = 3, color=black] (4,.10) -- (6,.10);
			\draw[line width = 3, color=black] (4.5,.20) -- (5.4,.20);
			% Overlaping
			\draw[line width = 3, color=black] (7,.10) -- (8.4,.10);
			\draw[line width = 3, color=black] (7.5,.20) -- (9.5,.20);
			%% Text boxes
			\node (l) at (.5,.3) {$l$};
			\node (disjoint) at (1.9, -.5) {disjoint};
			\node (incl) at (5, -.5) {in inclusion};
			\node (overlap) at (8.3, -.5) {overlapping};
		\end{tikzpicture}
		\caption{An illustration to the possible relative positions of pairs of intervals.}
	\end{figure}
	
	\begin{claim}
		If $E_1$ and $E_2$ satisfy the mixing condition, $-G_1$ has an interval intersection representation such that for every two vertices $u, v \in V$, $uv \in \overrightarrow{E_2}$ implies $I_u \subseteq I_v$.
	\end{claim}
	
	\begin{proof}[Proof of claim]
		Let us call a violation a pair of endpoints of intervals $I_u$, $I_v$ with $uv \in \overrightarrow{E_2}$ such that $l_u < l_v$ or $r_v < r_u$. Observe that overlapping intervals provide one violation, while intervals in inclusion provide either $0$ (when $I_u \subseteq I_v$) or $2$ (when $I_v \subseteq I_u$) violations.
		
		Consider a representation $\mathcal{R}$ with the smallest possible number of violations. We will prove that
		this number is zero, i.e., that this $\mathcal{R}$ satisfies the statement of the Claim.
		
		Suppose, for the contrary, that $\mathcal{R}$ has $k > 0$ violations. For every violation, count the number of endpoints of other intervals lying between the two endpoints forming this violation, and let $m$ be the minimum of these counts over all violations. Assume that $\mathcal{R}$ is chosen such that $m$ is minimum possible among all representations with $k$ violations. In the case analysis which follows, we assume that the violation achieving this minimum number of endpoints is formed by the right endpoints of $I_u$ and $I_v$. If it is achieved by their left endpoints, the arguments are similar (and symmetric).
		
		\begin{itemize}[]
			\item \underline{Case $B1$. $m = 0$} Let $r_v < r_u$, with $uv \in \overrightarrow{E_2}$, be a violation such that there are no other endpoints between $r_v$ and $r_u$. Change the representation $\mathcal{R}$ to $\mathcal{R}'$ by changing the interval $I_v$ to $I_v' = [l_v , r_u + \epsilon]$ for a positive $\epsilon$ small enough so that no endpoint of any interval is between $r_u$ and $r_u + \epsilon$ (other intervals remain the same as in $\mathcal{R}$). Then $\mathcal{R}'$ is still an interval intersection representation of $-G_1$, but it has $k - 1 < k$ violations, contradicting the assumption that $k$ was the minimum possible number of violations in an interval representation of $-G_1$.
			
			\item \underline{Case $B2$. $m > 0$} Let $r_v < r_u$, with $uv \in \overrightarrow{E_2}$, be a violation with $m$ endpoints between $r_v$ and $r_u$. Let $P$ be the leftmost of these endpoints.
			
			\begin{itemize}[]
				\item \underline{Subcase $B2\alpha$. $P$ is the right endpoint of an interval.} Let $w \in V$ be such that $P = r_w$. Change the represen-\newline tation $\mathcal{R}$ to $\mathcal{R}'$ by changing the interval $I_v$ to $I_v' = [l_v , r_w + \epsilon]$ for a positive $\epsilon$ small enough so that no endpoint of any interval is between $r_w$ and $r_w + \epsilon$. Then $\mathcal{R}'$ is still an interval intersection representation of $-G_1$. The number of endpoints between $r_v'$ and $r_u$ is $m - 1 < m$. If $\mathcal{R}'$ had the same number of violations as $\mathcal{R}$, this would contradict the assumption of minimality of $m$. Thus $\mathcal{R}'$ must have more than $k$ violations. The only new violation can be formed by $r_w < r_v'$, which means that $vw \in \overrightarrow{E_2}$. Then the transitivity of $\overrightarrow{E_2}$ implies that $uw \in \overrightarrow{E_2}$ and $r_w < r_u$ is a violation in $\mathcal{R}$ with $m - 1 < m$ endpoints between $r_w$ and $r_u$, contradicting the assumption on minimality of $m$.
				
				\item \underline{Subcase $B2\beta$. $P$ is the left endpoint of an interval.} Let $w \in V$ be such that $P = l_w$. Then $I_v \cap I_w = \emptyset$ and hence $vw \in E_1$. The mixing property applied to $u, v, w$ then implies $uw \in E_1$, which is impossible since $P \in I_u \cap I_w \neq \emptyset$.
			\end{itemize}
		\end{itemize}
		
		Thus both Cases $B2$ and $B1$ lead to contradictions, and hence $k = 0$ and $\mathcal{R}$ satisfies the property described in the Claim.
	\end{proof}
	
	Assume we are having an interval intersection representation $\mathcal{R}$ guaranteed by the Claim. It follows that $E_2 \subseteq \mathcal{R}_\text{inclusion}$, and hence $\mathcal{R}_\text{overlap} \subseteq E$. In the first step, define, for every vertex $u \in V$, the interval filament $f_u$ as the half-circle with diameter $I_u$. At this point we observe the following
	
	\begin{enumerate}
		\item if $I_u$ and $I_v$ are disjoint, so are also $f_u$ and $f_v$, which corresponds to the fact that $uv \in E_1$ and thus $uv \notin E$;
		\item if $I_u$ and $I_v$ are overlapping, $f_u$ and $f_v$ cross each other, which corresponds to the fact that $uv \in E$ which is observed above; however
		\item if $I_u$ and $I_v$ are in inclusion, say $I_u \subseteq I_v$, then either $uv \in E_2$ (and thus $uv \in \overrightarrow{E_2}$) or $uv \in E$, but $f_u \cap f_v = \emptyset$ in both cases.
	\end{enumerate}
	
	We will now modify some of the filaments in order to make filaments of case 3 intersect if the corresponding vertices are adjacent in $G$. For every pair of vertices $u, v \in V$ such that $I_u \subseteq I_v$ and $uv \in E$, choose a line $l_u$ perpendicular to the base line $l$ and crossing it in an interior point of $I_u$. Then pull the filament $f_u$ up along $l_u$ to make it cross $f_v$ (cf. an illustrative Fig. \ref{left}). To avoid creating undesired intersections with other filaments, we must modify also those filaments which cross the line $l_u$ between its crossings with $f_u$ and $f_v$. Imagine this as a dynamic process, as if $f_u$ would slowly grow a spike upward and this spike would be pushing every filament $f_w$ in front of it, if $w$ is such that $uw \notin E$. Moreover, if there were another filament $f_z$ above $f_w$ such that $wz \notin E$, $f_z$ would be pushed by $f_w$ etc. See Fig. \ref{right}. We claim that in this way no undesired intersections arise. Such could be only between one of the pushed filaments, say $f_y$, and the filament $f_v$. If $f_y$ is pushed, there is a sequence of vertices $u_0 = u, u_1 , \dots , u_t = y$ such that $f_{u_{i-1}}$ pushes $f_{u_{i}}$ for $i = 1, 2, \dots , t$, i.e., $u_{i-1} u_i \notin E$	and $I_{u_{i-1}} \subseteq I_{u_{i}}$, hence $u_{i-1} u_i \in \overrightarrow{E_2}$. Transitivity of $\overrightarrow{E_2}$ then implies $uy \in \overrightarrow{E_2}$. Similarly, if $f_y$ should not cross $f_v$, i.e., $yv \notin E$, we would have $yv \in \overrightarrow{E_2}$. But that would imply $uv \in \overrightarrow{E_2}$ contradicting the assumption that $uv \in E$.
	
	\begin{figure}[!ht]\centering
		\begin{subfigure}{0.45\textwidth}\centering
			\begin{tikzpicture}
				%% Lines
				\draw[line width = 3, color=lightgray] (0,0) -- (6,0);
				\draw[color=myred, thick] (1,0) arc (180:0:2);
				\draw[color=myblue, thick] (2,0) arc(180:70:1);
				\draw[color=myblue, thick] (4,0) arc(-180:-130:-1);
				\draw[dotted] (3.5, 0) -- (3.5, 3);
				\draw[color=myblue, thick] (3.35,0.93) -- (3.5, 2.5);
				\draw[color=myblue, thick] (3.65,0.75) -- (3.5, 2.5);
				%% Text boxes
				\node (l) at (.5,.3) {$l$};
				\node (lu) at (3.7,2.8) {$l_u$};
				\node (fv) at (4.8,1.5) {$f_v$};
				\node (fu) at (2.7,.6) {$f_u$};
			\end{tikzpicture}
			\caption{Crossing filaments.}
			\label{left}
		\end{subfigure}
		\begin{subfigure}{0.45\textwidth}\centering
			\begin{tikzpicture}
				%% Lines
				\draw[line width = 3, color=lightgray] (0,0) -- (6,0);
				\draw[color=myred, thick] (1,0) arc (180:0:2);
				
				\draw[color=mygreen, thick] (1.2,0) arc (180:0:1.8);
				
				\draw[color=myblue, thick] (2,0) arc(180:70:1);
				\draw[color=myblue, thick] (4,0) arc(-180:-130:-1);
				\draw[color=myblue, thick] (3.35,0.93) -- (3.5, 2.5);
				\draw[color=myblue, thick] (3.65,0.75) -- (3.5, 2.5);
				
				\draw[color=mygreen, thick] (1.5,0) arc(180:80:1.5);
				\draw[color=mygreen, thick] (4.5,0) arc(-180:-120:-1.5);
				\draw[color=mygreen, thick] (3.24,1.47) -- (3.5, 3);
				\draw[color=mygreen, thick] (3.76,1.28) -- (3.5, 3);
				%% Text boxes
				\node (l) at (.5,.3) {$l$};
				\node (fv) at (4.8,1.5) {$f_v$};
				\node (fu) at (2.7,.6) {$f_u$};
			\end{tikzpicture}
			\caption{Pushing outwards.}
			\label{right}
		\end{subfigure}
		\caption{Final modification of interval-filaments in the representation.}
	\end{figure}
\end{proof}
	\include{grg-i/04-clique and ind sets}
	\include{grg-i/05-chord-recog}
	\include{grg-i/06-com-reg}
	\chapter{Visibility and contact representations of planar graphs}

\section{$s-t$ numbering}

\begin{defn}
	An $s - t$ numbering (also called a bipolar orientation) of a graph is an acyclic orientation of its edges which has exactly one source (a vertex with no in-coming arcs) and exactly one sink (a vertex with no out-going arcs).
\end{defn}

\begin{prop}
	Every vertex 2-connected graph has an $s - t$ numbering.
\end{prop}

\begin{proof}
	By induction on adding ears, using the Ear Decomposition Lemma. Given the starting cycle, choose two distinct vertices on it (to be the source and the sink) and orient the cycle as two directed paths from the source to the sink. The loop invariant will be that every vertex lies on a directed path from the source to the sink. When adding an ear, orient it in the direction from the source to the sink if both end-vertices of the ear lie on the same path from the source to the sink. If the end-vertices of the ear lie on different paths from the source to the sink, the ear may be oriented either way (but all of its edges in the same direction).
\end{proof}

\begin{thm}
	Every vertex 2-connected plane graph has an $s - t$ numbering and a noncrossing planar drawing such that
	
	\begin{enumerate}[i)]
		\item every edge is drawn as a $y$-monotone curve (i.e., every horizontal line crosses the drawing of the edge in at most one point), and
		\item the drawing of every edge is oriented upward in the $s - t$ numbering.
	\end{enumerate}
	\label{thm-1}
\end{thm}

This $s - t$ numbering and a corresponding planar drawing can be constructed in polynomial time.

\begin{comm}
	By saying a plane graph it is meant a planar graph with a given noncrossing drawing in the plane, and it is understood that only drawings which are homeomorphic to the given one are considered, including the choice of the outerface.
\end{comm}

\begin{proof}
	By Ear Decomposition Lemma, the given graph $G$ can be constructed from the cycle bounding its outerface by adding ears. Choose two distinct vertices, say $a$ and $b$, on this cycle, place them in the plane so that they have different $y$-coordinates, the coordinate of $a$ being smaller than the coordinate of $b$, orient the two $a - b$ paths forming the cycle from $a$ to $b$ and draw them as $y$-monotone paths from (the drawing of) $a$ to (the drawing of) $b$.
	
	Then continue adding the ears, orienting their edges and adding them to the drawing constructed while keeping the following loop invariant -- the drawing is homeomorphic to the so far constructed part of $G$, it satisfies i) and ii), all vertices have distinct $y$-coordinates, and each face is bounded by two upward oriented paths connecting its vertices with the lowest and the highest $y$-coordinates (we will refer to these paths as the \textit{left} one and the \textit{right} one). Note that i) and ii) together imply that the drawing of any directed path is $y$-monotone.
	
	When an ear is added, it is added inside a face of the so far constructed part of $G$. Direct the edges of the ear in the direction from the vertex with the lower $y$-coordinate to the vertex with the higher one (the end-vertices of the ear belong to the so far constructed part of $G$ and so they have already been drawn). If the end-vertices belong to the same bounding path (the left one or the right one) of the face they should be draw in, draw the ear as a $y$-monotone curve contouring the bounding path. If the end-vertices belong to different bounding paths, draw the ear as a $y$-monotone curve that contours the path which contains the lower endpoint and traverse the face to the other endpoint almost horizontally close to the $y$-coordinate of this other endpoint, in order to avoid crossings with other edges. It is easy to check that the loop invariant of the construction is fulfilled.
\end{proof}

\begin{figure}[!ht]\centering
	\begin{subfigure}{0.45\textwidth}\centering
		\begin{tikzpicture}[main/.style = {draw, circle, thick, fill}]
			% Draw ellipsoid
			\draw[blue, fill=blue!20] (0,0) ellipse (2 and 3);
			\node[main] (1) at (0,-2) {};
			\node[main] (2) at (-1,-1.3) {};
			\node[main] (3) at (1,-1.1) {};
			\node[main] (4) at (-0.8,-0.3) {};
			\node[main] (5) at (0.7,-0.2) {};
			\node[main] (6) at (-1.1,1) {};
			\node[main] (7) at (0.9,1) {};
			\node[main] (8) at (0,2) {};
			
			\draw[->, thick] (1) edge (2);
			\draw[->, thick] (1) edge (3);
			\draw[->, thick] (2) edge (4);
			\draw[->, thick] (3) edge (5);
			\draw[->, thick] (4) edge (6);
			\draw[->, thick] (5) edge (7);
			\draw[->, thick] (6) edge (8);
			\draw[->, thick] (7) edge (8);
			
			\draw[->, thick] (0, -2.6) -- (1);
			\draw[->, thick] (8) -- (0, 2.6);
			
			\draw[->, thick, color=mygreen, bend right = 60] (2) edge (6);
		\end{tikzpicture}
	\end{subfigure}
	\begin{subfigure}{0.45\textwidth}\centering
		\begin{tikzpicture}[main/.style = {draw, circle, thick, fill}]
			% Draw ellipsoid
			\draw[blue, fill=myblue!20] (0,0) ellipse (2 and 3);
			\node[main] (1) at (0,-2) {};
			\node[main] (2) at (-1,-1.3) {};
			\node[main] (3) at (1,-1.1) {};
			\node[main] (4) at (-0.8,-0.3) {};
			\node[main] (5) at (0.7,-0.2) {};
			\node[main] (6) at (-1.1,1) {};
			\node[main] (7) at (0.9,1) {};
			\node[main] (8) at (0,2) {};
			
			\draw[->, thick] (1) edge (2);
			\draw[->, thick] (1) edge (3);
			\draw[->, thick] (2) edge (4);
			\draw[->, thick] (3) edge (5);
			\draw[->, thick] (4) edge (6);
			\draw[->, thick] (5) edge (7);
			\draw[->, thick] (6) edge (8);
			\draw[->, thick] (7) edge (8);
			
			\draw[->, thick] (0, -2.6) -- (1);
			\draw[->, thick] (8) -- (0, 2.6);
			
			\draw[->, thick, color=mygreen, bend right, out=-50, in=120] (2) edge (7);
		\end{tikzpicture}
	\end{subfigure}
	\caption{An illustration to adding an ear in the construction of an $s-t$ numbering and a corresponding upward drawing.}
\end{figure}

\section{Rectangle visibility representations}

\begin{defn}
	A \textbf{rectangle visibility representation} of a plane graph $G$ is an arrangement of disjoint axes-aligned rectangles in the plane such that the (unions of) horizontal sides of the rectangles correspond to the vertices of $G$, the (unions of) vertical sides to the faces of $G$, and each rectangle corresponds to the edge joining the vertices containing the upper and lower sides of the rectangle, and at the same time to the dual edge joining the faces corresponding to the left and right sides of the rectangle.
\end{defn}

\begin{comm}
	In this section we allow multiple edges without explicitly talking about \\ multigraphs. Moreover, we artificially choose two vertices on the boundary of the outerface and divide the outerface into two faces by "infinite" dummy edges starting in these points.
\end{comm}

\begin{figure}[!ht]\centering
	\begin{subfigure}{0.6\textwidth}\centering
		\begin{tikzpicture}[main/.style = {draw, circle, thick, color = myblue}]
			\node[main] (4) at (0,0) {4};
			\node[main] (3) at (-2,1.7) {3};
			\node[main] (6) at (3,3) {6};
			\node[main] (2) at (-0.8,4) {2};
			\node[main] (5) at (3,6) {5};
			\node[main] (1) at (1,6) {1};
			\node[color=mygreen] (b) at (-0.9, 1.7) {$b$};
			\node[color=mygreen] (a) at (-2, 4.3) {$a$};
			\node[color=mygreen] (c) at (1, 3.3) {$c$};
			\node[color=mygreen] (d) at (2.4, 5) {$d$};
			\node[color=mygreen] (e) at (4, 7) {$e$};
			\draw[thick, color=myblue] (4) edge (3);
			\draw[thick, color=myblue] (4) edge (2);
			\draw[thick, color=myblue] (4) edge (6);
			\draw[thick, color=myblue] (5) edge (6);
			\draw[thick, color=myblue] (6) edge (1);
			\draw[thick, color=myblue] (2) edge (1);
			\draw[thick, color=myblue] (1) edge (5);
			\draw[thick, color=myblue] (2) edge (3);
			\draw[dotted] (0, -1) -- (4);
			\draw[dotted] (1) -- (1, 7);
			\draw[thick, color=mygreen] (b) edge (c);
			\draw[thick, color=mygreen] (d) edge (c);
			\draw[thick, color=mygreen] (b) edge (a);
			
			\coordinate (6') at (4,2);
			\coordinate (3') at (-3, 1);
			
			\draw[thick, color=mygreen] (b) to[out = 240, in = -70] (3') to[out = 110, in = 190] (a);
			\draw[thick, color=mygreen, bend left = 80] (a) edge (c);
			\draw[thick, color=mygreen] (e) to[out = 45, in = 45] (6') to[out = -135, in = -30] (c);
			\draw[thick, color=mygreen, bend left = 80] (d) edge (e);
			\draw[thick, color=mygreen, bend right = 90] (d) edge (e);
		\end{tikzpicture}
	\end{subfigure}
	\begin{subfigure}{0.35\textwidth}\centering
		\begin{tikzpicture}
			\draw[black] (0,0) rectangle (1,1);
			\draw[black] (0,1) rectangle (1,2);
			\draw[black] (1,0) rectangle (2,2);
			\draw[black] (2,0) rectangle (5,1);
			\draw[black] (0,2) rectangle (2,3);
			\draw[black, fill=orange!30] (2,1) rectangle (3,3);
			\draw[black] (3,1) rectangle (5,2);
			\draw[black] (3,2) rectangle (5,3);
			\node[myblue] at (0.4, 3.2) {1};
			\node[myblue] at (0.4, 2.2) {2};
			\node[myblue] at (0.4, 1.2) {3};
			\node[myblue] at (0.4, 0.2) {4};
			\node[mygreen] at (-0.2, 1.5) {$a$};
			\node[mygreen] at (1.2, 1.2) {$b$};
			\node[mygreen] at (1.8, 1) {$c$};
			\node[myblue] at (4, 1.2) {5};
			\node[myblue] at (4, 2.2) {6};
			\node[mygreen] at (3.2, 2.5) {$d$};
			\node[mygreen] at (4.8, 2.5) {$e$};
		\end{tikzpicture}
	\end{subfigure}
	\caption{An example of a rectangle visibility representation of a planar graph. The highlighted rectangle in the representation corresponds to the primal edge 13 and dual edge $cd$.}
\end{figure}

\begin{thm}
	Every planar vertex 2-connected graph has a rectangle visibility \\ representation.
	\label{thm-2}
\end{thm}

\begin{proof}
	We in fact describe an algorithm how such a representation can be constructed. The bonus is that the construction runs in polynomial time.
\end{proof}

\begin{algorithm}[!ht]
	\caption{RectangleVisibilityRep}
	\begin{algorithmic}[1]
		\Require A planar graph $G$.
		\State Construct an $s - t$ numbering and a corresponding upward planar drawing as in Theorem \ref{thm-1}.
		\State Number the vertices as $1, \dots, n$ according to their $y$-coordinates (1 is the lowest vertex, $n$ is the highest one).
		\State Add a dummy arc from 1 to $n$ drawn to the right of the drawing of $G$, denote by $G'$ the resulting graph.
		\State Construct the dual graph $G^\ast$ to $G'$, and orient its edges so that each edge of $G$ is crossed by its dual edge from left to right, while the added dummy edge is crossed from right to left. Let $n^\ast$ be the number of vertices of $G^\ast$.
		\State Consider a topological sorting of the vertices of $G^\ast$ and name them $A, B, \dots$ according to this sorting.
		\State Take a grid of size $n \times n^\ast$. For a vertex $i$ of $G$, let $\alpha(\beta)$ be the face incident with $i$ with the lowest (highest, respectively) name in the topological sorting of $G^\ast$. Represent $i$ by a horizontal segment on the $i$-th line, starting at the vertical line $\alpha$ and ending on the vertical line $\beta$. For a face $\alpha$ of the drawing of $G'$ (i.e., a vertex of the dual graph $G^\ast$), let $i$ be its lowest vertex and $j$ its highest vertex (with respect to the topological sorting of $G$). Represent $\alpha$ by a segment on the vertical line $\alpha$, starting on the $i$-th horizontal line and ending on the $j$-th one.
		\State \Return this representation.
	\end{algorithmic}
\end{algorithm}

It is, however, necessary to prove that this Algorithm really outputs a rectangle visibility representation of $G$. This is done via a series of claims. The first ones talk about the upward drawing of $G$ constructed in Step 1 and the dual graph $G^\ast$ and its drawing inherited from the drawing of $G$.

\begin{claim}
	Vertex number 1 and vertex number $n$ are both on the boundary of the outerface of $G$ (and also of $G'$). The face that gets the name $A$ is the unbounded face and the face with the highest name, say $Z$, is the other face incident with the dummy edge $1n$.
\end{claim}

\begin{claim}
	The orientation of $G^\ast$ described in Step 4 is acyclic and $A$ is its only source and $Z$ is its only sink. Hence it is indeed an $s - t$ numbering of $G^\ast$. To see this, note first that the edge $AZ$ which crosses the dummy edge $1n$ cannot be involved in any directed cycle, as $A$ is a source and $Z$ is a sink in $G^\ast$. Next observe that a clock-wise oriented cycle in $G^\ast$ would bound a region with at least one vertex of $G$ inside and all edges of $G$ crossing this cycle would be oriented from inside towards the outside of this region, hence, there would necessarily be a source of $G$ in this region, and this would be different from the vertex $1$. Similarly, a counter-clock-wise oriented cycle in $G^\ast$ would bound a region that would contain a sink different from $n$. This would be a contradiction with the assumption that we were working with an $s - t$-numbering of $G$. Finally, a source different from $A$ (a sink different from $Z$) in $G^\ast$ would yield a directed cycle on the boundary of this face of $G$. A contradiction again.
\end{claim}

\begin{claim}
	The boundary of every face $\alpha$ of $G$ consists of two directed paths, the left one and the right one, both connecting the vertex of the lowest number to the vertex of the highest number among the vertices of this face. We have seen this in the proof of Theorem \ref{thm-1}.
\end{claim}

\begin{claim}
	For every vertex $i$ of $G, i \neq 1, n$, the faces incident with $i$ induce two directed paths in $G^\ast$,
	both connecting the face to the left of $i$ to the face lying to the right of $i$, one of the paths passing
	through the faces for which $i$ is their topmost vertex, the other one passing through the faces for which
	$i$ is their lowest vertex. See Fig. \ref{fig-3} right.
	\label{claim-4}
\end{claim}

\begin{figure}[!ht]
	\begin{subfigure}{0.45\textwidth}\centering
		\begin{tikzpicture}
			\draw[->, color=mygreen, thick] (0,0) -- (2,0);
			\draw[->, color=myblue, thick] (1, -1) -- (1, 1);
		\end{tikzpicture}
		\caption{Orientation of the dual edges.}
	\end{subfigure}
	\begin{subfigure}{0.45\textwidth}\centering
		\begin{tikzpicture}
			\node[draw, circle, myblue, thick] (i) at (0,0) {$i$};
			\node[mygreen, thick] (a) at (-2, 0) {$\alpha$};
			\node[mygreen, thick] (w) at (2,0) {$\omega$};
			\draw[->, mygreen] (a) -- (-1, 1);
			\draw[->, mygreen] (-0.8,1) -- (0.5, 1);
			\draw[->, mygreen] (0.7, 1) -- (w);
			\draw[->, mygreen] (a) -- (-1.2, -1);
			\draw[->, mygreen] (-1, -1) -- (0.8, -1);
			\draw[->, mygreen] (1, -1) -- (w);
			\draw[->, myblue] (i) -- (-0.5, 2);
			\draw[->, myblue] (-0.5, -2) -- (i);
			\draw[->, myblue] (i) -- (-2.3, 1);
			\draw[->, myblue] (-2.3, -1) -- (i);
			\draw[->, myblue] (i) -- (2.3, 1);
			\draw[->, myblue] (2.3, -1) -- (i);
		\end{tikzpicture}
		\caption{An example to the statement of Claim \ref{claim-4}.}
	\end{subfigure}
	\caption{Examples for claims.}
	\label{fig-3}
\end{figure}

In the next claims we prove that the collection of segments constructed in Step 6 defines a rectangle visibility representation of $G$.

\begin{claim}
	Let $i$ be a vertex of $G, i \neq 1, n$. Let $\alpha$ and $\omega$ be the faces to the left and to the right of $i$, respectively. Then the (horizontal) segment $i$ touches the vertical segment for $\alpha$ from the right, it touches the vertical segment for $\beta$ from the left, it is touched by the segments representing the faces lying on the upper path from $\alpha$ to $\beta$ in $G^\ast$ from above, and it is touched by the faces lying on the lower path from $\alpha$ to $\beta$ in $G^\ast$ from below. The segment representing vertex $1$ spans the whole range from $A$ to $Z$ and is only touched by vertical segments from above, while the segment representing $n$ is only touched by vertical segments from below, and also spans the whole range from $A$ to $Z$.
	\label{claim-5}
\end{claim}

\begin{claim}
	Let $\alpha$ be an inner face of $G$, i.e., a face not incident with the dummy edge $1n$. The vertical segment representing $\alpha$ touches the horizontal segment representing its lowest vertex from above, it touches the segment representing its topmost vertex from below, it is touched by the segments representing the vertices on the left boundary path of $\alpha$ from the left and it is touched by the segments representing the vertices on the right boundary path of $\alpha$ from the right. The segment representing $A$ is touched only from right, while the segment representing $Z$ is touched only from left, always by the appropriate horizontal segments.
	\label{claim-6}
\end{claim}

\begin{figure}[!ht]
	\begin{subfigure}{.25\textwidth}\centering
		\begin{tikzpicture}
			\node[draw, circle, myblue, thick] (i) at (0,0) {$i$};
			\node[mygreen, thick] (a) at (-1, 0) {$\alpha$};
			\node[mygreen, thick] (w) at (1,0) {$\omega$};
			\draw[->, mygreen] (a) -- (-0.5, 0.5);
			\draw[->, mygreen] (-0.3,0.5) -- (0.25, 0.5);
			\draw[->, mygreen] (0.35, .5) -- (w);
			\draw[->, mygreen] (a) -- (-.6, -.5);
			\draw[->, mygreen] (-.5, -.5) -- (0.4, -.5);
			\draw[->, mygreen] (.5, -.5) -- (w);
			\draw[->, myblue] (i) -- (-0.25, 1);
			\draw[->, myblue] (-0.25, -1) -- (i);
			\draw[->, myblue] (i) -- (-1.15, .5);
			\draw[->, myblue] (-1.15, -.5) -- (i);
			\draw[->, myblue] (i) -- (1.15, .8);
			\draw[->, myblue] (1.15, -.8) -- (i);
		\end{tikzpicture}
	\end{subfigure}
	\begin{subfigure}{.2\textwidth}\centering
		\begin{tikzpicture}
			\draw[thick, myblue] (0,0) -- (3,0);
			\draw[thick, mygreen] (0,-1) -- (0,1);
			\draw[thick, mygreen] (3,-1) -- (3,1);
			\draw[thick, mygreen] (.8,0) -- (.8,1);
			\draw[thick, mygreen] (1.9,0) -- (1.9,1);
			\draw[thick, mygreen] (1.4,-1) -- (1.4,0);
			\draw[thick, mygreen] (2.3,-1) -- (2.3,0);
			\node[mygreen] at (.2, -1.3) {$\alpha$};
			\node[mygreen] at (2.8, -1.3) {$\omega$};
		\end{tikzpicture}
	\end{subfigure}
	\begin{subfigure}{.25\textwidth}\centering
		\begin{tikzpicture}
			\draw[->, thick, myblue] (0,0) -- (-1, .5);
			\draw[->, thick, myblue] (-1, .5) -- (-.3, 1);
			\draw[->, thick, myblue] (-.3, 1) -- (-1, 1.5);
			\draw[->, thick, myblue] (-1, 1.5) -- (0, 2);
			\draw[->, thick, myblue] (0,0) -- (1, .5);
			\draw[->, thick, myblue] (1, .5) -- (.7, 1.2);
			\draw[->, thick, myblue] (.7, 1.2) -- (.5, 1.6);
			\draw[->, thick, myblue] (.5, 1.6) -- (0,2);
			\node[mygreen, thick] at (0,1) {$\alpha$};
		\end{tikzpicture}
	\end{subfigure}
	\begin{subfigure}{.2\textwidth}\centering
		\begin{tikzpicture}
			\draw[mygreen, thick] (0,0) -- (0,2);
			\draw[myblue, thick] (-1,0) -- (1,0);
			\draw[myblue, thick] (-1,2) -- (1,2);
			\draw[myblue, thick] (-1,.3) -- (0,.3);
			\draw[myblue, thick] (-1,.8) -- (0,.8);
			\draw[myblue, thick] (-1,1.4) -- (0,1.4);
			\draw[myblue, thick] (0,.4) -- (1,.4);
			\draw[myblue, thick] (0,1.5) -- (1,1.5);
			\node[mygreen, thick] at (0.2, 1) {$\alpha$};
		\end{tikzpicture}
	\end{subfigure}
	\caption{Illustration to Claims \ref{claim-5} and \ref{claim-6}.}
\end{figure}

\begin{claim}
	For every edge $ij$ of $G$, let $\alpha\beta$ be its dual edge. Then the segments representing $i, j, \alpha$ and $\beta$ bound a rectangle in the representation.
	\label{claim-7}
\end{claim}

\begin{claim}
	No two segments constructed in Step 6 cross each other. For suppose segment $j$ crosses segment $\alpha$. By the way the segments are constructed, this means that there exist vertices $i, k, i < j < k$, and $\beta, \gamma, \beta < \alpha < \gamma$, such that $i$ is the lowest and $k$ the topmost vertex of face $\alpha$ and $\beta$ is the left and $\gamma$ the right face incident with $j$. Consider the upward drawing of $G$ and the horizontal stripe of vertices with their $y$-coordinates between $i$ and $k$. The face $\alpha$ spans this stripe from its bottom to its top lines, and thus it lies either to the left or to the right of vertex $i$. Suppose it is to the left. The horizontal ray starting in vertex $i$ and pointing to the left passes first through the face $\beta$ and then crosses several faces until it finally crosses $\alpha$. Since all edges of $G$ it crosses on this way are directed upward, these faces form a path directed from $\alpha$ to $\beta$ in $G^\ast$, which means that $\alpha < \beta$ in the topological sorting of $G^\ast$. Which is a contradiction.
	\label{claim-8}
\end{claim}

\begin{figure}
	\begin{subfigure}{.45\textwidth}\centering
		\begin{tikzpicture}
			\draw[thick, mygreen] (0,0) -- (0,3);
			\draw[thick, myblue] (-1.5,1.5) -- (1.5, 1.5);
			\draw[dotted] (-1.5, -.2) -- (-1.5, 3.2);
			\draw[dotted] (1.5, -.2) -- (1.5, 3.2);
			\draw[dotted] (-1.7, 0) -- (1.7, 0);
			\draw[dotted] (-1.7, 3) -- (1.7, 3);
			
			\node[mygreen] at (.3, .2) {$\alpha$};
			\node[myblue] at (1.2, 1.8) {$j$};
			\node[myblue] at (1.9, 0) {$i$};
			\node[myblue] at (1.9, 3) {$k$};
			\node[mygreen] at (-1.2, -.3) {$\beta$};
			\node[mygreen] at (1.2, -.3) {$\gamma$};
		\end{tikzpicture}
	\end{subfigure}
	\begin{subfigure}{.45\textwidth}\centering
		\begin{tikzpicture}
			\draw[dotted] (0,0) -- (7,0);
			\draw[dotted] (0,2) -- (7,2);
			\draw[dotted] (0,-2) -- (7,-2);
			\node[mygreen] (a) at (1, 0.2) {$\alpha$};
			\draw[->, thick, mygreen] (a) -- (2, -.4);
			\draw[->, thick, mygreen] (2, -.4) -- (3, .5);
			\draw[->, thick, mygreen] (3, .5) -- (5, -.2);
			\node[mygreen] (b) at (5.3, -.3) {$\beta$};
			\node[draw, circle, myblue, fill] (j) at (6, 0) {\textcolor{white}{$j$}};
			\draw[->, myblue, thick] (5.5, -1) -- (j);
			\draw[->, myblue, thick] (j) -- (5.4, 1);
			\draw[->, thick, myblue] (3.7, -1) -- (4.3,1);
			\draw[->, thick, myblue] (3.2, -1) -- (2.3,1);
			\draw[->, thick, myblue] (1.5, -1) -- (1.8,1);
			\draw[->, thick, myblue] (1,-2) -- (1.5, -1);
			\draw[->, thick, myblue] (1.8, 1) -- (1,2);
			\draw[->, thick, myblue] (1,-2) -- (.3, .8);
			\draw[->, thick, myblue] (.3, .8) -- (1,2);
		\end{tikzpicture}
	\end{subfigure}
	\caption{An illustration to Claim \ref{claim-8}.}
\end{figure}

\begin{claim}
	The rectangles from Claim \ref{claim-7} are disjoint and fill in the base rectangle formed by segments $A, Z, 1, n$. This now follows from the previous claims. And this means that we have indeed constructed a rectangle visibility representation of $G$.
\end{claim}

\begin{figure}[!ht]
	\begin{subfigure}{0.53\textwidth}\centering
		\begin{tikzpicture}[be/.style = {thick, myblue, ->}, ge/.style = {thick, mygreen, ->},
			bn/.style = {myblue, thick, circle, draw}, gn/.style = {mygreen, thick, circle, draw}]
			\node[bn] (1) at (0,0) {1};
			\node[bn] (2) at (-1.7, 1.7) {2};
			\node[bn] (3) at (.2, 3.2) {3};
			\node[bn] (4) at (-3, 3) {4};
			\node[bn] (5) at (1.8, 4) {5};
			\node[bn] (6) at (-1.3, 5) {6};
			\node[bn] (7) at (-.5, 6) {7};
			\node[gn] (A) at (-4, 4) {A};
			\node[gn] (B) at (-2.1, 3.2) {B};
			\node[gn] (C) at (-.8, 3.6) {C};
			\node[gn] (D) at (-.6, 1.7) {D};
			\node[gn] (E) at (1, 3.6) {E};
			\node[gn] (Z) at (1.6, 5.6) {Z};
			\draw[be] (1) -- (2);
			\draw[be] (1) -- (3);
			\draw[be] (1) -- (5);
			\draw[be] (2) -- (4);
			\draw[be] (2) -- (6);
			\draw[be] (2) -- (3);
			\draw[be] (5) -- (7);
			\draw[be] (3) -- (7);
			\draw[be] (6) -- (7);
			\draw[be] (4) -- (6);
			\draw[ge, bend left = 20] (A) edge (B);
			\draw[ge] (B) -- (C);
			\draw[ge] (C) -- (D);
			\draw[ge] (E) -- (Z);
			\draw[ge, bend right = 20] (D) edge (E);
			\draw[ge, bend left = 20] (C) edge (E);
			\draw[ge] (A) to[out=290, in=90] (-4, 2) to[out=270, in=300] (B);
			\draw[ge] (A) to[out=270, in=90] (-4.3, 1) to[out=270, in=280] (D);
			\draw[ge] (A) to[out=30, in=90] (-1, 6) to[out=270, in=50] (C);
			\draw[ge] (E) to[out=270, in=270] (2.5, 4) to[out=90, in=300] (Z);
			\draw[ge, dashed] (A) to[out=50, in=180] (0, 7.5) to[out=0, in=130] (Z);
			\draw[be, dashed] (1) to[out=20, in=270] (3, 4) to[out=90, in=50] (7);
		\end{tikzpicture}
	\end{subfigure}
	\begin{subfigure}{0.45\textwidth}\centering
		\begin{tikzpicture}[b/.style = {line width = 4, myblue}, g/.style = {line width = 4, mygreen}]
			\draw (1,1) -- (1,7);
			\draw (2,1) -- (2,7);
			\draw (3,1) -- (3,7);
			\draw (4,1) -- (4,7);
			\draw (5,1) -- (5,7);
			\draw (6,1) -- (6,7);
			\draw (1,1) -- (6,1);
			\draw (1,2) -- (6,2);
			\draw (1,3) -- (6,3);
			\draw (1,4) -- (6,4);
			\draw (1,5) -- (6,5);
			\draw (1,6) -- (6,6);
			\draw (1,7) -- (6,7);
			\draw[b] (1,1) -- (6,1);
			\draw[b] (1,2) -- (4,2);
			\draw[b] (3,3) -- (5,3);
			\draw[b] (1,4) -- (2,4);
			\draw[b] (5,5) -- (6,5);
			\draw[b] (1,6) -- (3,6);
			\draw[b] (1,7) -- (6,7);
			\draw[g] (1,1) -- (1,7);
			\draw[g] (2,2) -- (2,6);
			\draw[g] (3,2) -- (3,7);
			\draw[g] (4,1) -- (4,3);
			\draw[g] (5,1) -- (5,7);
			\draw[g] (6,1) -- (6,7);
			\node at (.5, 1) {1};
			\node at (.5, 2) {2};
			\node at (.5, 3) {3};
			\node at (.5, 4) {4};
			\node at (.5, 5) {5};
			\node at (.5, 6) {6};
			\node at (.5, 7) {7};
			\node at (1,.5) {A};
			\node at (2,.5) {B};
			\node at (3,.5) {C};
			\node at (4,.5) {D};
			\node at (5,.5) {E};
			\node at (6,.5) {Z};
		\end{tikzpicture}
	\end{subfigure}
	\caption{An overview of the construction by Algorithm RectangleVisibilityRep.}
\end{figure}

\section{Grid Contact graphs}

\begin{defn}
	A graph is a \textbf{Grid Intersection graph} if it has an intersection \\ representation by vertical and horizontal segments in which no two segments of the same direction share a point (in other words, all vertical, as well as all horizontal, segments are pairwise disjoint). A graph is a \textbf{Grid Contact} graph if it has a Grid Intersection representation in which no two segments cross, i.e., any two non-disjoint segments only touch each other.
\end{defn}

\begin{prop}
	Every grid intersection graph is bipartite. Moreover, every grid contact graph is planar.
\end{prop}

\begin{proof}
	The first claim is a simple observation. For the second claim, note that grid contact graphs form a subclass of triangle-free contact graphs of arc-connected regions in the plane. All such graphs are planar, since a non-crossing drawing can be constructed from a contact representation by selecting a point inside each region to represent its vertex, and connecting it to the contact points with the adjacent regions by curves inside the region.
\end{proof}

\begin{thm}
	Every planar bipartite graph is a grid contact graph.
	\label{thm-3}
\end{thm}

\begin{proof}
	Given a planar bipartite graph $G = (A \cup B, E)$, consider a non-crossing drawing and extend it to a non-crossing drawing of a supergraph $G' = (A' \cup B' , E')$ such that
	
	\begin{itemize}
		\item $G$ is an induced subgraph of $G'$,
		\item every face of the drawing of $G'$ is bounded by a cycle of length 4 (i.e., $G'$ is a so called quadrangulation),
		\item every vertex of $G'$ has degree greater than $2$, and
		\item no vertex of $B$ is on the boundary of the outerface of $G'$. 
	\end{itemize}
	
	Then construct the graph $\bar{G} = (A', \bar{E})$ by putting $\bar{E}$ the diagonals of the faces of $G'$ connecting their $A'$-vertices. It can be easily seen that $G$ is vertex 2-connected and that the faces of $\bar{G}$ are in 1-1 correspondence with the vertices of $B'$. Thus the segments of a rectangle visibility representation of $\bar{G}$ constructed as in the proof of Theorem \ref{thm-2} form a grid contact representation of $G'$ . Note the technical detail that the vertex (of $B'$) that corresponds to the outerface of $\bar{G}$ is represented by two vertical segments, not one. But this vertex does not belong to $B$, and so the segments corresponding to the vertices of $G$ form a grid contact representation of $G$.
\end{proof}

\begin{figure}[!ht]
	\begin{subfigure}{0.45\textwidth}\centering
		\begin{tikzpicture}[main/.style = {draw, circle, thick, myblue},
			edge/.style = {thick, myblue},
			fil/.style = {thick, fill, draw, circle, myblue}]
			\node[main] (1) at (0,0) {};
			\node[fil] (2) at (1.8, 2.3) {};
			\node[fil] (3) at (-3, 1) {};
			\node[main] (4) at (-1, 1.2) {};
			\node[fil] (5) at (-1.2, 3) {};
			\node[fil] (6) at (-2.8, 2.7) {};
			\node[main] (7) at (-4, 4) {};
			\node[main] (8) at (-2, 4) {};
			\node[main] (9) at (0, 4) {};
			\node[fil] (10) at (-1.5, 6) {};
			\draw[edge] (1) -- (2);
			\draw[edge] (1) -- (3);
			\draw[edge] (3) -- (4);
			\draw[edge] (2) -- (4);
			\draw[edge] (4) -- (5);
			\draw[edge] (4) -- (6);
			\draw[edge] (2) -- (9);
			\draw[edge] (9) -- (10);
			\draw[edge] (8) -- (10);
			\draw[edge] (7) -- (10);
			\draw[edge] (3) -- (7);
			\draw[edge] (6) -- (7);
			\draw[edge] (5) -- (8);
			\draw[edge] (6) -- (8);
			\draw[edge] (5) -- (9);
			\draw[edge] (1) to[out=0, in=270] (2.5, 2.3) to[out=90, in=0] (10);
		\end{tikzpicture}
		\caption{The graph $G'$.}
	\end{subfigure}
	\begin{subfigure}{0.53\textwidth}\centering
		\begin{tikzpicture}[main/.style = {draw, circle, thick, myblue!40},
			edge/.style = {thick, myblue!40},
			fil/.style = {thick, fill, draw, circle, myblue!40},
			red/.style = {thick, myred}]
			\node[main] (1) at (0,0) {};
			\node[fil] (2) at (1.8, 2.3) {};
			\node[fil] (3) at (-3, 1) {};
			\node[main] (4) at (-1, 1.2) {};
			\node[fil] (5) at (-1.2, 3) {};
			\node[fil] (6) at (-2.8, 2.7) {};
			\node[main] (7) at (-4, 4) {};
			\node[main] (8) at (-2, 4) {};
			\node[main] (9) at (0, 4) {};
			\node[fil] (10) at (-1.5, 6) {};
			\draw[edge] (1) -- (2);
			\draw[edge] (1) -- (3);
			\draw[edge] (3) -- (4);
			\draw[edge] (2) -- (4);
			\draw[edge] (4) -- (5);
			\draw[edge] (4) -- (6);
			\draw[edge] (2) -- (9);
			\draw[edge] (9) -- (10);
			\draw[edge] (8) -- (10);
			\draw[edge] (7) -- (10);
			\draw[edge] (3) -- (7);
			\draw[edge] (6) -- (7);
			\draw[edge] (5) -- (8);
			\draw[edge] (6) -- (8);
			\draw[edge] (5) -- (9);
			\draw[edge] (1) to[out=0, in=270] (2.5, 2.3) to[out=90, in=0] (10);
			\draw[red] (1) edge (4);
			\draw[red, bend right = 30] (7) edge (4);
			\draw[red, bend left = 90] (1) edge (7);
			\draw[red, bend right = 20] (8) edge (4);
			\draw[red] (4) edge (9);
			\draw[red] (8) edge (9);
			\draw[red] (1) to[out=10, in=270] (2.35, 2.3) to[out=90, in=0] (9);
			\draw[red] (7) edge (8);
		\end{tikzpicture}
		\caption{The graph $G$.}
	\end{subfigure}
	\caption{An illustration to the proof of Theorem \ref{thm-3}.}
\end{figure}
	\chapter{Schnyder woods}

\section{Canonical ordering}

\begin{defn}
	Given a plane triangulation (i.e., a non-crossing embedding of a triangulation in the plane, which means that the outerface is fixed) $G = (V, E)$, a canonical ordering of it is a numbering $1, 2, \dots, n$ of its vertices such that
	
	\begin{enumerate}[i)]
		\item the vertices of the outerface are $1, 2$ and $n = |V|$,
		\item for every $i = 3, 4, \dots, n - 1$,
		\begin{enumerate}[a)]
			\item the graph $G_i = G[1, 2, \dots, i]$ is vertex 2-connected,
			\item all vertices $j \leq i$ are drawn inside (or on the boundary) of the embedding of $G_i$ inherited from the embedding of $G$,
			\item all vertices $j > i$ are drawn outside the embedding of $G_i$ inherited from the embedding of $G$,
			\item the neighbors of vertex $i + 1$ in $G_i$ are lying consecutively on the boundary of $G_i$.
		\end{enumerate}
	\end{enumerate}
\end{defn}

\begin{thm}
	Every triangulation has a canonical ordering.
\end{thm}

\begin{proof}
	By induction from $n$ downto $3$, assign the numbers to the vertices. Once $n, n - 1, \dots, i + 1$ are assigned, choose as vertex $i$ such a vertex on the boundary of $G_i$ whose deletion from $G_i$ leaves $G_{i-1}$ vertex 2-connected. The only obstacle to preserving 2-connectedness is if $i$ would be incident to a diagonal edge (an edge with both end-vertices on the boundary of the outerface, but which itself is not a part of the boundary). But there is always a vertex which is not incident with any diagonal (to observe this, consider a vertex which is incident with a shortest possible diagonal, with the length of a diagonal being measured by the number of vertices of the boundary that it cuts off of $G_i$). So assign $i$ to a vertex (there may be more options) which is not incident to any diagonal. Then a), b) and c) are fulfilled for $G_{i-1}$. Note that d) follows from b) and c).
\end{proof}

\section{Schnyder woods}

\begin{algorithm}[!ht]
	\caption{Schnyder}
	\begin{algorithmic}[1]
		\Require A plane triangulation $G = (V, E)$ and a canonical ordering of it.
		\For{$i := 3 \dots n -1$}
			\State set $b(i)$ to be the leftmost neighbor of $i$ on the boundary of $G_{i-1}$, direct the edge $ib(i)$ in this direction and color it blue;
			\State set $g(i)$ to be the rightmost neighbor of $i$ on the boundary of $G_{i-1}$, direct the edge $ig(i)$ in this direction and color it green;
			\State set $r(i)$ to be the neighbor of $i$ with the highest number, direct the edge $ig(i)$ in this direction and color it red
		\EndFor
		\State \Return the orientation of $G$, the coloring of its edges and the mappings $b, g$ and $r$.
	\end{algorithmic}
\end{algorithm}

\begin{figure}[!ht]\centering
	\begin{subfigure}{0.45\textwidth}\centering
		\begin{tikzpicture}[main/.style = {draw, circle, thick}]
			\node[main] (1) at (-1,-1) {1};
			\node[main] (2) at (5,-1) {2};
			\node[main] (3) at (1,1) {3};
			\node[main] (4) at (3, 1.2) {4};
			\node[main] (5) at (2.7, 2.4) {5};
			\node[main] (6) at (1.6, 2.9) {6};
			\node[main] (7) at (2, 6) {7};
			\draw[thick] (1) edge (3);
			\draw[thick] (1) edge (2);
			\draw[thick] (1) edge (7);
			\draw[thick] (1) edge (6);
			\draw[thick] (6) edge (3);
			\draw[thick] (4) edge (3);
			\draw[thick] (5) edge (3);
			\draw[thick] (2) edge (3);
			\draw[thick] (2) edge (4);
			\draw[thick] (2) edge (5);
			\draw[thick] (2) edge (7);
			\draw[thick] (5) edge (4);
			\draw[thick] (5) edge (6);
			\draw[thick] (7) edge (6);
			\draw[thick] (7) edge (5);
		\end{tikzpicture}
	\end{subfigure}
	\begin{subfigure}{0.45\textwidth}\centering
		\begin{tikzpicture}[main/.style = {draw, circle, thick}]
			\node[main] (1) at (-1,-1) {1};
			\node[main] (2) at (5,-1) {2};
			\node[main] (3) at (1,1) {3};
			\node[main] (4) at (3, 1.2) {4};
			\node[main] (5) at (2.7, 2.4) {5};
			\node[main] (6) at (1.6, 2.9) {6};
			\node[main] (7) at (2, 6) {7};
			\draw[thick, ->, color=myblue] (3) edge (1);
			\draw[thick] (1) edge (2);
			\draw[thick] (1) edge (7);
			\draw[thick, ->, color=myblue] (6) edge (1);
			\draw[thick, ->, color=myred] (3) edge (6);
			\draw[thick, ->, color=myblue] (4) edge (3);
			\draw[thick, ->, color=myblue] (5) edge (3);
			\draw[thick, ->, color=mygreen] (3) edge (2);
			\draw[thick, ->, color=mygreen] (4) edge (2);
			\draw[thick, ->, color=mygreen] (5) edge (2);
			\draw[thick] (2) edge (7);
			\draw[thick, ->, color=myred] (5) edge (4);
			\draw[thick, ->, color=mygreen] (6) edge (5);
			\draw[thick, ->, color=myred] (6) edge (7);
			\draw[thick, ->, color=myred] (5) edge (7);
		\end{tikzpicture}
	\end{subfigure}
	\caption{An illustration to canonical orderings and Schnyder woods.}
\end{figure}

\begin{thm}
	The blue edges form a tree rooted in vertex 1 and spanning the vertices $1, 3, \dots, n - 1$, the green edges form a tree rooted in vertex 2 and spanning the vertices $2, 3, \dots, n - 1$, and the red edges form a tree rooted in vertex n and spanning the vertices $3, \dots , n$. Every edge of $G$, except of $12, 1n, 2n$, belongs to exactly one of these trees.
\end{thm}

\begin{proof}
	Clear from the construction and properties of canonical orderings.
\end{proof}

\begin{cor}
	The edges of a planar triangulation can be partitioned into edge sets of 3 trees and a triangle. Such a collection of three trees is called a \textbf{Schnyder wood} of $G$.
\end{cor}

\begin{figure}[!ht]\centering
	\begin{tikzpicture}[main/.style = {draw, circle, thick}]
		\node[main] (1) at (-1,-1) {1};
		\node[main] (2) at (5,-1) {2};
		\node[main, fill] (3) at (.8,1.3) {};
		\node[main, fill] (4) at (3.2, 1.2) {};
		\node[main, fill] (5) at (2.4, 1) {};
		\node[main, fill] (6) at (1.6, 0.6) {};
		\node[main] (n) at (2, 6) {$n$};
		\node[main] (i) at (2, 3) {$i$};
		\node[main, fill] (7) at (2, 4.3) {};
		\node (G) at (2, 0) {$G_{i-1}$};
		\draw[thick] (1) edge (3);
		\draw[thick] (1) edge (2);
		\draw[thick] (1) edge (n);
		\draw[thick] (6) edge (3);
		\draw[thick] (2) edge (4);
		\draw[thick] (2) edge (n);
		\draw[thick] (5) edge (4);
		\draw[thick] (5) edge (6);		
		\draw[thick, color=myblue, ->] (i) edge (3);
		\draw[thick, color=myred, ->] (6) edge (i);
		\draw[thick, color=myred, ->] (5) edge (i);
		\draw[thick, color=mygreen, ->] (i) edge (4);
		\draw[thick, color=myred, ->] (i) edge (7);
		\draw[thick, color=mygreen, ->] (1, 3.2) -- (i);
		\draw[thick, color=mygreen, ->] (1, 2.8) -- (i);
		\draw[thick, color=myblue, ->] (i) -- (2.7, 3.4);
		\draw[thick, color=myblue, ->] (i) -- (2.7, 2.8);
		\draw[thick, color=myblue, ->] (i) -- (2.7, 3);
	\end{tikzpicture}
	\caption{The rotation scheme of incoming and outgoing edges around a vertex in the Schnyder wood.}
\end{figure}

\begin{prop}
	Locally around every inner vertex of $G$, we see (in the counter-clock-wise order) one outgoing blue edge, several (or none) incoming red edges, one outgoing green edge, several (or none) incoming blue edges, one outgoing red edge, and several (or none) incoming green edges.
\end{prop}

\section{Triangle contact representations}

\begin{thm}[de Fraysseix, Ossona de Mendez, Rosenstiehl]
	Every planar graph is a contact graph of isosceles triangles with horizontal bases.
\end{thm}

\begin{proof}
	It suffices to prove the theorem for triangulations, since every planar graph is an induced subgraph of a triangulation. Given a triangulation $G = (V, E)$, fix an embedding and consider a canonical ordering with respect to this embedding, and run algorithm Schnyder on this canonical ordering. Draw $n + 1 = |V| + 1$ parallel horizontal lines and build triangles as follows:
	
	\begin{enumerate}
		\item Triangle $T_i$ is isosceles and its base lies on the $i$-th line,
		\item The peaks of $T_1$ and $T_2$ are on the $(n + 1)$-st line, the left corner of $T_2$ touches the right side of $T_1$,
		\item For every $i = 1, 2, \dots, n$, the left corner of $T_i$ lies on the right side of $T_{b(i)}$, the right corner of $T_i$ lies on the left side of $T_{g(i)}$ and the peak of $T_i$ lies on the $r(i)$-th line (on the $(n + 1)$-st line for $i = n$).
	\end{enumerate}
	
	Construct the triangles from $T_1$ to $T_n$. For $T_1$, only the lines supporting its base and peak are prescribed, the triangle is free otherwise. For $T_2$, the freedom is restricted only to the position of the right corner (which then determines the position of the peak). For $i > 2$, the triangles are then determined uniquely. The loop invariant of this inductive construction is that for $i > 1$, the upper boundary of the union of triangles $T_1 , T_2 , \dots, T_i$ is connected and the order in which the triangles appear on this boundary is the same as the order of the corresponding vertices appearing on the upper boundary of $G_i$. This implies that $T_i$ is always placed in the way that it is touching the respective neighbors, but not crossing any triangle of the representation.
\end{proof}

\begin{figure}
	\begin{subfigure}{0.45\textwidth}\centering
		\begin{tikzpicture}[main/.style = {draw, circle, thick}]
			\node[main] (1) at (-1,-1) {1};
			\node[main] (2) at (5,-1) {2};
			\node[main] (3) at (1,1) {3};
			\node[main] (4) at (3, 1.2) {4};
			\node[main] (5) at (2.7, 2.4) {5};
			\node[main] (6) at (1.6, 2.9) {6};
			\node[main] (7) at (2, 6) {7};
			\draw[thick, ->, color=myblue] (3) edge (1);
			\draw[thick] (1) edge (2);
			\draw[thick] (1) edge (7);
			\draw[thick, ->, color=myblue] (6) edge (1);
			\draw[thick, ->, color=myred] (3) edge (6);
			\draw[thick, ->, color=myblue] (4) edge (3);
			\draw[thick, ->, color=myblue] (5) edge (3);
			\draw[thick, ->, color=mygreen] (3) edge (2);
			\draw[thick, ->, color=mygreen] (4) edge (2);
			\draw[thick, ->, color=mygreen] (5) edge (2);
			\draw[thick] (2) edge (7);
			\draw[thick, ->, color=myred] (5) edge (4);
			\draw[thick, ->, color=mygreen] (6) edge (5);
			\draw[thick, ->, color=myred] (6) edge (7);
			\draw[thick, ->, color=myred] (5) edge (7);
		\end{tikzpicture}
	\end{subfigure}
	\begin{subfigure}{0.45\textwidth}\centering
		\begin{tikzpicture}
			\draw[myblue, thick] (0,0) -- (7,0);
			\draw[myblue, thick] (0,1) -- (7,1);
			\draw[myblue, thick] (0,2) -- (7,2);
			\draw[myblue, thick] (0,3) -- (7,3);
			\draw[myblue, thick] (0,4) -- (7,4);
			\draw[myblue, thick] (0,5) -- (7,5);
			\draw[myblue, thick] (0,6) -- (7,6);
			\draw[myblue, thick] (0,7) -- (7,7);
			\draw[myblue, fill=myblue!20] (0,0) -- (1,0) -- (0.5,6) -- cycle;
			\draw[myblue, fill=myblue!20] (0.925,1) -- (7,1) -- (3.9625,6) -- cycle;
			\draw[myblue, fill=myblue!20] (0.85,2) -- (1.525,2) -- (1.1875,5) -- cycle;
			\draw[myblue, fill=myblue!20] (1.425,3) -- (2.125,3) -- (1.775,4) -- cycle;
			\draw[myblue, fill=myblue!20] (1.3,4) -- (2.75,4) -- (2.025,6) -- cycle;
			\draw[myblue, fill=myblue!20] (0.59,5) -- (1.65,5) -- (1.12,6) -- cycle;
			\draw[myblue, fill=myblue!20] (0,6) -- (7,6) -- (3.5,7) -- cycle;
			\node at (.5, .2) {$T_1$};
			\node at (3.9625, 1.2) {$T_2$};
			\node at (1.1875, 2.2) {$T_3$};
			\node at (1.775, 3.2) {$T_4$};
			\node at (2.025, 4.2) {$T_5$};
			\node at (1.12, 5.2) {$T_6$};
			\node at (3.5, 6.2) {$T_7$};
		\end{tikzpicture}
	\end{subfigure}
	\caption{An illustration to contact representations by isosceles triangles.}
\end{figure}

\section{Drawing planar graphs on small grids}

\begin{thm}
	Every planar $n$-vertex graph allows a straight-line non-crossing embedding on a grid of size $n \times n$.
\end{thm}

\begin{proof}
	It suffices to prove the theorem for planar triangulations. Given a triangulation $G = (V, E)$, fix an embedding and consider a canonical ordering with respect to this embedding, and run algorithm Schnyder on this canonical ordering. Assign barycentric coordinates $(x_i , y_i , z_i)$ to every vertex $i = 3, 4, \dots, n - 1$ as follows (see Fig. \ref{coord} for illustration): The triangle $12n$ is divided into three regions by the blue, green and red directed paths from $i$ to the roots of the trees in the Schnyder wood. Let $x_i$ be the number of vertices in the region bounded by the blue and green paths and the side $12$, with the vertices on the green path being counted in, but not the vertices of the blue path. Similarly, $y_i$ is the number of vertices in the region bounded by the blue and red paths and the side $1n$, with the vertices on the blue path being counted in, but not the vertices of the red path, and $z_i$ is the number of vertices in the region bounded by the red and green paths and the side $n2$, with the vertices on the red path being counted in, but not the vertices of the green one. Vertex $i$ itself is not counted in neither of the regions.

	\begin{figure}[!ht]\centering
		\begin{tikzpicture}[main/.style = {draw, circle, thick}]
			\node[main, myblue] (1) at (-1,-1) {1};
			\node[main, mygreen] (2) at (5,-1) {2};
			\node[main, mygreen] (3) at (1,1) {3};
			\node[main, mygreen] (4) at (3, 1.2) {4};
			\node[main, mygreen] (5) at (2.7, 2.4) {5};
			\node[main, fill, Black] (6) at (1.6, 2.9) {\textcolor{white}{6}};
			\node[main, myred] (7) at (2, 6) {7};
			\draw[thick] (1) edge (3);
			\draw[thick] (1) edge (2);
			\draw[thick] (1) edge (7);
			\draw[thick, color=myblue, ->] (6) edge (1);
			\draw[thick] (6) edge (3);
			\draw[thick] (4) edge (3);
			\draw[thick] (5) edge (3);
			\draw[thick] (2) edge (3);
			\draw[thick] (2) edge (4);
			\draw[thick, color=mygreen, ->] (5) edge (2);
			\draw[thick] (2) edge (7);
			\draw[thick] (5) edge (4);
			\draw[thick, color=mygreen, ->] (6) edge (5);
			\draw[thick, color=myred, ->] (6) edge (7);
			\draw[thick] (7) edge (5);
		\end{tikzpicture}
		\caption{An illustration to the definition of barycentric coordinates from Schnyder woods. The green vertices are counted for the definition of $x_6$, the blue one for the definition of $y_6$ and the red one for the definition of $z_6$. Vertex 6 itself does not contribute to any of the coordinates.}
		\label{coord}
	\end{figure}

	Then every vertex except of $i$ belongs to exactly one of the regions, end hence
	
	$$
	x_i + y_i + z_i = n - 1
	$$
	
	for every $i$. For $i = 1, 2$, and $n$, we set
	
	$$
	\begin{array}{lll}
		x_1 = 0,   &y_1 = 0,   &z_1 = n - 1\\
		x_2 = 0,   &y_2 = n-1, &z_2 = 0\\
		x_n = n-1, &y_n = 0,   & z_n = 0 	
	\end{array}
	$$
	
	Place vertex $i$ in the point with barycentric coordinates $(x_i , y_i , z_i)$ in a triangular $n \times n \times n$ grid (the lines of the grid have coordinates $0, 1, \dots, n - 1$). Draw the edges of $G$ as straight-line segments
	connecting their vertices.
	
	\begin{claim}
		Consider a vertex $i \in \{3, 4, \dots, n - 1\}$. In the drawing constructed as above, the blue edge $ib(i)$ is directed into the bottom-left sextant of $i$, the edge $ig(i)$ is directed into the bottom-right sextant of $i$, and the edge $ir(i)$ is directed into the top sextant of $i$.
		\label{claim-1}
	\end{claim}
	
	\begin{proof}[Proof of claim]
		From the definition of the coordinates, it follows that $xb(i) \leq x_i$, $yb(i) < y_i$ and $zb(i) \geq z_i$, and this implies the direction of the edge $ib(i)$. Similarly for the others.
	\end{proof}
	
	\begin{claim}
		The rotation scheme of the edges around each vertex $i$ in the barycentric drawing is the same as the rotation scheme of the edges incident to the same vertex in $G$.
	\end{claim}
	
	\begin{proof}[Proof of claim]
		This follows by application of Claim \ref{claim-1} to the other end-vertices of the edges directed into $i$.
	\end{proof}
	
	\begin{claim}
		The barycentric drawing is non-crossing and topologically equivalent to the plane embedding of $G$ we started with.
	\end{claim}
	
	\begin{proof}[Proof of claim]
		If the rotation schemes for all vertices of drawings of two 3-connected graphs are the same and one of the drawings is non-crossing, then so is the other one, and the drawings are topologically equivalent.
	\end{proof}
\end{proof}

\section{Boxicity of graphs}

\begin{defn}
	The \textbf{boxicity} of a graph $G$, denoted by $\text{box}(G)$, is the smallest integer $d$ such that $G$ is an intersection graph of boxes in $\R^d$ (i.e., of $d$-dimensional intervals in the $d$-dimensional Euclidean space).
\end{defn}

\begin{prop}
For every graph $G = (V, E)$, its boxicity is a correctly defined finite number. It equals the minimum number of interval graphs whose intersection is equal to $G$, i.e., the minimum number $d$ for which sets $E_i \subseteq \binom{V}{2} , i = 1, 2, \dots, d$ exist, such that $E = \cup_{i=1}^d E_i$ and $(V, E_i)$ is an interval graph for every $i = 1, 2, \dots, d$.
\end{prop}

\begin{proof}
	An exercise. Proof the claim for graphs $G$ which are not complete. For a complete graph, the boxicity is $0$, which corresponds to the fact that the intersection of an empty set of subsets of a ground set ($E$, in this case) is by default set to be equal to the ground set itself.
\end{proof}

\section{Grid intersection graphs}

\begin{defn}
	A graph is a \textbf{Grid Intersection graph} if it has an intersection representation by vertical and horizontal segments in which no two segments of the same direction share a point (in other words, all vertical, as well as all horizontal, segments are pairwise disjoint).
\end{defn}

It is easy to observe that every grid intersection graph has a grid intersection representation in which no two segments lie on the same line. Moreover, the exact $x$-coordinates of the vertical segments, nor the exact $y$-coordinates of the horizontal ones, are important, only their linear orders. Thus we assume that our bipartite graph $G = (A \cup B, E)$ has its vertices linearly ordered within the classes of bipartition, $A = \{a_1 , a_2 , \dots, a_n\}, B = \{b_1 , b_2 \dots, b_m\}$. We further assume that the vertices of $A$ are to be represented by vertical segments and the vertices of $B$ by horizontal ones. We say that a grid intersection representation respects the orders of $A$ and $B$ if for every $i < j$, the $x$-coordinate of $a_i$ is smaller than the $x$-coordinate of $a_j$, and the $y$-coordinate of $b_i$ is smaller than the $y$-coordinate of $b_j$.

\begin{thm}
	The graph $G = (A \cup B, E)$ has a grid intersection representation that respects the linear orders of $A$ and $B$ if and only if there are no 6 indices $i < j < k, \alpha < \beta < \gamma$ such that $a_i b_\beta , a_k b_\beta , a_j b_\alpha , a_j b_\gamma$ are edges of $G$ and $a_j b_\beta$ is not. Such a configuration of 6 vertices is called a \textbf{volswagen} in $G$.
	\label{thm-5}
\end{thm}

\begin{proof}
	Exercise.
\end{proof}

It is easy (i.e., decidable in polynomial time) to check if a bipartite graph contains a volkswagen with respect to given linear orderings of the vertices in its classes of bipartition. We will see in the last class that it is NP-complete to decide if a given bipartite graph is a grid intersection graph, which means that it is NP-complete to decide if a given bipartite graph allows linear orderings of its classes of bipartition with respect to which there is no volkswagen. The following problem is thus quite interesting in this context.

\textbf{Open problem:} Given a bipartite graph with one class of bipartition linearly ordered. How difficult is to decide if the other class of bipartition can be linearly ordered so that there is no volkswagen with respect to these orderings?

\begin{thm}[Bellantoni, Hartman, Przytycka, Whitesides]
	Every bipartite graph of boxicity $2$ is a grid intersection graph.
\end{thm}

\begin{figure}[!ht]\centering
	\begin{tikzpicture}
		\draw[black!0, fill=myblue!10] (0,-2) rectangle (6,2);
		\draw (0,2) -- (7,2);
		\draw (0,2) -- (0, -3);
		\draw[blue, fill=myblue!30] (0,0) rectangle (3,2);
		\node at (1.5, 1) {$C$};
		\node at (5,-1) {$BR(C)$};
	\end{tikzpicture}
	\caption{An illustration to the definition of region $BR(C)$.}
\end{figure}

\begin{proof}[Sketch of proof]
	Suppose $G = (A \cup B, E)$ is a bipartite graph and let $\mathcal{R} = \{R(u) : u \in A \cup B\}$ be an intersection representation of $G$ by axes-parallel rectangles in the plane. For a rectangle $C$ in the plane, we define regions $BL(C)$, $BR(C)$, $TL(C)$, $TR(C)$ as follows: $BR(C)$ contains all points with $x$-coordinate greater than the $x$-coordinate of the left side of $C$, with $y$-coordinate smaller than the $y$-coordinate of the top side of $C$, but which do not lie inside the rectangle $C$. The other regions ($BL$ standing for Bottom-Left, $TL$ standing for Top-Left, and $TR$ standing for Top-Right) are defined in a similar way. Then define two binary relations, one on $A$, the other one on $B$, as follows
	
	$$
	\begin{array}{c c c c}
		a_1 <_R a_2 & \Leftrightarrow & R(a_1) \cap BR(R(a_2)) \neq \emptyset & \text{ for } a_1, a_2 \in A, \\
		b_1 <_R b_2 & \Leftrightarrow & R(b_1) \cap BL(R(b_2)) \neq \emptyset & \text{ for } b_1, b_2 \in B. \\
	\end{array}
	$$
	
	\begin{claim}
		Both relations $<_R$ and $<_L$ are antireflexive, antisymmetric and acyclic. It follows that the transitive closures $<_{R}^T$ and $<_{L}^T$ of $<_R$ and $<_L$, respectively, are partial orders.
	\end{claim}
	
	Let $<_R^\ast$ and $<_L^\ast$ be topological sortings of $<_R^T$ and $<_L^T$, respectively.
	
	\begin{claim}
		With respect to the linear orderings $<_R^\ast$ and $<_L^\ast$ of its classes of bipartition, $G$ has no volkswagens. Hence $G$ has a grid intersection representation respecting these orders by Theorem \ref{thm-5}.
	\end{claim}
\end{proof}
	\include{grg-i/09-string-graphs}
	\part{Geometric Representations of Graphs II}
	\chapter{STRING $\in$ NP}

Now we will take a look at recognition of STRING graphs. Mainly that this problem is indeed in NP. We have already shown in the first part that STRING-recognition is NP-hard. The main problem is that some string graphs require representation with exponentially many crossings. Which is not possible to guess in an NP algorithm.

Also we have shown (Schaeffer and Štěfanovič) result that any string graph has a representation with at most exponentially many crossings.

Before we continue lets give us an example of a graph and its STRING representation which can be seen on Fig. \ref{string example}.

\begin{figure}[!ht]\centering
	\begin{subfigure}{.45\textwidth}\centering
		\begin{tikzpicture}[main/.style = {draw, circle, thick}]
			\node[main] (u) at (0,0) {$u$};
			\node[main] (v) at (2,0) {$v$};
			\node[main] (w) at (0,1.5) {$w$};
			\node[main] (x) at (2,1.5) {$x$};
			\node[main] (y) at (1,3) {$y$};
			\draw[thick] (u) -- (v);
			\draw[thick] (u) -- (w);
			\draw[thick] (v) -- (x);
			\draw[thick] (w) -- (x);
			\draw[thick] (w) -- (y);
			\draw[thick] (x) -- (y);
		\end{tikzpicture}
		\caption{Graph $G$.}
	\end{subfigure}
	\begin{subfigure}{.45\textwidth}\centering
		\begin{tikzpicture}
			\draw[thick] (-1,0) to[out=0, in=180] (1,.5) to[out=0, in=180] (2,-.5) to[out=0, in=180] (4,0);
			\node at (-1,.3) {$y$};
			\draw[thick] (.3, -1.4) to[out=90, in=180] (1,1) to[out=0, in=90] (1.7, -1.4);
			\node at (0, -1.4) {$w$};
			\draw[thick] (2, -0.8) to[out=180, in=90] (-1, -2) to[out=270, in=180] (2, -2.7);
			\node at (2.3, -2.7) {$u$};
			\draw[thick] (1.4, -1.3) to[out=0, in=225] (3.3, -.7) to[out=45, in=270] (3.6, 1);
			\node at (3.9, 1) {$x$};
			\draw[thick] (0, -3) to (3.7, -.3);
			\node at (0, -3.2) {$v$};
		\end{tikzpicture}
		\caption{String representation.}
	\end{subfigure}
	\caption{Example of a STRING graph.}
	\label{string example}
\end{figure}

Firstly lets define another problem which can be converted from STRING-recognition.

\section{Weak AT-realization}

The \textbf{INPUT} is a graph $G = (V,E)$ and $R \subseteq \binom{E}{2}$. The \textbf{GOAL} is to find a drawing of $G$ where only the pairs of edges from $R$ are allowed to cross.

Note that AT stands for abstract topological and weak means that the pairs do \textbf{not hove} to cross, only they can.

For now the drawing is somewhat basic. That is no edges crosses through a vertex. Vertices are points and edges are curves.

\section{Reduction of STRING-recognition to Weak AT-realization}

For a given graph $G = (V_G,E_G)$ we define the following graph $H = (V_H, E_H)$ and $R$. $V_H = V_G \cup E_G$ and $E_H = \{\{v,e\}, v \in V_G, e \in E_G, v \in e\}$. Lets see an example on Fig. \ref{reduction of strings}. We may see that the graph is obtained by subdividing all edges.

\begin{figure}[!ht]\centering
	\begin{subfigure}{.45\textwidth}\centering
		\begin{tikzpicture}[main/.style = {draw, circle, thick}, e/.style = {myblue}]
			\node[main] (u) at (0,0) {$u$};
			\node[main] (v) at (2,0) {$v$};
			\node[main] (w) at (0,2) {$w$};
			\node[main] (x) at (2,2) {$x$};
			\node[main] (y) at (1,4) {$y$};
			\draw[thick] (u) -- (v) node[midway, e, below] {$e$};
			\draw[thick] (u) -- (w) node[midway, e, left] {$f$};
			\draw[thick] (v) -- (x) node[midway, e, right] {$g$};
			\draw[thick] (w) -- (x) node[midway, e, below] {$h$};
			\draw[thick] (w) -- (y) node[midway, e, above, left] {$i$};
			\draw[thick] (x) -- (y) node[midway, e, above, right] {$j$};
		\end{tikzpicture}
		\caption{Original graph $G$.}
	\end{subfigure}
	\begin{subfigure}{.45\textwidth}\centering
		\begin{tikzpicture}[main/.style = {draw, circle, thick}, e/.style = {draw, circle, fill, myblue}]
			\node[main] (u) at (0,0) {$u$};
			\node[main] (v) at (2,0) {$v$};
			\node[main] (w) at (0,2) {$w$};
			\node[main] (x) at (2,2) {$x$};
			\node[main] (y) at (1,4) {$y$};
			\draw[thick] (u) -- (v) node[midway, e] {\textcolor{white}{$e$}};
			\draw[thick] (u) -- (w) node[midway, e] {\textcolor{white}{$f$}};
			\draw[thick] (v) -- (x) node[midway, e] {\textcolor{white}{$g$}};
			\draw[thick] (w) -- (x) node[midway, e] {\textcolor{white}{$h$}};
			\draw[thick] (w) -- (y) node[midway, e] {\textcolor{white}{$i$}};
			\draw[thick] (x) -- (y) node[midway, e] {\textcolor{white}{$j$}};
		\end{tikzpicture}
		\caption{Graph $H$.}
	\end{subfigure}
	\caption{Creating $H$ from $G$.}
	\label{reduction of strings}
\end{figure}

Lastly we define $R = \{\{\{v, e\}, \{w,f\}\}, \{v,w\} \in E_G\}$.

\begin{claim}
	$G$ is a STRING graph $\Leftrightarrow$ $(H,R)$ has Weak AT-realization.
\end{claim}

\begin{proof}
	"$\Rightarrow$" Suppose $G$ is a STRING graph. Pick endpoints as vertices and crossings as edge-vertices of $H$. Choose these arbitrary (see Fig. \ref{rightarrow} for an example). The edges will follow the lines in a STRING representation.
	
	"$\Leftarrow$" Suppose $(H,R)$ has a Weak AT-realization. See that $H$ is bipartite. Now represent it by going from the vertex alongside every edge to the edge-vertices of $H$ and almost return to the vertex. See Fig. \ref{leftarrow} for an example. It is easily observable that when lines cross then indeed they need be connected by an edge.
	
	\begin{figure}[!ht]\centering
		\begin{subfigure}{.4\textwidth}\centering
				\begin{tikzpicture}[main/.style = {draw, circle, thick, myblue}, c/.style = {thick, myblue},
					side/.style = {draw, circle, thick, mygreen}]
					\draw[c,name path=A] (-1,0) to[out=0, in=180] (1,.5) to[out=0, in=180] (2,-.5) to[out=0, in=180] (4,0);
					\node[main] at (-1.35,0) (y) {$y$};
					\draw[c,name path=B] (.3, -1.4) to[out=90, in=180] (1,1) to[out=0, in=90] (1.7, -1.4);
					\node[main] at (.3, -1.75) (w) {$w$};
					\draw[c,name path=C] (2, -0.8) to[out=180, in=90] (-1, -2) to[out=270, in=180] (2, -2.7);
					\node[main] at (2.35, -2.7) (u) {$u$};
					\draw[c,name path=D] (1.4, -1.3) to[out=0, in=225] (3.3, -.7) to[out=45, in=270] (3.6, 1);
					\node[main] at (3.6, 1.35) (x) {$x$};
					\draw[c,name path=E] (0, -3) to (3.7, -.3);
					\node[main] at (-.2, -3.25) (v) {$v$};
					% Intersections and dots
					\coordinate[name intersections={of=A and B, by=ab}];
					\fill[mygreen] (ab) circle (2pt) node[circle,draw] {};
					\coordinate[name intersections={of=A and D, by=ad}];
					\fill[mygreen] (ad) circle (2pt) node[circle,draw] {};
					\coordinate[name intersections={of=B and C, by=bc}];
					\fill[mygreen] (bc) circle (2pt) node[circle,draw] {};
					\coordinate[name intersections={of=B and D, by=bd}];
					\fill[mygreen] (bd) circle (2pt) node[circle,draw] {};
					\coordinate[name intersections={of=C and E, by=ce}];
					\fill[mygreen] (ce) circle (2pt) node[circle,draw] {};
					\coordinate[name intersections={of=D and E, by=de}];
					\fill[mygreen] (de) circle (2pt) node[circle,draw] {};
					\draw[dashed, thick, mygreen, bend right = 10] (v) edge (ce);
					\draw[dashed, thick, mygreen, bend left = 10] (u) edge (ce);
					\draw[dashed, thick, mygreen, bend right = 10] (v) edge (de);
					\draw[dashed, thick, mygreen, bend left = 10] (w) edge (bc);
					\draw[dashed, thick, mygreen, bend left = 10] (x) edge (ad);
				\end{tikzpicture}
			\caption{$\Rightarrow$ (Only some edges are present.)}
			\label{rightarrow}
		\end{subfigure}
		\begin{subfigure}{.58\textwidth}\centering
			\begin{tikzpicture}[main/.style = {draw, circle, thick, myblue}]
				\node[main] at (0,0) (v) {$v$};
				\node[main] at (-3, -2) (1) {};
				\node[main] at (0, -2) (2) {};
				\node[main] at (3, -2) (3) {};
				\draw[thick, myblue] (v) to[] (1);
				\draw[thick, myblue] (v) to[] (2);
				\draw[thick, myblue] (v) to[] (3);
				\draw[thick, mygreen] (v) to[out=190, in=180] (-3.2, -2.4) to[out=0, in=180] (-.4,-.6) to[out=0, in=180] (0, -2.3) to[out=0, in=180] (.4,-.6) to[out=0, in=180] (3.2, -2.4) to[out=0, in=-10] (.5, 0);
			\end{tikzpicture}
			\caption{$\Leftarrow$}
			\label{leftarrow}
		\end{subfigure}
		\caption{Examples for the proof.}
	\end{figure}
\end{proof}

\subsection{Test Weak AT-realization in NP}

Firstly we will write down all our assumption about the drawing we will be using.

\begin{enumerate}
	\item We are on the sphere.
	\item Vertex $v$ is drawn as an open disc $D_v$ (also we will denote $\partial D_v$ as the boundary of $D_v$ and $\overline{D_v}$ as the closure of $D_v$, i.e. $\overline{D_v} = D_v \cup \partial D_v$).
	\item An edge $e = \{v,w\}$ is represented by a curve $\gamma_e$ connecting a point from $\partial D_v$ to a point from $\partial D_w$ otherwise $\gamma_e$ is disjoint from $\bigcup_{x \in V} \overline{D_x}$.
	\item For $e \neq f$ $\gamma_e$ and $\gamma_f$ have distinct endpoints.
\end{enumerate}

We can easily see that these assumption are not restricting since it can be switch to "normal" drawing we are using by shrinking the discs to single points and switch back by expanding the single points.

Now we will proceed with the first part of the algorithm \ref{AT-1}.

\begin{algorithm}[!ht]
	\begin{algorithmic}[1]
		\Require $G = (V,E), R \subseteq \binom{E}{2}$
		\State $\forall v \in V:$ choose $D_v$ (so that $v \neq w$ have $\overline{D_v} \cap \overline{D_w} = \emptyset$).
		\State For any edge $e \in E$ incident to $v \in V:$ choose an endpoint of $\gamma_e$ on $\partial D_v$.
		\State \textcolor{mygreen}{$\dots$}
	\end{algorithmic}
	\caption{NP algorithm for testing Weak AT-realization.}
	\label{AT-1}
\end{algorithm}

Before we continue with the full algorithm we will take a short topological detour to some definitions and facts.

\subsection{Topological detour}

\begin{defn}
	$\mathcal{S} := (\text{sphere}) \setminus \bigcup_{v \in V} D_v$. Which is a compact surface with boundary.
\end{defn}

\begin{defn}
	$\mathcal{S}-\text{curve}$ is a curve in $\mathcal{S}$ with endpoints in the $\partial \mathcal{S}$ and no other point in $\partial \mathcal{S}$.
\end{defn}

\begin{defn}
	Two $\mathcal{S}$-curves $\gamma, \delta$ are \textbf{isotopic} (written as $\gamma \sim \delta$) if $\gamma$ and $\delta$ have the same endpoints and $\gamma$ can be transformed into $\delta$ by a continuous deformations of $\mathcal{S}$ which fixes the boundary.
\end{defn}

\begin{figure}[!ht]\centering
	\begin{subfigure}{.45\textwidth}\centering
		\begin{tikzpicture}[main/.style = {draw, circle, thick, fill, myblue}]
			\node[main] (0) at (0,2) {};
			\node[main] (1) at (-2,0) {};
			\node[main] (2) at (2,0) {};
			\node[main] (3) at (0, -2) {};
			\draw[thick, myblue] (1) to[out=30, in=180] (0,1.5) to[out=0, in=150] (2);
			\node[blue] (gamma) at (-1.8, .4) {$\gamma$};
			\draw[thick, myblue] (1) to[out=-30, in=180] (-.5,-1) to[out=0, in=180] (.5, -1.4) to[out=0, in=210] (2);
			\node[blue] (delta) at (-1.8, -.5) {$\delta$};
			\draw[thick, mygreen, ->] (0, 1.45) to[] (0,-1.15);
			\draw[thick, mygreen, ->] (-1, .8) to[] (-1,-.8);
			\draw[thick, mygreen, ->] (-.5, 1.3) to[] (-.5,-.9);
			\draw[thick, mygreen, ->] (.5, 1.3) to[] (.5,-1.3);
			\draw[thick, mygreen, ->] (1, .8) to[] (1,-1.05);
		\end{tikzpicture}
		\caption{Example of $\mathcal{S}$-curves $\gamma, \delta$ that are isotopic.}
	\end{subfigure}
	\begin{subfigure}{.45\textwidth}\centering
		\begin{tikzpicture}[main/.style = {draw, circle, thick, fill, myblue}]
			\node[main] (0) at (0,2) {};
			\node[main] (1) at (-2,0) {};
			\node[main] (2) at (2,0) {};
			\node[main] (3) at (0, 0) {};
			\draw[thick, myblue] (1) to[out=40, in=140] (2);
			\node[myblue] (gamma) at (-1.8, 1) {$\gamma$};
			\draw[thick, myblue] (1) to[out=60, in=180] (0,3) to[out=0, in=120] (2);
			\node[myblue] (delta) at (-1.3, .2) {$\delta$};
			\draw[thick, myblue] (1) to[out=190, in=180] (0,-2) to[out=0, in=350] (2);
			\node[myblue] (deltap) at (-1.3, -1.2) {$\delta'$};
		\end{tikzpicture}
		\caption{Counterexample of non-isotopic $\gamma \not \sim \delta$ and example of isotopic $\gamma \sim \delta'$, since we are on the sphere.}
	\end{subfigure}
	\caption{Example of isotopic and non-isotopic $\mathcal{S}$-curves.}
\end{figure}

\begin{defn}
	For two $\mathcal{S}$-curves $\gamma_1, \gamma_2$ the intersection number $i(\gamma_1, \gamma_2)$ is
	
	$$
	\min_{\delta_1 \sim \gamma_1, \delta_2 \sim \gamma_2} |\delta_1 \cap \delta_2|.
	$$
\end{defn}

\begin{fact}
	Let $\gamma_1, \dots, \gamma_n$ be $\mathcal{S}$-curves. Then there are $\mathcal{S}$-curves $\delta_1, \dots, \delta_n$ s.t.
	
	\begin{enumerate}
		\item $\delta_i \sim \gamma_i$ for $i = 1, \dots, n$,
		\item $|\delta_i \cap \delta_j| = i(\gamma_i, \gamma_j)$ for $i \neq j \in \{1, \dots, n\}$. 
	\end{enumerate}
\end{fact}

\noindent
With all these definitions we will choose the following triangulation $T = (V_T, E_T)$ of $\mathcal{S}$:

\begin{enumerate}
	\item $\forall v \in V: \partial D_v$ has three vertices and three edges of $T$.
	\item There are no other vertices of $T$, the remaining edges are chosen arbitrarily as $\mathcal{S}$-curves, so that $\mathcal{S}$ is partitioned into triangle faces.
	\item Vertices of $T$ do not coincide with endpoints of $\gamma_e, e \in E$ from algorithm.
\end{enumerate}

\begin{defn}
	An $\mathcal{S}$-curve $\gamma$ is \textbf{normal} w.r.t. T if
	
	\begin{enumerate}
		\item The endpoints of $\gamma$ are distinct from $V_T$.
		\item Any point where an edge of $T$ meets $\gamma$ is either an endpoint of $\gamma$ or a proper crossing (no touching is allowed).
		\item $\gamma$ does not have two consecutive crossings with any edge of $T$.
		\item $\gamma$ does not cross the same edge of $T$ no more than $(2^{|E_T| + |E|})$-times.
	\end{enumerate}
\end{defn}

\begin{claim}
	If $G$ has a weak AT-realization, then it has a realization where each $\gamma_e$ is normal w.r.t. T.
\end{claim}

\begin{proof}
	Take AT with minimum number of crossings. 1. is satisfied by the definition. 2. otherwise minimality is broken (see Fig. \ref{second-before} and \ref{second-after}), 3. again the minimality is broken (see Fig. \ref{third-before} and \ref{third-after}) and lastly 4. follows from GRG1.
	
	\begin{figure}[!ht]\centering
		\begin{subfigure}{.2\textwidth}\centering
			\begin{tikzpicture}[main/.style = {draw, circle}, e/.style = {thick, myblue}]
				\node[main,e] at (0,0) (0) {};
				\node[main,e] at (-2,2) (1) {};
				\draw[e] (0) -- (1);
				\draw[thick, mygreen] (0, .5) to[out=170, in=310] (-1,1) to[out=130,in=290] (-1.5,2);
			\end{tikzpicture}
			\caption{Second before.}
			\label{second-before}
		\end{subfigure}
		\begin{subfigure}{.2\textwidth}\centering
			\begin{tikzpicture}[main/.style = {draw, circle}, e/.style = {thick, myblue}]
				\node[main,e] at (0,0) (0) {};
				\node[main,e] at (-2,2) (1) {};
				\draw[e] (0) -- (1);
				\draw[thick, mygreen] (0, .5) to[out=170, in=310] (-.8,1.2) to[out=130,in=290] (-1.5,2);
			\end{tikzpicture}
			\caption{Second after.}
			\label{second-after}
		\end{subfigure}
		\begin{subfigure}{.2\textwidth}\centering
			\begin{tikzpicture}[main/.style = {draw, circle}, e/.style = {thick, myblue}]
				\node[main,e] at (0,0) (0) {};
				\node[main,e] at (-2,2) (1) {};
				\draw[e] (0) -- (1);
				\draw[e] (-1.5,.5) -- (-.5,1.5);
				\draw[thick, mygreen] (0, .5) to[out=170, in=310] (-1.2,.8) to[out=130,in=290] (-1.5,2);
			\end{tikzpicture}
			\caption{Third before.}
			\label{third-before}
		\end{subfigure}
		\begin{subfigure}{.2\textwidth}\centering
			\begin{tikzpicture}[main/.style = {draw, circle}, e/.style = {thick, myblue}]
				\node[main,e] at (0,0) (0) {};
				\node[main,e] at (-2,2) (1) {};
				\draw[e] (0) -- (1);
				\draw[e] (-1.5,.5) -- (-.5,1.5);
				\draw[thick, mygreen] (0, .5) to[out=170, in=310] (-.8,1.2) to[out=130,in=290] (-1.5,2);
			\end{tikzpicture}
			\caption{Third after.}
			\label{third-after}
		\end{subfigure}
		\caption{Examples for the proof.}
	\end{figure}
\end{proof}

\begin{defn}
	Let $\gamma$ be an $\mathcal{S}$-curve, normal w.r.t T for $e \in E_T: c_e(\gamma) := |e \cap \gamma|$.
\end{defn}

Sometimes when we will be talking about $\gamma_e$ for some $e \in E$ then for $f \in E_T$ we will denote $c_f(e)$ short for $c_f(\gamma_e)$. Also in the context we will sometimes leave out $\gamma_e$ and only have $c_f$ or $c_{uv}$ if $f = \{u,v\}$.

\begin{claim}
	The numbers $\left( c_f(e)\right)_{f \in E_T}$ determine $\gamma_e$ up to isotopy.
	\label{string-claim}
\end{claim}

\subsection{Word equations}

Before we showcase the proof we will show us a quick introduction to word algebra and its equations. We will be considering a finite alphabet $\mathcal{A} = \{\mathtt{a}, \mathtt{b}, \mathtt{c}, \dots\}$ and variables $\mathcal{V} = \{X,Y, \dots\}$. Each variable $X \in V$ has prescribed length $l(X) \in \N_0$ and represents a word over $\mathcal{A}$ of length $l(X)$.

\begin{example}
	$\mathcal{A} = \{\mathtt{a}, \mathtt{b}\}, \mathcal{V} = \{X,Y\}, l(X) = 2, l(Y) = 3$ and the equation is:
	
	$$
	XX\mathtt{a} = XY,
	$$
	
	\noindent where the operation is concatenation. One simple solution is $X := \mathtt{ba}$ and $Y := \mathtt{baa}$.
\end{example}

We may see that having one equation or a whole system of equation is the same. As we can see here:

$$
\begin{array}{c c c}
	L_1 = R_1 & & \\
	L_2 = R_2 & \Leftrightarrow & L_1L_2 \dots = R_1 R_2 \dots \\
	\vdots & &
\end{array}
$$

Also it may be that the size of a solution can be even exponential. But there is usually a repeating pattern. We can examine the following example to see that it is indeed true.

$$
\begin{array}{r c l}
	X_1 &=& \mathtt{a} \\
	X_2 &=& X_1 X_1 \\
	X_3 &=& X_2 X_2 \\
	&\vdots&
\end{array}
$$

\begin{fact}
	There is an encoding ("LZ-encoding") such that for any system of word equations we can in \textbf{polynomial time} (in $\abs{\mathcal{A}} + \abs{\mathcal{V}} + \abs{\text{equations}} + \sum_{X \in \mathcal{V}} \log |l(X)|$) determine if a solution exists and compute the LZ-encoding of the lexicographically smallest solution.
\end{fact}

\begin{fact}
	From the LZ-encoding of a word $X \in \mathcal{A}^{\ast}$ we can compute the number of occurrences of any symbol $\mathtt{a} \in \mathcal{A}$ in $X$.
\end{fact}

\subsubsection{LZ-encoding}

Shortly we will talk about the exact LZ-encoding. Firstly for a word $\mathtt{a}_1 \mathtt{a}_2 \mathtt{a}_3 \dots \mathtt{a}_n \in \mathcal{A}^\ast$, the \textbf{LZ-factorization} is an expresion of the form $\mathtt{a}_1 \mathtt{a}_2 \mathtt{a}_3 \dots \mathtt{a}_{n} = \mathtt{b}_1 X_1 \mathtt{b}_2 X_2 \mathtt{b}_3 X_3 \dots \mathtt{b}_n X_n$ where $X_i$ is the longest possible sub-word that also occurs in $\mathtt{b}_1 X_1 \mathtt{b}_2 X_2 \mathtt{b}_3 X_3 \dots \mathtt{b}_i$.

Then for LZ-encoding we replace each $X_i$ with a point to previous occurrence and the length of $X_i$.

\begin{example}
	Here is a simple example of converting a word to LZ-factorization and then to LZ-encoding.
	$$
	\begin{array}{l l l l l l l l l l}
		1 & 2 & 3 & 4 & 5 & 6 & 7 & 8 & 9 & 10 \\
		\hline
		\mathtt{a}&\textcolor{mygreen}{\mathtt{a}}&\mathtt{b}&\textcolor{myblue}{\mathtt{a}}&\textcolor{myblue}{\mathtt{b}}&\mathtt{a}&\textcolor{myred}{\mathtt{a}}&\textcolor{myred}{\mathtt{b}}&\mathtt{b}&\textcolor{myviolet}{\mathtt{b}}\\
		\mathtt{a}&\textcolor{mygreen}{X_1}       &\mathtt{b}&\textcolor{myblue}{X_2}       &                            &\mathtt{a}&\textcolor{myred}{X_3}       &                           &\mathtt{b}&\textcolor{myviolet}{X_4}\\
		\mathtt{a}&\textcolor{mygreen}{(1,1)}     &\mathtt{b}&\textcolor{myblue}{(2,2)}     &                            &\mathtt{a}&\textcolor{myred}{(2,2)}     &                           &\mathtt{b}&\textcolor{myviolet}{(3,1)}\\
	\end{array}
	$$
\end{example}

Now we will return back to the claim.

\begin{proof}[Proof of claim \ref{string-claim}]
	To clarify this we must showcase the following two points.
	
	\begin{enumerate}
		\item Verify that for every $e \in E$, the numbers $(c_f(e), f \in E_T)$ actually describe a curve $\gamma_e$ normal to $T$, with correct endpoints.
		\begin{itemize}
			\item Choose $e \in E$, check that for $f \in E_T$ belonging to $\partial \mathcal{S}$ we have
			
			$$
			c_f(e) =
			\left\{
			\begin{array}{ll}
				1 & \text{if } f \text{ contains an endpoint of } \partial \mathcal{S}\\
				0 & \text{otherwise}
			\end{array}
			\right. .
			$$
			
			\item For every face $t = \{x,y,z\}$ of $T$ we can see that the number of connections from $xy$ to $xz$ is equal to $\frac{c_{xy} + c_{xz} - c_{xz}}{2}$, therefore we must check that the number is in $\N_0$ for every face $t$ and every pair of edges $xy, xz$ and $yz$.
			
			\begin{figure}[!ht]\centering
				\begin{tikzpicture}[main/.style = {draw, circle, thick}, e/.style = {mygreen, bend right = 20, line width = 2pt}]
					\node[main] (u) at (0,0) {$u$};
					\node[main] (v) at (5,0) {$v$};
					\node[main] (w) at (2.5, 4) {$w$};
					\draw[thick] (u) -- (v);
					\draw[thick] (w) -- (v);
					\draw[thick] (u) -- (w);
					\node[main, fill, mygreen] (0) at (1,0) {};
					\node[main, fill, mygreen] (1) at (2,0) {};
					\node[main, fill, mygreen] (2) at (3,0) {};
					\node[main, fill, mygreen] (3) at (4,0) {};
					
					\node[main, fill, mygreen] (4) at (2.5/4,1) {};
					\node[main, fill, mygreen] (5) at (2.5/2,2) {};
					\node[main, fill, mygreen] (6) at (7.5/4,3) {};
					
					\node[main, fill, mygreen] (7) at (2.5 + 7.5/4,1) {};
					\node[main, fill, mygreen] (8) at (2.5 + 2.5/2,2) {};
					\node[main, fill, mygreen] (9) at (2.5 + 2.5/4,3) {};
					
					\draw[e] (0) edge (4);
					\draw[e] (1) edge (5);
					\draw[e] (7) edge (3);
					\draw[e] (8) edge (2);
					\draw[e] (6) edge (9);
				\end{tikzpicture}
				\caption{Example of connecting the triangle face edges.}
			\end{figure}
			
			\item Also we need to verify that this gives a correct curve $\gamma_e$, therefore no loops are present. For which we will use word equations $\mathcal{A} = \{\mathtt{a}, \mathtt{b}\}$. For every $f = \{u,v\} \in E_T$ we will introduce two variables $X_{uv}$ and $X_{vu}$ for which $l(X_{uv}) = l(X_{vu}) = c_{uv}(e)$. For every face $t = \{u,v,w\}$ and vertex $u \in t$ we will introduce variables $Y_{ut}$ and $Y_{tu}$, where $l(Y_{ut}) = l(Y_{tu}) = \frac{c_{uv} + c_{uw} - c_{wv}}{2}$. The equations will be as follows. For every $\{u,v\} = f \in E_T, f \in \partial \mathcal{S}$:
			
			$$
			X_{uv} = X_{vu} =
			\left\{
			\begin{array}{ll}
				\mathtt{b} & \text{if } f \text{ contains an endpoint of } \partial \mathcal{S}\\
				\emptyset & \text{otherwise}
			\end{array}
			\right.
			$$
			
			\noindent and for every face $t = \{u,v,w\}$ and edge $\{u,v\}$ we will have:
			
			$$
			X_{uv} = Y_{ut} Y_{tv}, X_{vu} = Y_{vt} Y_{tu}.
			$$
			
			\begin{observ}
				If $\gamma_e$ is a single connected curve, the system has a unique solution (all $\mathtt{b}$'s). Otherwise, there is a solution containing $\mathtt{a}$, we can find it efficiently.
			\end{observ}
		\end{itemize}
		\item Verify that for each $e_1,e_2 \in E, e_1 \neq e_2, \{e_1,e_2\} \notin R$ have $i (\gamma_{e_1}, \gamma_{e_1}) = 0$.
		
		\begin{itemize}
			\item Now we will denote $c_{uv} := c_{uv}(e_1) + c_{uv}(e_2)$ and then we will check that correctness as we did before.
			\item Next we also introduce similiar word equations. For alphabet $\mathcal{A} = \{\mathtt{a}, \mathtt{b}, \mathtt{c}\}$ and variables $X_{uv}, X_{vu},\\ Y_{ut}, Y_{tu}$ as was defined earlier. Only the length is different, for example $l(X_{uv}) = c_{uv} (e_1) + c_{vu}(e_2)$ and similarly for $l(Y_{ut}) = \dots$.
			
			For the first equations it will be more tricky since we can have both $e_1$ and $e_2$ intersecting $f \in E_T, f \subseteq \partial\mathcal{S}$. Otherwise it is straightforward how it is defined.
			
			For the next equations it will remain the same, which is $X_{uv} = Y_{ut} Y_{tv}, X_{vu} = Y_{vt} Y_{tu}$.
			
			\item Lastly we need to verify $X_{uv}$ has $c_{uv}(e_1)$ occurrences of $\mathtt{b}$ and also $c_{uv}(e_2)$ occurrences of $\mathtt{c}$.
		\end{itemize}
	\end{enumerate}
\end{proof}


By all these steps we have finished the proof that Weak AT-realization is in NP, because we only need to guess the numbers $c_{uv}(e)$ and add it to the NP algorithm \ref{NP-alg}, because with that we can in polynomial time check it is correct.

\begin{algorithm}[!ht]
	\begin{algorithmic}[1]
		\Require $G = (V,E), R \subseteq \binom{E}{2}$
		\State $\forall v \in V:$ choose $D_v$ (so that $v \neq w$ have $\overline{D_v} \cap \overline{D_w} = \emptyset$).
		\State For any edge $e \in E$ incident to $v \in V:$ choose an endpoint of $\gamma_e$ on $\partial D_v$.
		\State \textcolor{mygreen}{Create a triangulation $T$.}
		\State \textcolor{mygreen}{Guess all the numbers $c_{f}(e)$.}
	\end{algorithmic}
	\caption{NP algorithm for testing Weak AT-realization.}
	\label{NP-alg}
\end{algorithm}
	\chapter{$\chi$-boundedness}

\begin{notation}
	$\chi(G) = $ chromatic number $G$ and $\omega(G) = $ is the maximal clique in $G$.
\end{notation}

\begin{observ}
	$\forall G: \chi(G) \geq \omega(G)$.
\end{observ}

\begin{defn}
	Graph class $\mathcal{C}$ is $\chi$-bounded if $\exists f : \N \to \N$ s.t. $\forall G \in \mathcal{C} : \chi(G) \leq f(\omega(G))$.
\end{defn}

\begin{example}
	$\chi$-omezené třídy: úplné grafy, rovinné grafy (funkce $f$ je konstantní), perfektní grafy.
\end{example}

\begin{observ}
	Intervalové, permutační, chordální, comparability, $\dots$ jsou perfektní a tudíž i $\chi$-omezené.
\end{observ}

\section{$\chi$-boundedness of CIRC graphs}

\begin{observ}
	Circular-Arc grafy splňují: $\chi(G) \leq 2 \omega(G)$.
\end{observ}

\begin{proof}
	Na kružnici máme intervaly. V jednom bodě kružnici rozstřihneme a obarvíme $\omega(G)$ barvami. Potom už máme jen intervalový graf, který je perfektní.
\end{proof}

\begin{defn}
	Následující definice jsou si ekvivalentní:
	
	\begin{itemize}
		\item $G \in \text{CIR}$.
		\item $G$ je průnikový graf sečen v kružnici.
		\item $G$ je průnikový graf půlkružnic nad $x$-ovou osou.
		\item $G$ se dá reprezentovat posloupností čísel, kde se každé číslo $1, \dots, n$ vyskytuje právě dvakrát a platí, že vrcholy $v_i, v_j$ spolu sousedí právě když ta posloupnost obsahuje podposloupnost $i,j,i,j$ nebo $j,i,j,i$.
	\end{itemize}
\end{defn}

\begin{thm}
	$\text{CIR}$ je $\chi$-omezená.
\end{thm}

\begin{notation}
	Pro $k \in \N$ definujeme $\text{CIR}(k) := \{G \in \text{CIR}: \omega (G) \leq k\}$.
\end{notation}

\begin{observ}
	$\text{CIR}(k) \subseteq \text{CIR}(k+1) \subseteq \dots \subseteq \text{CIR}$.
\end{observ}

Chceme dokázat, že $(\forall k \in \N) \ (\exists f(k) \in \N) \ : \forall G \in \text{CIR}(k)$ platí, že $\chi(G) \leq f(k)$.

\begin{defn}
	Mějme $G \in \text{CIR}$ reprezentovaný jako průnikový graf půlkružnic. Pro $p,q \in \N_0$ $(p,q)$\textbf{-konfigurace} je množina $p+q$ vrcholů $x_1, \dots, x_p, y_1, \dots, y_q$ takové, že půlkružnice mají tuto vzájemnou polohu zobrazenou na obrázku \ref{konf}.
	
	\begin{figure}[!ht]\centering
		\begin{tikzpicture}
			\draw[thick] (0,0) -- (13,0);
			\draw[color=myblue, thick] (1,0) arc (180:0:4);
			\draw[color=myblue, thick] (2,0) arc (180:0:4);
			\draw[color=myblue, thick, dashed] (3,0) arc (180:0:4);
			\draw[color=myblue, thick] (4,0) arc (180:0:4);
			\draw[color=mygreen, thick] (4.4,0) arc (180:0:.3);
			\draw[color=mygreen, thick] (5.6,0) arc (180:0:.3);
			\draw[color=mygreen, thick, dashed] (6.8,0) arc (180:0:.3);
			\draw[color=mygreen, thick] (8,0) arc (180:0:.3);
			\node at (1, -.3) {$x_1$};
			\node at (2, -.3) {$x_2$};
			\node at (3, -.3) {$x_i$};
			\node at (3.9, -.3) {$x_p$};
			\node at (4.4, -.3) {$y_1$};
			\node at (5, -.3) {$y_1$};
			\node at (5.6, -.3) {$y_2$};
			\node at (6.2, -.3) {$y_2$};
			\node at (6.8, -.3) {$y_i$};
			\node at (7.4, -.3) {$y_i$};
			\node at (8, -.3) {$y_q$};
			\node at (8.6, -.3) {$y_q$};
			\node at (9.1, -.3) {$x_1$};
			\node at (10, -.3) {$x_2$};
			\node at (11, -.3) {$x_i$};
			\node at (12, -.3) {$x_p$};
		\end{tikzpicture}
		\caption{Definice $(p-q)$-konfigurace.}
		\label{konf}
	\end{figure}
\end{defn}

\begin{notation}
	$\text{CIR}(k,p,q) = \{G \in \text{CIR}, \omega(G) \leq k, G \text{ má reprezentaci bez } (p,q)\text{-konfigurace}\}$.
\end{notation}

\begin{observ}
	Pro $p > k$ a libovolné $q \in \N_0$: $\text{CIR}(k,p,q) = \text{CIR}(k)$.
\end{observ}

\begin{claim}
	$\forall k \in \N \ \forall p, q \in \N_0 \ \exists g(k,p,q) \in \N$ t.ž. $\forall G \in \text{CIR}(k,p,q) : \chi(G) \leq g(k,p,q)$.
\end{claim}

\begin{observ}
	Důkazem tohoto tvrzení platí věta.
\end{observ}

\begin{proof}
	Dané tvrzení dokážeme pomocí dvojité indukce: Indukcí podle $p \in \N_0$:
	
	\begin{lemma}
		$\forall G \in \text{CIR}(k,0,q) : \chi (G) \leq k (q-1)$ pro $q \geq 2$.
	\end{lemma}
	
	\begin{proof}[Proof of lemma]
		Indukcí podle $q \in \N_0$. Pokud $q = 2$ tak můžeme pozorovat, že všechny obloučky protínají společnou souvislou přímku. Tím pádem $\forall k : \text{CIR}(k,0,2) \subseteq \text{PER}$ které jsou perfektní.
		
		Nyní nechť $q > 2$. Nechť $G \in \text{CIR}(k, 0, q)$, vrcholy $G$ rozdělíme na 2 části $V_1, V_2$ tak, že
		
		\begin{itemize}
			\item $V_1$ bude indukovat graf z $\text{CIR}(k,0,2)$ a
			\item $V_2$ bude indukovat graf z $\text{CIR}(k,0,q-1)$.
		\end{itemize}
		
		Označme $\pi$ jakožto nejpravější levý konec obloučku. Potom $V_1$ jsou vrcholy, které mají levý konec vlevo od $\pi$. A $V_2$ jsou vrcholy, které mají levý konec napravo od $\pi$. Pak už použijeme indukci.
	\end{proof}
	
	\begin{lemma}
		Nechť $p \geq 1, q \geq 1$. Potom $\forall G \in \text{CIR}(k,p,q):$ vrcholy $G$ lze rozdělit na 2 části $V_A, V_B$ tak, že každá komponenta $G[V_A]$ i $G[V_B]$ patří do $\text{CIR}(k, p-1, 2q+1)$.
	\end{lemma}
	
	\begin{cor}
		Lze vzít $g(k,p,q) = g (k, p-1, 2q+1)$ tedy tvrzení pak platí.
	\end{cor}
	
	\begin{proof}[Proof of lemma]
		Mějme $G \in \text{CIR}(k,p,q)$. BÚNO: $G$ je souvislý a máme danou reprezentaci. Nechť $x_0$ je vrchol $G$ jehož oblouček má nejlevější levý konec. Definujeme $V_i := \{x \in V(G) : \text{ nejkratší cesta v } G \text{ od } x_0 \text{ do } x$ má délku $i\}$.
		
		\begin{observ}
			Pokud vede hrana z $V_i$ do $V_j$ tak $|i - j| \leq 1$.
		\end{observ}
		
		\begin{observ}
			Z každého vrcholu $V_{i+1}$ vede aspoň 1 hrana do $V_i$.
		\end{observ}
		
		Nyní tvrdíme, že žádná komponenta $G[V_i]$ neobsahuje $(p-1, 2q + 1)$-konfiguraci.
		
		Nechť $C$ je komponenta $G[V_i]$ pro spor obsahující $(p-1, 2q + 1)$-konfiguraci. Označme $y_{q+1}$ nějaký oblouček $w$ musí protnout $y_{q+1}$, tedy $w \in V_{i-1}$. Aspoň jeden konec je mimo $C$. $V_{i-1} = V_0$ tedy hotovo. Kdyby měl oba konce v $C$, tak nelze protnout oblouček z $V_{i-2}$.
		
		Nyní nechť $x_1, \dots, x_{p-1}, y_{1}, \dots, y_{2q+1}$ je $(p-1, 2q + 1)$-konfigurace. Nechť $I$ je interval mezi nejlevějším a nejpravějším koncem obloučku v $C$.
		
		\begin{observ}
			Každý soused $w \in V_{i-1}$ vrcholu $y_{q+1}$ musí mít jeden konec mimo $I$.
		\end{observ}
		
		Potom $w, x_1, \dots, x_{p-1}, y_{1}, \dots, y_q$ nebo $w, x_1, \dots, x_{p-1}, y_{q+2}, \dots, y_{2q+1}$ je $(p,q)$-konfigurace v $G$, což je spor.
		
		Závěrem $V_{A} := \cup_{i \text{ sudé}} V_{i}$ a $V_{B} := \cup_{i \text{ liché}} V_{i}$.
	\end{proof}
\end{proof}

\section{$\chi$-unboundedness of SEG graphs}

\begin{defn}
	SEG is the class of graphs of intersection graphs of segments in the plane.
\end{defn}

Now the question might be if SEG graphs are $\chi$-bounded or not. This was firstly stated by Erdös. The answer came later by the following theorem.

\begin{thm}[Pawlik, Kozik, Krawczyk, Lagos, Micek, Trotter, Wolezak, 2012]
	$\forall k \in \N$ there exists triangle-free graph $G_k \in \text{SEG}$ with $\chi(G_k) \geq k$.
	\label{PKKLMTW thm}
\end{thm}

\begin{defn}
	\textbf{L-curve} is a union of a horizontal and vertical segment sharing a common bottom-left endpoint. \textbf{L-graph} is an intersection graph of L-curves.
\end{defn}

\begin{thm}
	Any L-graph is also SEG-graph.
\end{thm}

\begin{proof}
	Suppose $\mathcal{L} = \{l_1, l_2, \dots, l_n\}$ is a set of L-curves, we will show by induction on $\abs{\mathcal{L}} = n$ that there is a set $\mathcal{S} = \{s_1, s_2, \dots, s_n\}$ of segments s.t. $l_i \cap l_j \neq \emptyset \Leftrightarrow s_i \cap s_j \neq \emptyset$, and moreover:
	
	\begin{enumerate}
		\item all $s_1, \dots, s_n$ lie in the half plane below the $x$-axis;
		\item $s_i$ touches the $x$-axis \ifft $l_i$ can be extended upwards to infinity without crossing any other $l_j \in \mathcal{L}$;
		\item left-to-right order of $s_i$'s touching the $x$-axis is the same as for $l_i$'s.
	\end{enumerate}
	
	Now for $n = 1$ it is simple, see Fig. \ref{start-l-curves}, and for $n > 1$ \wlogt $l_n$ has topmost horizontal part. By induction we represent $\mathcal{L}' = \{l_1, \dots, l_{n-1}\}$ by $\mathcal{S}' = \{s_1, \dots, s_{n-1}\}$ then
	
	\begin{enumerate}
		\item shorten the $s_i$ if $l_i$ no longer can extend upwards and
		\item insert $s_n$ touching $x$-axis, nearly horizontal. See Fig. \ref{induction-l-curves}.
	\end{enumerate}
	
	\begin{figure}[!ht]\centering
		\begin{subfigure}{.45\textwidth}\centering
			\begin{tikzpicture}[l/.style = {line width = 2pt}, b/.style = {myblue}]
				\draw[l, b] (0,2) to[out=270, in=90] (0,0) to[out=0, in=180] (4,0);
				\node[b] at (-.3,0) {$l_1$};
			\end{tikzpicture}
			\caption{L-curves.}
		\end{subfigure}
		\begin{subfigure}{.45\textwidth}\centering
			\begin{tikzpicture}[l/.style = {line width = 2pt}, b/.style = {myblue}]
				\draw[l, gray] (0,0) -- (5,0);
				\node[gray] at (-.3,0) {$x$};
				\draw[l, b] (1, -2) -- (1,0);
				\node[b] at (.7, -2) {$p_1$};
			\end{tikzpicture}
			\caption{Segments.}
		\end{subfigure}
		\caption{Simple of start for an induction.}
		\label{start-l-curves}
	\end{figure}
	
	\begin{figure}[!ht]\centering
		\begin{subfigure}{.45\textwidth}\centering
			\begin{tikzpicture}[l/.style = {line width = 2pt}]
				\draw[l] (0,3) to[out=270, in=90] (0,0) to[out=0, in=180] (4,0);
				\node at (-.3,0) {$l_1$};
				\draw[l] (1,1.5) to[out=270, in=90] (1,-2) to[out=0, in=180] (3,-2);
				\node at (.7,-2) {$l_2$};
				\draw[l] (2,-.5) to[out=270, in=90] (2,-1.5) to[out=0, in=180] (5,-1.5);
				\node at (1.7,-1.5) {$l_3$};
				\draw[l, myblue] (.5,3) to[out=270, in=90] (.5,2) to[out=0, in=180] (4,2);
				\node[myblue] at (.8,3) {$l_4$};
			\end{tikzpicture}
			\caption{L-curves.}
		\end{subfigure}
		\begin{subfigure}{.45\textwidth}\centering
			\begin{tikzpicture}[l/.style = {line width = 2pt}]
				\draw[l, gray] (0,0) -- (5,0);
				\node[gray] at (-.3,0) {$x$};
				\draw[l] (1, -4) -- (1,0);
				\node at (.7, -4) {$p_1$};
				\draw[l] (0, -3) -- (2.7,-.3);
				\draw[dotted, line width = 1pt] (0, -3) -- (3,0);
				\node at (-.3, -3) {$p_2$};
				\draw[l] (2, -3) -- (3.6,-.6);
				\draw[dotted, line width = 1pt] (2, -3) -- (4,0);
				\node at (1.7, -3) {$p_3$};
				\draw[l, myblue] (4.7, -.5) -- (2,0);
				\node[myblue] at (5, -.5) {$p_4$};
			\end{tikzpicture}
			\caption{Segments.}
		\end{subfigure}
		\caption{Simple example for conversion.}
		\label{induction-l-curves}
	\end{figure}
\end{proof}

\begin{proof}[Proof of theorem \ref{PKKLMTW thm}]
	In fact we show that $\forall k \in \N$ there exists triangle-free L-graph $G_k$ with $\chi(G_k) \geq k$.
	
	\begin{defn}
		A \textit{configuration} is
		
		\begin{enumerate}
			\item a collection of L-curves inside the unit square $[0,1] \times [0,1]$, whose intersection graph is triangle-free;
			\item a set $\{P_1, \dots, P_m\}$ of \textit{"probes"} which are pairwise disjoint rectangles inside $[0,1] \times [0,1]$ touching its bottom boundary;
			\item any L-shape from the collection intersecting a probe $P_i$ must cross it from left to right; \label{conf-3}
			\item the L-shape crossing a given probe are pairwise disjoint. \label{conf-4}
		\end{enumerate}
		
		\begin{figure}[!ht]\centering
			\begin{subfigure}{.45\textwidth}\centering
				\begin{tikzpicture}[l/.style = {line width = 1pt}]
					\draw[myblue, fill=myblue!20] (0,0) rectangle (6,6);
					\draw[myred, fill=myred!20] (1,0) rectangle (2,4);
					\draw[myred, fill=myred!20] (3,0) rectangle (5,3);
					\draw[l] (.5,5) to[out=270, in=90] (.5,1) to[out=0, in=180] (2.5,1);
					\draw[l] (2.3,2) to[out=270, in=90] (2.3,1.5) to[out=0, in=180] (5.4,1.5);
					\draw[l] (2.6,3) to[out=270, in=90] (2.6,2) to[out=0, in=180] (5.6,2);
					\draw[l] (3,5.5) to[out=270, in=90] (3,4) to[out=0, in=180] (5,4);
				\end{tikzpicture}
				\caption{Example of configuration where we have \textcolor{myred}{two probes} in a \textcolor{myblue}{[0,1] box} with four L-curves.}
			\end{subfigure}
			\begin{subfigure}{.45\textwidth}\centering
				\begin{tikzpicture}[l/.style = {line width = 1pt}]
					\draw[myblue, fill=myblue!20] (0,0) rectangle (6,6);
					\draw[myred, fill=myred!20] (1,0) rectangle (2,4);
					\draw[myred, fill=myred!20] (3,0) rectangle (5,3);
					\draw[l] (2.6,2) to[out=270, in=90] (2.6,.5) to[out=0, in=180] (4.6,.5);
					\draw[l] (4,4) to[out=270, in=90] (4,2) to[out=0, in=180] (4.6,2);
					\draw[l] (.5,5) to[out=270, in=90] (.5,1) to[out=0, in=180] (2.5,1);
					\draw[l] (.7,5) to[out=270, in=90] (.7,.5) to[out=0, in=180] (2.5,.5);
				\end{tikzpicture}
				\caption{Counterexample of violating the definitions \ref{conf-3} and \ref{conf-4}.}
			\end{subfigure}
			\caption{Example and counterexamples of the definition.}
		\end{figure}
	\end{defn}
	
	We will construct two sequences of $A_1, A_2, A_3, \dots$ and $B_1, B_2, B_3, \dots$ of configurations s.t.:
	
	\begin{enumerate}
		\item $\forall k \in \N$ in any proper coloring of the L-curves in $A_k$, the L-curves seen inside the probes use at least $k$ different colors.
		\item $\forall k \in \N$ in any proper coloring of the L-curves in $B_k$ there exist a probe of $B_k$ which is crossed by L-curves of at least $k$ different colors.
	\end{enumerate}
	
	By induction for $k = 1$ it is straightforward, see Fig. \ref{start-configuration}. Now we show two parts.
	
	\begin{figure}[!ht]\centering
		\begin{tikzpicture}[l/.style = {line width = 1pt}]
			\draw[myblue, fill=myblue!20] (0,0) rectangle (3,3);
			\draw[myred, fill=myred!20] (1,0) rectangle (2,2);
			\draw[l] (.5,2.6) to[out=270, in=90] (.5,1) to[out=0, in=180] (2.5,1);
		\end{tikzpicture}
		\caption{Start of the induction with \textcolor{myred}{one probe} and one L-curve.}
		\label{start-configuration}
	\end{figure}
	
	\begin{enumerate}
		\item From $B_k$ to $A_k+1$. See Fig. \ref{ak+1}.
		\begin{enumerate}
			\item Insert one new L-shape inside every probe of $B_k$ and
			\item replace each probe by 2 new probes.
		\end{enumerate}
		\item From $B_k$ and $A_{k+1}$ to $B_{k+1}$. See Fig. \ref{bk+1}.
		\begin{enumerate}
			\item Insert a small copy of $A_{k+1}$ near the top of every probe of $B_k$ and
			\item extend the probes of these small copies all the way down to obtain probes of $B_{k+1}$.
		\end{enumerate}
	\end{enumerate}
	
	\begin{figure}[!ht]\centering
		\begin{subfigure}{.45\textwidth}\centering
			\begin{tikzpicture}[l/.style = {line width = 1pt}]
				\draw[myblue, fill=myblue!20] (0,0) rectangle (6,6);
				\draw[myred, fill=myred!20] (1,0) rectangle (2,4);
				\draw[myred, fill=myred!20] (3,0) rectangle (5,3);
				\draw[l] (1,1) to (2,1);
				\draw[l] (1,2) to (2,2);
				\draw[l] (1,3) to (2,3);
				\draw[l] (3,1) to (5,1);
				\draw[l] (3,2) to (5,2);
				\node at (5,5) {$B_k$};
			\end{tikzpicture}
			\caption{Simplification with \textcolor{myred}{two probes}.}
		\end{subfigure}
		\begin{subfigure}{.45\textwidth}\centering
			\begin{tikzpicture}[l/.style = {line width = 1pt}]
				\draw[myblue, fill=myblue!20] (0,0) rectangle (6,6);
				\draw[myred, fill=myred!20] (1,0) rectangle (2,4);
				\draw[myred, fill=myred!20] (3,0) rectangle (5,3);
				\draw[myorange, fill=myorange!20] (1.1,0) rectangle (1.2,3.7);
				\draw[myorange, fill=myorange!20] (1.4,0) rectangle (1.6,.8);
				\draw[l] (1,1) to (2,1);
				\draw[l] (1,2) to (2,2);
				\draw[l] (1,3) to (2,3);
				\draw[l, mygreen] (1.3,3.6) to[out=270, in=90] (1.3,.5) to[out=0, in=180] (1.7,.5);
				\draw[myorange, fill=myorange!20] (3.1,0) rectangle (3.4,2.7);
				\draw[myorange, fill=myorange!20] (3.6,0) rectangle (4.4,.8);
				\draw[l] (3,1) to (5,1);
				\draw[l] (3,2) to (5,2);
				\draw[l, mygreen] (3.5,2.6) to[out=270, in=90] (3.5,.5) to[out=0, in=180] (4.5,.5);
				\node at (4.3,5) {$B_k \to A_{k+1}$};
			\end{tikzpicture}
			\caption{\textcolor{mygreen}{Two new L-curves} and \textcolor{myorange}{four new probes}.}
		\end{subfigure}
		\caption{Conversion from $B_k$ to $A_{k+1}$.}
		\label{ak+1}
	\end{figure}
	
	\begin{figure}[!ht]\centering
		\begin{subfigure}{.45\textwidth}\centering
			\begin{tikzpicture}[l/.style = {line width = 1pt}]
				\draw[myblue, fill=myblue!20] (0,0) rectangle (6,6);
				\draw[myred, fill=myred!20] (1,0) rectangle (2,4);
				\draw[myred, fill=myred!20] (3,0) rectangle (5,3);
				\draw[l] (1,1) to (2,1);
				\draw[l] (1,2) to (2,2);
				\draw[l] (1,3) to (2,3);
				\draw[l] (3,1) to (5,1);
				\draw[l] (3,2) to (5,2);
				\node at (5,5) {$B_k$};
			\end{tikzpicture}
			\caption{Simplification with \textcolor{myred}{two probes}.}
		\end{subfigure}
		\begin{subfigure}{.45\textwidth}\centering
			\begin{tikzpicture}[l/.style = {line width = 1pt}]
				\draw[myblue, fill=myblue!20] (0,0) rectangle (6,6);
				\draw[myred, fill=myred!20] (1,0) rectangle (2,4);
				\draw[myred, fill=myred!20] (3,0) rectangle (5,3);
				\draw[Violet, fill=Violet!20] (1.1, 3.1) rectangle (1.9, 3.9);
				\draw[myorange, fill=myorange!20] (1.2, 0) rectangle (1.6, 3.6);
				\draw[Violet, fill=Violet!20] (3.6, 2.1) rectangle (4.4, 2.9);
				\draw[myorange, fill=myorange!20] (3.7, 0) rectangle (4.1, 2.6);
				\draw[l] (1,1) to (2,1);
				\draw[l] (1,2) to (2,2);
				\draw[l] (1,3) to (2,3);
				\draw[l] (3,1) to (5,1);
				\draw[l] (3,2) to (5,2);
				\node at (4,5) {$B_k \ \& \ A_{k+1} \to B_{k+1}$};
			\end{tikzpicture}
			\caption{\textcolor{Violet}{Copies of $A_{k+1}$} and \textcolor{myorange}{prolonged probes}.}
		\end{subfigure}
		\caption{Conversion from $A_{k+1}$ and $B_{k}$ to $B_{k+1}$.}
		\label{bk+1}
	\end{figure}
\end{proof}

Now we may ask ourselves if this prove can be extended to some other type of intersection graphs. Well indeed it can be done. Note that \textit{arc-connected} set means that all pairs from the set can be connected via an arc.

\begin{fact}
	For any compact arc-connected set $X \subseteq \R^2$ other then an axis parallel rectangle there is a triangle-free graph $G_k$, $\chi(G_k) \geq k$, which is the intersection graph of horizontally and vertically scaled copies of $X$.
\end{fact}

\begin{defn}
	A graph $G$ is $d$-degenerate if every non-empty subgraph of $G$ contains a vertex of degree at most $d$.
\end{defn}

\begin{observ}
	$G$ is $d$-degenerate then $\chi(G) \leq d+1$.
\end{observ}

\begin{proof}
	This observation can be seen by induction. Lets take $u$ which has $\deg(u) \leq d$ now take $G - u$ which is also $d$-degenerate and therefore by induction hypothesis colorable by $d+1$ colors. Lets take such coloring and now give $u$ a possible color. Because it only has $d$ neighbors we will have at least 1 possibility.
\end{proof}

\section{Axis-parallel rectangles intersection graphs}

\begin{thm}[Asblund, Grunbaum]
	If $G$ is an intersection graph of axis parallel rectangles in the plane then $\chi(G) \leq O(\omega^2 (G))$.
\end{thm}

\begin{proof}
	Suppose we have $G = (V,E)$ as above: each vertex $u \in V$ is represented by a rectangle $R_u$. Let $E = E_1 \dot{\cup} E_2$ as follows
	
	\begin{enumerate}
		\item $\{u,v\} \in E_1$ if a vertex of $R_u$ is inside $R_v$ or vice versa (Fig. \ref{inter-box})
		\item $E_2 = E \setminus E_1$ if it is cross-like (Fig. \ref{cross-box}).
	\end{enumerate}
	
	\begin{figure}[!ht]\centering
		\begin{subfigure}{.45\textwidth}\centering
			\begin{tikzpicture}
				\draw[myorange, fill=myorange!20] (0, 0) rectangle (2, 2);
				\draw[myorange, fill=myorange!20, opacity=0.8] (1, 1) rectangle (3, 3);
			\end{tikzpicture}
			\caption{One vertex is inside the other rectangle.}
			\label{inter-box}
		\end{subfigure}
		\begin{subfigure}{.45\textwidth}\centering
			\begin{tikzpicture}
				\draw[myorange, fill=myorange!20] (0, 1) rectangle (3, 2);
				\draw[myorange, fill=myorange!20, opacity=0.8] (1, 0) rectangle (2, 3);
			\end{tikzpicture}
			\caption{Cross-like intersection.}
			\label{cross-box}
		\end{subfigure}
		\caption{Two possibilities.}
	\end{figure}
	
	Lets denote $G_1 = (V, E_1)$, $G_2 = (V, E_2)$ and $k = \omega(G)$. We will show $\chi(G_1) = O(k)$ and $\chi(G_2) = O(k)$.
	
	\begin{itemize}
		\item For $G_1$ we claim that $|E_1| \leq 4 k \cdot |V|$ (and any subgraph of $G_1$ induced by $W \subseteq V$ has at most $4 k \cdot |W|$ edges). This is because each vertex of a rectangle is inside at most $k-1$ other rectangles, due to the maximal clique. So this means one for each vertex of a rectangle is in total $4 (k-1)$. Now we direct an edge $uv$ if vertex of $R_u$ is inside $R_v$. Thus the outdegree of any vertex is $\leq 4 (k-1)$ so $|E_1| \leq 4 (k-1) \cdot |V|$. Hence $G_1$ (and any of its induced subgraphs) has average degree $\frac{2 |E_1|}{|V|} = 8 (k-1)$. So the min degree is $\leq 8 (k-1)$ therefore it is $8(k-1)$-degenerate and hence $\chi(G_1) \leq 8 (k-1) + 1$.
		\item For $G_2$ we claim that $\chi(G_2) = \omega(G_2) = O(k)$ because $G_2$ is a comparability graph (for example sort the boxes from tallest to shortest).
		\item For $G$ we have $\chi(G) \leq \chi(G_1) \cdot \chi(G_2) = O(k^2)$. For this lets have $E_1 \dot{\cup} E_2$ and see Fig. \ref{two-colors}. That is we have coloring $1,2, \dots, p$ colors for $E_1$ and $1,2, \dots, q$ colors for $E_2$ and we create a coloring by tuples $(x,y)$ where $x \in [p]$ and $y \in [q]$.
	\end{itemize}
	
	\begin{figure}[!ht]\centering
		\begin{tikzpicture}[node distance = 30mm, main/.style = {draw, circle, thick}]
			\node[main] (0) {(1,\textcolor{mygreen}{1})};
			\node[main, above of = 0] (1) {(2,\textcolor{mygreen}{1})};
			\node[main, left of = 1] (2) {(3,\textcolor{mygreen}{1})};
			\node[main, right of = 1] (3) {(2,\textcolor{mygreen}{2})};
			\node[main, left of = 0] (4) {(1,\textcolor{mygreen}{3})};
			\node[main, above of = 2] (5) {(1,\textcolor{mygreen}{3})};
			\draw[thick] (0) -- (1);
			\draw[thick] (0) -- (2);
			\draw[thick] (1) -- (2);
			\draw[thick] (1) -- (5);
			\draw[thick, mygreen] (0) -- (3);
			\draw[thick, mygreen] (0) -- (5);
			\draw[thick, mygreen] (1) -- (4);
			\draw[thick, mygreen] (1) -- (3);
			\draw[thick, mygreen] (2) -- (4);
			\draw[thick, mygreen] (2) -- (5);
			\draw[thick, mygreen] (3) -- (4);
			\draw[thick] (3) -- (5);
		\end{tikzpicture}
		\caption{Disjoint union of edges for coloring.}
		\label{two-colors}
	\end{figure}
\end{proof}

\section{1-String with big girth}

Now we will continue further and see that L-graphs $\subseteq$ (proper, i.e. not overlapping segments) SEG $\subseteq$ 1-String $\subseteq$ String. For this we need to see the definitions first.

\begin{defn}
	$G \in$ 1-String if $G$ has a String representation in which every two strings intersect at most once.
\end{defn}

\begin{defn}
	The \textbf{girth} of a graph $G$ is the length of the shortest cycle in $G$, if $G$ has no cycle then girth is $+ \infty$. We will denote girth of $G$ as $\gamma(G)$.
\end{defn}

With the newly established terminology we have previously shown that L-graphs of girth $\geq 4$ have unbounded $\chi$, but what about girth $\geq 5$ (or $6, \dots$). Also we will give a remark which is that for arbitrary large girth and chromatic number can be created. This can be shown by probabilistic techniques.

\begin{thm}[Kostochka \& Nešetřil]
	Let $G$ be a 1-String graph, then
	$$
	\begin{array}{r c l r}
		\text{if } \gamma(G) \geq 5 & \Rightarrow & \chi(G) \leq 6 & (\text{degeneracy } \leq 5), \\
		\text{if } \gamma(G) \geq 6 & \Rightarrow & \chi(G) \leq 4 & (\text{degeneracy } \leq 3), \\
		\text{if } \gamma(G) \geq 8 & \Rightarrow & \chi(G) \leq 3 & (\text{degeneracy } \leq 2). \\
	\end{array}
	$$
\end{thm}

\begin{proof}
	Let $G = (V,E)$ be a connected 1-String graph, with a given 1-String representation. $n = |V|, m = |E|$. We will show the minimal degree of this graph where the degeneracy will follow, since its subgraphs can only have larger girth. Let $H = (V_H, E_H)$ be a plane graph whose drawing is obtained by placing a vertex on every crossing of strings of $G$, pieces of string between adjacent crossings are the edges of $H$. Recall Euler's formula $v - e + f = 2 \geq 0$.
	
	$$
	\begin{aligned}
		&v = |V_H| = \frac{1}{2} \sum_{v \in V} \deg(v) = m \\
		&e = |E_H| = \sum_{w \in V}(\deg(w) - 1) = \left(\sum_{w \in V} \deg (w)\right) - n = 2m - n \\
		&f = \text{\# of faces of } H
	\end{aligned}
	$$
	
	\begin{observ}
		If $H$ has a face bounded by a cycle of length $l$, then $G$ has a closed walk of length $l$ in which each edge appears at most once. Hence $\gamma(G) \leq l$.
	\end{observ}
	
	And consequently every face of $H$ is incident to at least $\gamma(G)$ edges of $H$. Hence $f \cdot \gamma(G) \leq 2 e$.
	
	$$
	f \leq \frac{2e}{\gamma(G)} = \frac{4m - 2n}{\gamma(G)}
	$$
	
	So we can use the following computation.
	
	$$
	0 < v - e + f \leq m - 2m + n + \frac{4m - 2n}{\gamma(G)} = n \left(1 - \frac{2}{\gamma(G)} \right) - m \left(1 - \frac{4}{\gamma(G)} \right)
	$$
	
	\noindent This enforces the following.
	
	$$
	m \left(1 - \frac{4}{\gamma(G)}\right) < n \left(1 - \frac{2}{\gamma(G)} \right)
	$$
	
	\noindent Thus the min. degree $\leq$ average degree which is
	
	$$
	= \frac{2m}{n} \leq \frac{2 \left(1 - \frac{2}{\gamma(G)} \right)}{\left(1 - \frac{4}{\gamma(G)}\right)} = \frac{2 (\gamma(G) -2)}{\gamma(G) - 4}
	$$
	
	\noindent Lastly we compute the exact values. If $\gamma(G) \geq 5$ then min degree $< 6$ so $\leq 5$ hence $\chi(G) \leq 6$. If $\gamma(G) \geq 6$ then min degree $< 4$ so $\leq 3$ hence $\chi(G) \leq 4$. If $\gamma(G) \geq 8$ then min degree $< 3$ so $\leq 2$ hence $\chi(G) \leq 3$.
\end{proof}

Now we will further generalize to String graphs, which will use completely new topic which is called Game of robber and cops. Note that there exists more versions of this game.

\section{Game of robber and cops}

We have a given graph $G$. Then 1 robber (\textcolor{myred}{\faCircle}) and $c$ cops (\textcolor{mygreen}{\faSquare}). This game is for 2 players. Both players take turns. First one plays with cops and the second with robber.

\begin{topic}{Rules}
	\begin{itemize}[]
		\item \textbf{Start of the game:} First player places cops on vertices of graph $G$. Then second player places the robber on some vertex.
		\item \textbf{Single turn:} First player can with each cop either move to adjacent vertex or stay in the same one. Then the robber can move to adjacent or stay in the same vertex.
		\item \textbf{Goal:} If both cop and robber end up in the same vertex cops have won. Alternatively robber wants to stay as long as possible.
	\end{itemize}
\end{topic}

\begin{example}
	We have an easy example of the given graph and robber and cops. See Fig. \ref{cops-robber}
	
	\begin{figure}[!ht]\centering
		\begin{tikzpicture}[node distance = 15mm, main/.style = {draw, circle}, thick]
			\node[main] (0) {\textcolor{mygreen}{\faSquare}};
			\node[main, below of = 0] (1) {\textcolor{mygreen}{\faSquare}};
			\node[main, left of =1] (2) {};
			\node[main, right of =1] (3) {};
			\node[main, below of =1] (4) {};
			\node[main, left of=4] (5) {\textcolor{mygreen}{\faSquare} \textcolor{mygreen}{\faSquare}};
			\node[main, right of=4] (6) {\textcolor{myred}{\faCircle}};
			\node[main, below of =6] (7) {};
			\draw (0) -- (1);
			\draw (1) -- (2);
			\draw (0) -- (2);
			\draw (1) -- (3);
			\draw (0) -- (3);
			\draw (1) -- (4);
			\draw (2) -- (4);
			\draw (3) -- (4);
			\draw (5) -- (4);
			\draw (6) -- (4);
			\draw (7) -- (4);
			\draw (7) -- (6);
			\draw (2) -- (5);
		\end{tikzpicture}
		\caption{Example for cops and robber game.}
		\label{cops-robber}
	\end{figure}
\end{example}



\begin{defn}
	The \textbf{cop-number} $\cn(G) :=$ smallest number of cops for which the cops have a winning strategy on $G$. And for class $\mathcal{C}$ of graphs we define $\cn(\mathcal{C}) := \sup \{\cn (G), G \in \mathcal{C}, G \text{ connected}\}$.
\end{defn}

\begin{example}[Cycles]
	For cycles $C_n$ with $n \geq 4$ we can easily see that $\cn(G) = 2$ because we may put two cops in one vertex and then enclose the circle by two pats. Alternatively one is not enough since the robber may only wait and if cop is in the neighboring vertex the robber moves away.
\end{example}

\begin{example}[Int]
	In Int which is class of interval graphs we may find out that $\cn(\text{Int}) = 1$. Firstly we sort the intervals by its left endpoints, then place cop in the leftmost interval. The strategy is either catch a robber or move right to the next interval. If robber could move away the robber must stay way apart to the right or left. But note that if he would be in the left we didn't follow our strategy.
	
	Alternatively we can say the Int has a clique-path decomposition and in each step we are either in the same clique so we can catch the robber or we may move to the next clique.
\end{example}

\begin{prop}
	Let $G$ be a graph with $\gamma(G) \geq 5$. If every vertex of $G$ has degree $\geq d$, then $\cn(G) \geq d$.
	\label{girth-cop}
\end{prop}

\begin{proof}
	If $v$ is the robbers current vertex and $v_1, v_2, \dots, v_k$ are the neighbours of $v$, then each cop can be in the closed neighbourhood of at most one of $v_1, v_2, \dots, v_k$. If $k > \#$ of cops, robber may move to a "safe" neighbour $v_i$ if needed.
\end{proof}

\begin{cons}
	If $\mathcal{C}$ is a class of graphs closed under induced subgraphs with $\cn(\mathcal{C}) = c < + \infty$, then every graph $G \in \mathcal{C}$ of girth $\geq 5$ is $c$-degenrate and hence $\chi(G) \leq c+1$.
\end{cons}

\begin{thm}[Das and Gahlawat, 2022]
	$\cn(\text{String}) \leq 13$.
\end{thm}

We won't show this theorem. Also the lower bound is known to be 3. And String graphs of girth $\geq 5$ have $\chi \leq 14$ which follows from the theorem and proposition \ref{girth-cop}.

\begin{thm}
	$\cn(\text{Outer planar}) \in \{3,4\}$.
\end{thm}

\begin{proof}[Proof of the lower bound]
	We may construct a so called "$3 \times 5$ grid" see Fig. \ref{grid} for the graph itself and a representation. Then if 2 cops are adjacent to the robber then there is always one vertex for "lazy strategy" robber.
	
	\begin{figure}[!ht]\centering
		\begin{subfigure}{.45\textwidth}\centering
			\begin{tikzpicture}[scale=.7]
				\foreach \i in {1,...,5} {
					\node[draw, circle, fill] (O-\i) at ({360/5 * (\i-1)}:3) {};
			    	}
				\foreach \i in {1,...,4} {
					\pgfmathtruncatemacro{\j}{\i+1}
					\draw[thick, myblue, bend right = 25] (O-\i) edge (O-\j);
				}
				\draw[thick, myblue, bend right = 25] (O-5) edge (O-1);
			       	\foreach \i in {1,...,5} {
			    		\node[draw,circle, fill] (M-\i) at ({360/5 * (\i-1) + 20}:2) {};
			    	}
			    	\foreach \i in {1,...,4} {
					\pgfmathtruncatemacro{\j}{\i+1}
					\draw[thick, myblue, bend right = 25] (M-\i) edge (M-\j);
				}
				\draw[thick, myblue, bend right = 25] (M-5) edge (M-1);
			    	\foreach \i in {1,...,5} {
					\node[draw, circle, fill] (I-\i) at ({360/5 * (\i-1)}:1) {};
			    	}
			    	\foreach \i in {1,...,4} {
					\pgfmathtruncatemacro{\j}{\i+1}
					\draw[thick, myblue, bend right = 25] (I-\i) edge (I-\j);
				}
				\draw[thick, myblue, bend right = 25] (I-5) edge (I-1);
				\foreach \i in {1,...,5} {
					\draw[thick, mygreen] (I-\i) -- (M-\i);
					\draw[thick, mygreen] (O-\i) -- (M-\i);
					\draw[thick, mygreen] (O-\i) -- (I-\i);
				}
			\end{tikzpicture}
		\end{subfigure}
		\begin{subfigure}{.45\textwidth}\centering
			\begin{tikzpicture}[thick, scale=.8]
				\draw[Black!30] (0,0) -- (7,0);
				\foreach \i in {1,...,4}{
					\pgfmathtruncatemacro{\j}{\i+1}
					\draw[myblue] (\i,0) to (\i,1) to (\j.1,.5);
					\draw[myblue] (\i.2,0) to (\i.2,2) to (\j.3, 1);
					\draw[myblue] (\i.4,0) to (\i.4,3) to (\j.5, 2);
				}
				\draw[myblue] (5,0) to (5,1) to (6.6,.5) to[out=-10, in=0] (2,4.8) to[out=180, in=90] (1.1,.5);
				\draw[myblue] (5.2,0) to (5.2,2) to (6.3, 1) to[out=-10, in=0] (2,4.4) to[out=180, in=90] (1.3,1.5);
				\draw[myblue] (5.4,0) to (5.4,2) to[out=120, in=0] (2,4) to[out=180, in=90] (1.5,2.5);
			\end{tikzpicture}
		\end{subfigure}
		\caption{Graph on the left and on the right its representation.}
		\label{grid}
	\end{figure}
\end{proof}

\begin{thm}
	$\cn(\text{Planar}) = 3$.
\end{thm}

\begin{proof}
	For the lower bound we can use the proposition \ref{girth-cop} and draw a planar graph which has girth $\gamma = 5$, one such graph is a dodecahedron, see Fig. \ref{dodecahedron}.
	
	\begin{figure}[!ht]\centering
		\begin{tikzpicture}[scale=.7]
			\foreach \i in {1,...,5} {
				\node[draw, circle, fill] (O-\i) at ({360/5 * (\i-1)}:3) {};
		    	}
			\foreach \i in {1,...,4} {
				\pgfmathtruncatemacro{\j}{\i+1}
				\draw[thick, myblue] (O-\i) -- (O-\j);
			}
			\draw[thick, myblue] (O-5) -- (O-1);
		       	\foreach \i in {1,...,10} {
		    		\node[draw,circle, fill] (M-\i) at ({360/10 * (\i-1)}:2) {};
		    	}
		    	\foreach \i in {1,...,9} {
				\pgfmathtruncatemacro{\j}{\i+1}
				\draw[thick, myblue] (M-\i) -- (M-\j);
			}
			\draw[thick, myblue] (M-10) -- (M-1);
		    	\foreach \i in {2,4,6,8,10} {
		    		\pgfmathtruncatemacro{\j}{\i / 2}
				\node[draw, circle, fill] (I-\j) at ({360/10 * (\i - 1)}:1) {};
		    	}
		    	\foreach \i in {1,...,4} {
				\pgfmathtruncatemacro{\j}{\i+1}
				\draw[thick, myblue] (I-\i) -- (I-\j);
			}
			\draw[thick, myblue] (I-5) -- (I-1);
			\foreach \i in {1,...,5} {
				\pgfmathtruncatemacro{\j}{2*\i}
				\pgfmathtruncatemacro{\k}{2*\i-1}
				\draw[thick, myblue] (I-\i) -- (M-\j);
				\draw[thick, myblue] (O-\i) -- (M-\k);
			}
		\end{tikzpicture}
		\caption{Dodecahedron as a planar graph.}
		\label{dodecahedron}
	\end{figure}
	
	Now for the upper bound we will show a strategy for 3 cops on connected planar graph $G$. Firstly one lemma.
	
	\begin{lemma}[cop guards path $P$]
		Let $G$ be a graph, let $P$ be a shortest path from $x$ to $y$ in $G$, $x,y \in V(G)$. Then 1 cop after finitely many moves can take position on $P$ and play a strategy that catches the robber if the robber takes any vertexof $P$.
	\end{lemma}
	
	\begin{proof}
		For simplicity denote $d(u,v)$ as the distance from $u \to v$, $c$ for cop and $r$ for robber. Strategy: If there is a vertex $w \in P$ s.t. $d(w,c) > d(w,r)$ on cops turn, then $c$ moves towards $w$, otherwise $c$ stays in place.
		
		If there is such $w$, then for $w' \in P$ on the other side of $c$ than $w$, we have $d(r,w') > d(w', c)$. Eventually, the game reaches a situation, where after every cop move $\forall w \in P : d(c,w) \leq d(r,w)$.
	\end{proof}
	
	The overview of the whole strategy is to constrain the robber by two paths and use the third cop to shrinken the graph even more. We will restrict the robber to smaller and smaller subgraphs $G = G_1 \supsetneq G_2 \supsetneq G_3 \supsetneq \dots \supsetneq \emptyset$ s.t. for $G_i$:
	
	\begin{enumerate}[I)]
		\item There are two paths $P,Q$ each guarded by 1 cop, $G_i$ is the connected component of $G \setminus (P \cup Q)$ containing the robber. $P,Q$ share the same endpoints $x_i, y_i$ and $P$ is the shortest $x_i \to y_i$ path in $G_i \cup P \cup Q$ and $Q$ is the shortest $x_i \to y_i$ path in $G_i \cup Q$. Where $G_i \cup P \cup Q$ is the same as $G_i \cup G[P] \cup G[Q]$. \label{typeI-planar-cop}
		\item There is a vertex $x$ s.t. $G_i$ is a component of $G-x$ containing the robber, $x$ is occupied by a cop. \label{typeII-planar-cop}
	\end{enumerate}
	
	Now for the strategy itself. Cops will take any vertex $x$ and robber takes any vertex $y \neq x$, let $G_2$ be the component of $G - x$ containing $y$, therefore we get type II. Now consider subcases of the starting type.
	
	\begin{itemize}[]
		\item Type \ref{typeII-planar-cop}: consider few subcases.
		
		\begin{itemize}
			\item $x$ has only one neighbour $z$ in $G_i$, so put a cop there, and let $G_{i+1}$ be the component of $G_i - z$ containing the robber. Therefore we obtain again type II.
			\item $x$ has more neighbours, choose $z \neq y$ neighbour of $x_i$ in $G_i$, $P :=$ shortest $xz$-path in $G_i$. Guard $P$ with a cop and let $G_{i+1}$ be the component of $G_i \setminus P$, so we obtain type I.
		\end{itemize}
		
		\item Type \ref{typeI-planar-cop}: again some subcases.
		
		\begin{itemize}
			\item If $G_i$ is adjacent to one vertex of $P \cup Q$ then we have type II. So see above case.
			\item Otherwise there is a path $R$ from $x$ to $y$ in $P \cup Q \cup G_i$, containing a vertex from $G_i$. Let $R$ be shortest such path, put cop to guard it. Afterwards either $P$ or $Q$ is no longer needed to constraint the robber so we will have free cop and smaller graph. This can be seen by dividing the face $G_i$ by the path $R$, after the cops guard all three paths then the orbber is left out in one face which is constrained by $R$ and one of $P$ or $Q$. Also we won't need to guard whole $R$ and the second path, since the face may be way smaller, only the common endpoints.
		\end{itemize}
	\end{itemize}
\end{proof}
	\chapter{PQ trees}

\begin{notation}
	A set $\{1, \dots, n\}$ will be denoted as $[n]$. Then \textbf{permutation} of $[n]$ is a sequence $\pi = \pi_1, \pi_2, \dots, \pi_n$ in which each $i \in [n]$ appears exactly once. \textbf{Interval} in a permutation $\pi$ of $[n]$ is a set $S = \{\pi_i, \pi_{i+1}, \dots, \pi_j\}$ for some $1 \leq i \leq j \leq n$. \textbf{Cyclic shift} of $\pi_1, \pi_2, \dots, \pi_n$ is a permutation of the form $\pi_i, \pi_{i+1}, \dots, \pi_n, \pi_1, \pi_2, \dots, \pi_{n-1}$ for some $i \in [n]$. Lastly \textbf{cyclic interval} of $\pi$ is interval in a cyclic shift of $\pi$.
\end{notation}

\begin{example}
	Lets see an example for a permutation $\pi = 31524$ of $[5]$. Then $\{1,2,5\}$ is an interval and $\{2,3,5\}$ is \textbf{not} an interval. Furthermore its cyclic shift can be $52431$. Where one cyclic interval of the original permutation can be $\{2,3,4\}$.
\end{example}

Now we will introduce two problems.

\begin{problem}{Consecutivity}
	\begin{itemize}[]
		\item \textbf{Input:} $n \in \N$, sets $S_1, S_2, \dots, S_k \subseteq [n]$.
		\item \textbf{Question:} is there a permutation $\pi$ of $[n]$ in which $S_1, S_2, \dots, S_k$ are all intervals?
	\end{itemize}
\end{problem}

\begin{problem}{Cyclic consecutivity}
	\begin{itemize}[]
		\item \textbf{Input:} $n \in \N$, sets $S_1, S_2, \dots, S_k \subseteq [n]$.
		\item \textbf{Question:} is there a permutation $\pi$ of $[n]$ in which $S_1, S_2, \dots, S_k$ are all cyclic intervals?
	\end{itemize}
\end{problem}

\begin{lemma}
	Consecutivity can be reduced to cyclic consecutivity.
\end{lemma}

\begin{proof}
	We are given $n \in \N$, and sets $S_1, S_2, \dots, S_k \subseteq [n]$. Now see the following: $\exists \pi$ of $[n]$ in which $S_1, \dots, S_k$ are intervals $\iff$ $\exists \pi^+$ of $[n+1]$ in which $S_1, \dots, S_k$ are cyclic intervals. For one way see that if we have $\pi$ which has intervals $S_1, \dots, S_k$ and create $(\pi, n+1) = \pi^+$ permutation which has cyclic intervals. For the other way lets have $\pi^+$ of $[n+1]$ that has cyclic intervals in $S_1, \dots, S_k$, choose the cyclic shift of $\pi^+$ with $n+1$ at the end. Then $\pi^+ = (\pi, n+1)$ where $\pi$ will be the permutation of $[n]$ containing $S_1, \dots, S_k$ intervals.
\end{proof}

\textbf{Cyclic permutation} is determined by a permutation $\pi$ and it is the set of all cyclic shifts of $\pi$. (Unformally we may draw a circle with the elements on the boundary and all cyclic permutations are when going clockwise around the circle.) We also denote $\cyc{n}{S_1, \dots, S_k}$ as the set of cyclic permutations of $[n]$ in which all the sets $S_1, \dots, S_k$ are cyclic intervals.

\begin{defn}
	A \textbf{PQ-tree} of order $n$ is an (unrooted, undirected) tree with $n$ leaves labeled $1,2, \dots, n$ and two types of interval nodes: P-nodes \faCircle, Q-nodes \faCircle[regular], every internal node has a prescribed cyclic permutation of its neighbours.
\end{defn}

\begin{defn}
	$\pi_T$ for PQ-tree $T$ is the cyclic permutation of $[n]$ induced by the clockwise order of the leaves of $T$.
\end{defn}

\begin{defn}
	Two PQ-trees $T,T'$ are equivalent if $T'$ can be obtained from $T$ by a sequence of the following operations:
	
	\begin{enumerate}
		\item change the cyclic order of neighbours of P-node arbitrarily, and
		\item reverse the order of neighbours of Q-node.
	\end{enumerate}
\end{defn}

\begin{defn}
	The set of cyclic permutations represented by $T$, denoted by $R_T$ is $\{\pi_{T'} | T'$ equivalent to $T\}$.
\end{defn}

\begin{figure}[!ht]\centering
	\begin{subfigure}{.45\textwidth}\centering
		\begin{tikzpicture}[node distance = 40, p/.style = {draw, circle, fill}, q/.style = {draw, circle}, thick]
			\node[p] (c) {};
			\node[q, left of = 0] (l) {};
			\node[q, right of = 0] (r) {};
			\node[left of = l] (5) {5};
			\node[above of = l] (1) {1};
			\node[below of = l] (2) {2};
			\node[above left of = c] (4) {4};
			\node[above right of = c] (3) {3};
			\node[right of = r] (8) {8};
			\node[above of = r] (7) {7};
			\node[below of = r] (6) {6};
			\draw (c) edge (4);
			\draw (c) edge (3);
			\draw (l) edge (5);
			\draw (l) edge (1);
			\draw (l) edge (2);
			\draw (r) edge (8);
			\draw (r) edge (7);
			\draw (r) edge (6);
			\draw (r) edge (c);
			\draw (l) edge (c);
		\end{tikzpicture}
		\caption{PQ-tree $T$ with permutation $\pi_T = 14378625$.}
	\end{subfigure}
	\begin{subfigure}{.45\textwidth}\centering
		\begin{tikzpicture}[node distance = 40, p/.style = {draw, circle, fill}, q/.style = {draw, circle}, thick]
			\node[p] (c) {};
			\node[q, left of = 0] (l) {};
			\node[q, right of = 0] (r) {};
			\node[left of = l] (5) {5};
			\node[above of = l] (2) {2};
			\node[below of = l] (1) {1};
			\node[above of = c] (3) {3};
			\node[below of = c] (4) {4};
			\node[right of = r] (8) {8};
			\node[above of = r] (7) {7};
			\node[below of = r] (6) {6};
			\draw (c) edge (4);
			\draw (c) edge (3);
			\draw (l) edge (5);
			\draw (l) edge (1);
			\draw (l) edge (2);
			\draw (r) edge (8);
			\draw (r) edge (7);
			\draw (r) edge (6);
			\draw (r) edge (c);
			\draw (l) edge (c);
		\end{tikzpicture}
		\caption{Equivalent PQ-tree $T'$ with $\pi_{T'} = 15237864$.}
	\end{subfigure}
	\caption{Example of PQ-tree, its permutation and equivalent tree.}
\end{figure}

\begin{thm}
	For any $n \in \N$, sets $S_1, \dots, S_k \subseteq [n]$, we can, in time $O(n + \sum_{i=1}^k |S_i|)$ determine whether $\cyc{n}{S_1, \dots, S_k}$ is non-empty, and if it is, construct a PQ-tree $T$ such that $R_T = \cyc{n}{S_1, \dots, S_k}$.
\end{thm}

\begin{proof}[Construction of the PQ-tree]
	We won't show the whole proof, but only the construction of $T$ which will be shown by an induction on $k$. For $k=0$ we create one internal P-node which has all $[n]$ leaves ordered as $1,2, \dots, n$ clockwise. Suppose for $k > 0$ we have constructed PQ-tree $T_{k-1}$ with $R_{T_{k-1}} = \cyc{n}{S_1, \dots, S_{k-1}}$. The goal is to find $T$ with $R_T = \cyc{n}{S_1, \dots, S_k}$.
	
	Let $e$ be an edge in $T_{k-1}$, then $T_{k-1}-e$ has two components "substrees determined by $e$". Then subtrees is \textbf{full} if each of its leaves is in $S_k$, \textbf{empty} if none of its leaves are from $S_k$ and \textbf{mixed} otherwise. Then an edge $e$ of $T_{k-1}$ is \textbf{mixed} if both subtrees of $T_{k-1}-e$ are mixed.
	
	\begin{observ}
		Mixed edges form a connected subgraph of $T_{k-1}$.
	\end{observ}
	
	\begin{proof}
		If it is not true then there is a path connecting two mixed edges, but the edges on the path has to be also mixed.
	\end{proof}
	
	\begin{observ}
		If there is a vertex of $T_{k-1}$ incident to three or more mixed edges, then $\cyc{n}{S_1, \dots, S_k} = \emptyset$.
	\end{observ}
	
	Suppose mixed edges form a path $P$. Now we will show steps to create new PQ-tree.
	
	\begin{enumerate}
		\item Replace $T_{k-1}$ by an equivalent tree in which around every vertex of $P$ the edges towards full subtrees are above $P$, the edges towards empty subtrees are below. If this is not possible, then $\cyc{n}{S_1, \dots, S_k} = \emptyset$.
		\item Replace every node $v_i$ of $P$ by two nodes $v_i^+$ connected to the full subtrees only and $v_i^-$ connected to the empty subtrees only.
		\item Insert a new Q-node adjacent to $v_1^+, v_2^+, \dots, v_m^+, v_m^-, v_{m-1}^-, \dots, v_1^-$ in this order ($m$ is for the number of nodes of $P$), call the new node $w$.
		\item If $v_i^+$ or $v_i^-$ is a Q-node, then contract the edge $w v_i^+$ (or $w v_i^-$). But keep the order.
		\item If there is a node of degree 2, suppress it, if $v_i^+$ or $v_i^-$ has degree 1, delete it. (Where suppressing is swapping the 2-edge path by a single edge.) 
	\end{enumerate}
	
	The correctness of this process involves more checking if all representations are still preserved and that all present in the new one was already there.
	
	For time complexity one must use clever data structure and use amortization arguments to obtain such result.
\end{proof}

\section{Applications}

\subsection{Recognition of INT in linear time}

Recall that we have already shown that INT = Chordal $\cap$ co-Co and also $G \in$ INT $\iff$ the maximal cliques of $G$ can be arranged into a sequence $Q_1, Q_2, \dots, Q_l$ so that for every vertex $v$, the cliques containing $v$ form an interval (in the permutation of maximal cliques).

\begin{algorithm}[!ht]
	\caption{Idea of the algorithm for recognizing INT.}
	\begin{algorithmic}[1]
			\Require Graph $G = (V,E)$.
			\Ensure Is $G \in$ INT?
			\State Create a set $\{Q_{i}; i \text{ is maximal clique of } G\} = \{Q_{1}, Q_{2}, \dots, Q_{l}\}$.
			\State $\forall v \in V(G):$ create a set $S_{v} = \{i; v \in Q_{i}\}$.
			\State Solve consecutivity for $\{S_{v}; v \in V\}$ in $[l]$.
	\end{algorithmic}
\end{algorithm}

Note that from PQ-trees this can be solved in linear time $O(|V| + \sum_{v \in V} |S_{v}|)$. Now we will dig deeper into the details of each step.

\begin{enumerate}
	\item We know that $G$ is chordal $\iff$ $G$ has PES \PES. Also we have shown how to compute PES in linear time. Therefore create PES $v_{1}, \dots, v_{n}$. If it does not exists, return that $G$ is not INT. From now on assume it exists. Create $\forall i = 1, \dots, n$ cliques $Q_{i} := \{v_{i}\} \cup \{\text{left neighbours of } v_{i}\}$. Firstly observe that all maximal cliques are among these cliques (simply because all cliques have their rightmost vertex in PES, therefore its $Q_{i}$ since it is maximal). Next we will discard $Q_{i}$  if $\exists j : Q_{i} \subsetneq Q_{j}$. Note that this happens if $j > i$ and $v_{i}v_{j} \in E$ and any more to the left neighbour of $v_{j}$ is also a neighbour of $v_{i}$ so we only need to compute sizes $|Q_{i}| = S_{i}$ and denote $k$ as the number of left neighbours starting by $v_{i}$.
		\begin{itemize}
			\item If $k = S_{i}$ then $Q_{i} \subseteq Q_{j}$.
			\item If $k < S_{i}$ then $Q_{i} \nsubseteq Q_{j}$.
			\item Also $k > S_{i}$ cannot happen.
		\end{itemize}

	\item We will also use PES.

	$$
	S_{v} = (\{i\} \cup \{j | v_{j} \text{ is a right neighbour of } v_{i}\}) \cap \{k; Q_{k} \text{ is maximal}\}
	$$

	Which also implies that $|S_{v}| \leq$ right degree of $v$ + 1. Therefore $\sum_{v \in V} |S_{v}| \leq |E| + |V|$. So all together it is $O(|E| + |V|)$.
\end{enumerate}

\subsection{Planarity testing with PQ trees}

\begin{problem}{Planarity testing}
	\begin{itemize}[]
		\item \textbf{Input:} Graph $G = (V,E)$.
		\item \textbf{Output:} Planar embedding of $G$ or a $K_{5}$ or $K_{3,3}$ minor of $G$.
	\end{itemize}
\end{problem}

For our usecase we will denote "planar embedding" as the rotation scheme, which is for any $v \in V$, cyclic order of edges incident to $v$ in a planar drawing.

\begin{defn}
	Fragment of a graph $G = (V,E)$ induced by a set $X \subseteq V$ contains

	\begin{enumerate}
			\item The subgraph of $G$ induced by $X$.
			\item For any edge $e = \{u,v\} \in E$ s.t. $u \in X$ and $v \notin X$ create new vertex $s_{e}$ ("stump of $e$") and an edge $\{u, s_{e}\}$.
	\end{enumerate}
\end{defn}

We say that fragment induced by $X$ in $G$ is \textbf{good} if both $X$ and $V \setminus X$ induces a connected subgraph of $G$. Also a planar embedding of a fragment is \textbf{good} if all the stumps are embedded on the boundary of a single face ("outer face").

\begin{observ}
	If $G$ is planar, then also any good fragment has a good embedding.
\end{observ}

From now on assume that $G$ is 2-connected.

\begin{fact}
	Any good embedding of a good fragment induces a cyclic order of the stumps. Moreover for any good fragment with at least one good embedding, there is a PQ-tree whose leaves correspond to the stumps and which represents precisely the cyclic orders of stupms in the good embeddings of the fragments.
\end{fact}

\begin{proof}[Sketch of proof]
	Firstly generally for a graph $H = (V_{H}, E_{H})$ equivalence on $E_{H}$ $e,f \in E_{h}: e \sim f$ if either $e = f$ or $e, f$ belongs to common cycle. Then classes of $\sim$ are 1. bridges and 2. biconnected components.

	Now the constructions is roughly that we will replace all biconected components by Q nodes. Then cutvertices to P nodes, stumps to leaves, bridge to edge and incidence between cutvertex and biconnected component will also form an edge.

	Every single operation with PQ-trees can be also done to change the embedding of the graph. It is also good to show the other way and that if PQ-tree wouldn't be able to keep pace with the embedding then there is a $K_{5}$ or $K_{3,3}$ minor.
\end{proof}

\begin{algorithm}[!ht]
	\caption{Planarity testing.}
	\begin{algorithmic}[1]
			\Require Graph $G = (V,E)$.
			\Ensure Planar embedding of $G$ or a $K_{5}$ or $K_{3,3}$ minor of $G$.
			\State Create $T$ as a DFS tree of $G$.
			\State Number its vertices in postorder (in order left-right-root).
			\State $T_{i}:=$ subtree rooted in vertex numbered $i$ and $F_{i} :=$ good fragment induced by vertices of $T_{i}$.
			\State Proceed bottom to top and construct PQ-trees for $F_{1}, F_{2}, \dots, F_{n}$.
 	\end{algorithmic}
\end{algorithm}

The exact procedure of building fragments is that in the leaf we just create a single P node and all its stumps. Then by induction let all children have their PQ-trees. Lets take one of these PQ-trees and it has stumps which has to be connected to the current vertex and other vertices. The former vertices has to be arranged consecutive, if it is not possible then end. That is similar to the PQ-trees where we tried to put all free subtrees to the above part and others to the below, now it is pretty much the same. Also in this step we remember the orientation of the stumps that will be connected to the root. This will merge to one big Q node and connect to the root which will be new P node. Also we will create such Q nodes for all children of the current root.

	\chapter{SPQR trees}

Before we show another interesting structure called SPQR-trees note that they are not related to the previously mentioned PQ-trees. Also in this section we will be considering multigraphs without loops, i.e. graphs can have parallel edges.

\begin{defn}
	Let $G = (V_G, E_G)$ be a biconnected multigraph, a \textbf{skeleton} of $G$ is multigraph $H = (V_H, E_H)$ with these properties:
	
	\begin{enumerate}
		\item $V_H \subseteq V_G$;
		\item every edge $e = \{u,v\} \in E_H$ represents a connected subgraph $G_e$ of $G$ ("pertinent graph of $e$") which contains the vertices $u,v$;
		\item every edge of $G$ belongs to exactly one pertinent graph;
		\item for $e,f \in E_H, e \neq f$, then $V(G_e) \cap V(G_f) = e \cap f$.
	\end{enumerate}
\end{defn}

\begin{figure}[!ht]\centering
	\begin{subfigure}{.3\textwidth}\centering
		\begin{tikzpicture}[node distance = 50, main/.style = {draw, circle, fill}, thick]
			\node[main] (1) {};
			\node[main, above right of = 1] (2) {};
			\node[main, below right of = 1] (3) {};
			\node[main, above right of = 3] (4) {};
			\node[main, right of = 4] (5) {};
			\node[below of = 1] (p1) {};
			\node[main, below of = p1] (6) {};
			\node[below of = 5] (p5) {};
			\node[main, below of = p5] (7) {};
			\node[main, above right of = 6] (8) {};
			\node[main, above left of = 7] (9) {};
			\node[main, below right of = 6] (10) {};
			\draw (1) -- (2) node[midway, below right] {1};
			\draw (1) -- (3) node[midway, above right] {2};
			\draw (3) -- (4) node[midway, above left] {3};
			\draw (2) -- (4) node[midway, below left] {4};
			\draw (4) -- (5) node[midway, above left] {5};
			\draw (3) -- (5) node[midway, below right] {6};
			\draw (2) -- (5) node[midway, above right] {7};
			\draw (1) -- (6) node[midway, left] {8};
			\draw (5) -- (7) node[midway, right] {9};
			\draw (6) -- (8) node[midway, below right] {10};
			\draw (8) -- (9) node[midway, above] {11};
			\draw (7) -- (9) node[midway, below left] {12};
			\draw (7) -- (10) node[midway, above left] {13};
			\draw (6) -- (10) node[midway, above right] {14};	
		\end{tikzpicture}
	\end{subfigure}
	\begin{subfigure}{.3\textwidth}\centering
		\begin{tikzpicture}[node distance = 70, main/.style = {draw, circle, fill}, thick]
			\node[main] (1) {};
			\node[main, left of = 1] (2) {};
			\node[main, below of = 1] (3) {};
			\node[main, below of = 2] (4) {};
			\draw (1) -- (2) node[midway, above] {1,2,3,4,5,6,7};
			\draw (1) -- (3) node[midway, left] {9};
			\draw (2) -- (4) node[midway, right] {8};
			\draw (3) -- (4) node[midway, below] {10,11,12,13,14};
		\end{tikzpicture}
	\end{subfigure}
	\begin{subfigure}{.3\textwidth}\centering
		\begin{tikzpicture}[node distance = 70, main/.style = {draw, circle, fill}, thick]
			\node[main] (1) {};
			\node[above right of = 1] (l1) {1,2,3,4,5,6,7,8};
			\node[below right of = 1] (l2) {9,10,11,12,13,14};
			\node[main, below right of = l1] (2) {};
			\draw[bend left = 80] (1) edge (2);
			\draw[bend right = 80] (1) edge (2);
		\end{tikzpicture}
	\end{subfigure}
	\caption{Example of graph $G$ and some of its skeletons.}
\end{figure}

\begin{defn}
	\textbf{Separation tree $T$} of a biconnected $G = (V_G, E_G)$ is a tree whose leaves correspond bijectively to edges of $G$, every internal node $\alpha$ has degree $\geq 3$ and has an associated skeleton $S_\alpha$ of $G$ with $\deg(\alpha)$ edges, such that the edge sets of the pertinent graphs of $S_\alpha$ correspond to the sets of leaves in the components of $T - \alpha$.
\end{defn}

\begin{defn}[SPQR-tree]
	An SPQR-tree of a biconnected multigraph is a separation tree $T$ whose internal nodes are of three types:
	
	\begin{enumerate}
		\item S-nodes (series): nodes whose skeleton is a cycle of length $\geq 3$.
		\item P-nodes (parallel): nodes whose skeleton is a graph with 2 vertices and $\geq 3$ parallel edges.
		\item R-nodes (ridged): nodes whose skeleton is a simple 3-connected graph.
	\end{enumerate}
	
	\noindent Moreover, no two S-nodes are adjacent and also no two P-nodes are adjacent. Sometimes Q-node is referred as leaf.
\end{defn}

\begin{fact}
	Every biconnected multigraph has unique SPQR-tree which can be computed in linear time.
\end{fact}

\begin{thm}
	Let $G$ be a biconnected multigraph with $m$ edges, $m \geq 3$, let $T$ be a separation tree of $G$. Then all the skeletons in $T$ have together at most $3m-6$ edges.
\end{thm}

\begin{proof}
	By induction on $m$. For $m = 3$ we have only one type of SPQR tree which is a tree with one internal node and three nodes. In the internal node there can either be a triangle or three-parallel edges graph. But in both case we have that $3m - 6 = 9 - 6 = 3$ which is true.
	
	Let $m > 3$ and consider case 1: $T$ has a single internal node, then it is also fine. Case 2 consider that $T$ has 2 adjacent internal nodes $\alpha, \beta$ and $|E_\alpha| \geq 2, |E_\beta| \geq 2$. Split $G$ into: $G_\alpha$ which is formed by $E_\alpha$ plus new edge $uv$, similarly $G_\beta$ which is formed by $E_\beta$ plus new edge $uv$. Where $u$ and $v$ are the split vertices between $\alpha$ and $\beta$. From these we create their separation trees, which will be made by "replacing" the second vertex by a leaf representing edge $uv$.
	
	By "disconnecting" edge $\alpha\beta$ in $T$ we obtain separation trees $T_\alpha, T_\beta$ for $G_\alpha, G_\beta$ respectively. Let $k := |E_\alpha|, m -k := |E_\beta|$ and $G_\alpha$ has $k+1$ edges and $G_\beta$ has $m-k+1$ edges.
	
	Finally use induction. Skeletons in $T_\alpha$ have at most $3(k+1)-6$ edges and skeletons in $T_\beta$ have at most $3(m-k+1)-6$ edges. Therefore skeletons in $T$ have at most $3k + 3 - 6 + 3m - 3k + 3 - 6 = 3m -6$ edges.
\end{proof}

\section{Create SPQR tree}

In this section we will show some techniques how an SPQR tree can be constructed.

\begin{defn}
	Suppose $G = (V_G, E_G)$ is a biconnected multigraph, $uv \in V_G, u \neq v$, edges $e,f \in E_G$ are in the same separation class w.r.t. $uv$ (that is $e \sim_{uv} f$) if
	
	\begin{enumerate}
		\item $e = f$;
		\item there is a path containing $e$ and $f$ whose internal vertices are different from $u,v$;
		\item there is a cycle containing $e$ and $f$ and containing at most one of $u,v$.
	\end{enumerate}
\end{defn}

\begin{fact}
	$\sim_{uv}$ is an equivalence on $E_G$.
\end{fact}

\begin{defn}
	$u,v$ are separation pair if $\sim_{uv}$ has more than one classes.
\end{defn}

We may see that cuts are separation pairs, but in fact not the only. For example $K_4$.

\begin{defn}
	A separation pair is \textbf{trivial} if
	
	\begin{enumerate}
		\item $\sim_{uv}$ has two classes, one of them contains just an edge
		\item $\sim_{uv}$ has 3 classes, all of them contains just one edge.
	\end{enumerate}
\end{defn}

\begin{observ}
	If $G$ has non-trivial separation pair, then $G$ is one of the following graphs: two vertices with two parallel edges, two vertices with three parallel edges, triangle, any simple 3-connected graph.
\end{observ}

\noindent Now we will show how to find an SPQR tree:

\begin{enumerate}
	\item Start with one internal node having entire $G$ in it and $|E_G|$ leaf nodes.
	\item As long as any skeleton has non-trivial separation pair $uv$ then split skeleton along $uv$.
	
	\begin{itemize}
		\item Splitting skeleton $S = (V_S, E_S)$ along a non-trivial separation $uv$. Firstly partition $E_S$ into two parts $E_{I}, E_{II}$ s.t. each part has at least two edges and each part is a union of $\sim_{uv}$-classes.
	\end{itemize}
	
	\item Merge adjacent P-nodes and S-nodes.
	
	\begin{itemize}
		\item When we have two S-nodes then we have two circles where one edge is representing the other graph of the second node. So together they form one circle. For P-nodes we have some parallel edges in between two vertices, where one is for the other part of the graph. Thus all together it is all of their parallel edges together.
	\end{itemize}
\end{enumerate}

\noindent This procedure will lead to SPQR tree.

\section{SPQR-trees and planarity}

We remind ourselves what a planar embedding of a connected graph is: rotation scheme (which is the cyclic order of edges around vertex).

\begin{thm}
	A biconnected $G$ is planar \ifft every skeleton in its SPQR-tree is planar $\iff$ every R-skeleton is planar.
\end{thm}

\begin{proof}
	"$\Rightarrow$" $K$ is a skeleton of $G \Rightarrow G$ has a subdivision of $K$ as a subgraph. Thus $K$ has to be planar.
	
	"$\Leftarrow$" We have embedding for all skeletons. Choose arbitrary root of SPQR-tree and process from bottom to top. For every edge of every skeleton other than the edge corresponding to the parent: construct the embedding of its pertinent graph, where the poles are in the same face (poles are the two special vertices). For vertex representing a skeleton: ignore the edge of a parent and combine the children to the current skeleton.
\end{proof}

\begin{defn}
	$G$ biconnected graph with SPQR-tree $T$, $K$ skeleton of $T$, $\mathcal{G}$ embedding of $G$, $\mathcal{K}$ embedding of $K$, $\mathcal{G}$ is \textbf{consistent} with $\mathcal{K}$ if
	
	\begin{enumerate}
		\item For every edge $e = \{x,y\} \in E(K)$, the edge of $G_e$ (pertitent graph of $e$) incident to $x$ form a cyclic interval in the rotation scheme $x$ in $\mathcal{G}$, same for $y$.
		\item For $e,f,g \in E(K)$ meeting in $x$, for $e' \in G_e, f' \in G_f, g' \in G_g$ meeting in $x$ the order of $e,f,g$ in $\mathcal{K}$ is the same as the order of $e',f',g'$ in $\mathcal{G}$.
	\end{enumerate}
\end{defn}

\begin{thm}
	$G,T$ as above, then
	
	\begin{enumerate}
		\item For every embedding $\mathcal{G}$ of $G$ each skeleton in $T$ has unique embedding consistent with $\mathcal{G}$.
		\item For any choice of an embedding of each skeleton of $T$, $G$ has unique embedding consistent with all skeletons.
	\end{enumerate}
\end{thm}

\begin{proof}
	We will showcase both properties.
	
	\begin{enumerate}
		\item We have $\mathcal{G}$ embedding of $G$ and skeleton $K$.
		
		\begin{enumerate}
			\item Firstly consider $x$ is not an articulation, since $G$ and $K$ are biconnected thus they have to be paths to $y$. We have 4 paths from $x$ to $y$ either $f_1, f_2$ are in the same pertinent which is not possible since they would meet in vertex of $K$, or if they are different from P-skeleton, then it is R (ridgit) thus there would be the path. Which is a contradiction. Thus it is P-skeleton. Hence the first part of the definition is done.
			
			\item Due to the cyclic intervals for $e,f,g \in E(K)$ they form a cyclic intervals. To show that it forms an embedding choose a path representing in its edge in $K$ and draw it.
		\end{enumerate}
	
		\item Fix an embedding of every skeleton. There exist one embedding that was constructed like in the previous theorem and we have two embeddings of $\mathcal{G}$ s.t. $\mathcal{G} \neq \mathcal{G}'$ so one cyclic order differs. We have to find node where $e',f',g'$ are in different pertinent. We may find an internal node which separates $e',f',g'$ so it suffices what we were looking for. So due 2. only one $\mathcal{G}$ or $\mathcal{G}'$ can be consistent.
	\end{enumerate}
\end{proof}

\begin{note}
	S-skeleton has exactly 1 embedding. R-skeleton has exactly 2 embeddings. P-skeletons with $m$ edges has $(m-1)!$ embeddings.
\end{note}

\noindent We can have more restricitions or we may count how many embeddings given graph has.

\begin{problem}{Partially Embedded Planarity (or PEP for short)}
	\begin{itemize} []
		\item \textbf{Input:} graph $G = (V_G, E_G)$, subgraph $H = (V_H, E_H)$, embedding $\mathcal{H}$ of $H$.
		\item \textbf{Output:} Is there a way to extend $\mathcal{H}$ into an embedding of $G$?
	\end{itemize}
\end{problem}

\begin{rem}
	This is a special example of general cases such that partially representation is given and we have to construct the whole representation.
\end{rem}

\begin{note}
	Embedding of $H$ is given by
	
	\begin{enumerate}
		\item rotation scheme
		\item for every cycle $C$ on the boundary of a face of $\mathcal{H}$ on every vertex $x \notin C$, specify on which side of $C$ $x$ is drawn. -- \textit{"cyclic-vertex position"}
	\end{enumerate}

	Note that again it does not keep the information about the outer face.
\end{note}

\begin{thm}
	PEP can be solved in polynomial (in fact linear) time.
\end{thm}

\begin{proof}
	Assume $G$ is biconnected. Then see this algorithm.
	
	\begin{algorithm}[!ht]
		\begin{algorithmic}[1]
			\State Compute SPQR-tree of $G$. This can be uniquely computed in linear time.
			\If{$G$ is not planar}
				\State \Return
			\EndIf
			\ForAll{skeletons}
				\State Determine whether it can be embedded in not obviously wrong way or have a counter example.
			\EndFor
		\end{algorithmic}
	\end{algorithm}

	\begin{defn}
		Let $K$ be a skeleton of the SPQR tree of $G$, an embedding $\mathcal{K}$ of $K$ is \textbf{obviously wrong} if
		
		\begin{enumerate}
			\item $\mathcal{H}$ has three edges $e,f,g$ meeting in a single vertex $x$, $\mathcal{K}$ has three edges $\bar{e}, \bar{f}, \bar{g}$, each having one of $e,f,g$ in its pertinent graph, the cyclic order of $e,f,g$ in $\mathcal{H}$ differs from the cyclic order if $\bar{e}, \bar{f}, \bar{g}$ in $\mathcal{K}$.
			\item $\mathcal{H}$ has facial cycle $C$ and a vertex $x \notin V(C)$, the edges of $C$ projects to more then one pertinent graph of $K$, hence the edges of $K$ into which the edges of $C$ projects form a cycle $\bar{C}$, and $x$ is in the pertinent graph of an edge $\bar{E} \notin V(\bar{C})$, and in $\mathcal{K}$ the edge $\bar{e}$ is on the wrong side of $\bar{C}$. 
		\end{enumerate}
	\end{defn}

	\begin{observ}
		If a skeleton $K$ is embedded in obviously wrong way, then the embedding $\mathcal{G}$ of $G$ induced by the skeletons does not extend $\mathcal{H}$.
	\end{observ}

	\begin{thm}
		If every skeleton of the SPQR-tree of $G$ is embedded in a way that is not obviously wrong, then the embedding $\mathcal{G}$ of $G$ induced by these skeleton embeddings extends $\mathcal{H}$.
	\end{thm}

	\begin{proof}
		For the two rules if they are satisfied then the extension is rather straightforward. That is if we have some cyclic order in $\mathcal{H}$ we just insert new edges in between to the new cyclic order in $\mathcal{G}$. For the outer vertex $x$ outside cycle $C$ it is even easier.
		
		But suppose that the first rule is not satisfied. Then in SPQR tree is unique node having three leaves in different part. This node and particularly its skeleton must have been obviously wrong.
		
		Now if the second rule is note satisfied then take path from $x$ towards $\bar{C}$. $y \in \bar{C}$ first met and $e,f \in \bar{C}$ having $y$ as one vertex and $g \notin \bar{C}$ be the last edge of the path. Find separation node. Let $x\in \bar{e} \notin \bar{C}$ where $\bar{e}$ is skeleton edge and $\bar{C}$ cycle in the skeleton of $C$. Then this $\mathcal{K}$ is obviously wrong embedding.
	\end{proof}

	\begin{thm}
		There is a polynomial algorithm which for a given $\mathcal{H}$ and SPQR-tree skeleton $K$ of $G$ determines whether $K$ has an embedding which is not obviously wrong.
	\end{thm}

	\begin{proof}
		\begin{enumerate}[I)]
			\item If $K$ is S-skeleton: it has 1 embedding, which is never obviously wrong.
			\item If $K$ is R-skeleton: it has 2 embeddings, so we just check the properties for both possible embeddings.
			\item If $K$ is P-skeleton with $d \geq 3$ edges: it has $(d-1)!$ embeddings. Do the following. $E_{xy} :=$ edges of $K$ which contains edge of $H$ incident ot $x$ and also edge of $H$ incident ot $y$ in their pertinent graph.
			
			\begin{enumerate}[1.]
				\item Determine cyclic order of $E_{xy}$ that is not obviously wrong.
				\item Insert into the cyclic order of $E_{xy}$ edges having an edge of $\mathcal{H}$ incident to $x$ or $y$, in a way that is not obviously wrong.
				\item For edge if $K$ that contains no edge of $\mathcal{H}$ incident to $x$ or $y$ in its pertinent graph: insert it in a way that is not obviously wrong (if possible).
			\end{enumerate}
		
			\noindent Note that if constraint related to cycle vertex position affects me, then it is because of a cycle $\bar{C}$ in $K$ where both edges of $\bar{C}$ are in $E_{xy}$.
		\end{enumerate}
	\end{proof}
\end{proof}
	\include{grg-ii/05-contact-discs}
	\include{grg-ii/06-separators}
\end{document}

