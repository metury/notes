\chapter{Cliques and independent sets}

We will show the computational complexity of the following optimization problems:

\begin{itemize}[]
	\item \textbf{Clique}
	\item \textit{Input:} A graph $G$.
	\item \textit{Output:} $\omega(G)$.
\end{itemize}

and

\begin{itemize} []
	\item \textbf{Independent Set}
	\item \textit{Input:} A graph $G$.
	\item \textit{Output:} $\alpha(G)$.
\end{itemize}

and their weighted variants

\begin{itemize}[]
	\item \textbf{Weighted Clique}
	\item \textit{Input:} A graph $G$ and a non-negative weight function $w : V(G) \to R_0^+$.
	\item \textit{Output:} A clique $C \subseteq V(G)$ which maximizes $w(C) = \sum_{u \in C} w(u)$.
\end{itemize}

and

\begin{itemize}[]
	\item \textbf{Weighted Independent Set}
	\item \textit{Input:} A graph $G$ and a non-negative weight function $w : V(G) \to R_0^+$.
	\item \textit{Output:} An independent set $A \subseteq V(G)$ which maximizes $w(A) = \sum_{u \in A} w(u)$.
\end{itemize}

Our goal is to show that for many of the intersection-defined graph classes that we have seen so far, these problems can be solved in polynomial time. For the sake of brevity, we denote by $\omega_w(G)$ the maximum possible weight $w(C)$ over all cliques $C$ in $G$, and by $\alpha_w(G)$ the maximum possible weight $w(A)$ over all independent sets $A$ in $G$.

\section{Interval graphs}

\begin{thm}
	\textbf{Weighted Clique} can be solved in polynomial time for interval graphs.
\end{thm}

\begin{proof}
	Interval graphs have only linearly many maximal cliques. We look at all of them and compare
	their weights.
\end{proof}

\begin{thm}
	\textbf{Weighted Independent Set} can be solved in polynomial time for interval graphs.
\end{thm}

\begin{proof}
	Suppose we are given an interval intersection representation $\mathcal{R} = \{I_u = [l_u , r_u] : u \in V\}$ of	a graph $G = (V, E)$, equipped with a weight function $w$. We may assume that all endpoints of the intervals are different. Number the endpoints $P_1 , P_2 , \dots , P_{2n}$ so that $P_i < P_{i+1}$ for $i = 1, 2, \dots , 2n - 1$.	Use dynamic programming to compute $w_i$ which is defined as the maximum possible weight of an	independent set $A$ in $G$ such that $r_u < P_i$ for all $u \in A$. This can be computed as follows:
	
	\begin{algorithm}[!ht]
		\begin{algorithmic}[1]
			\State $w_i := 0$
			\For{$i := 1$ to $2n$}
				\If{$P_i$ is the left endpoint of an interval}
					\State $w_{i+1} := w_i$
				\Else
					\State let $u \in V$ be such that $r_u = P_i$ and let $j$ be such that $P_j = l_u$;
					\State set $w_{i+1} = \max \{w_j + w(u), w_i\}$
				\EndIf
			\EndFor
			\State \Return $w_{2n+1}$
		\end{algorithmic}
	\end{algorithm}
\end{proof}

\begin{cor}
	\textbf{Weighted Clique} and \textbf{Weighted Independent Set} can both be solved in polynomial	time on co-INT graphs.
\end{cor}

\section{Comparability graphs}

\begin{thm}
	\textbf{Weighted Clique} can be solved in polynomial time for comparability graphs.
\end{thm}

\begin{proof}
	Given a transitive orientation $\overrightarrow{E}$ of $G = (V, E)$, order the vertices linearly $V = \{v_1 , v_2 , \dots, v_n\}$ in a topological sorting according to $\overrightarrow{E}$ (i.e., $v_i v_j \in \overrightarrow{E}$ implies $i < j$). For each $i$, set $W_i = \{j : v_j v_i \in E\}$ and let $w_i$ be the maximum weight $w(C)$ over all cliques $C \subseteq W_i \cup \{v_i\}$. Note that if $w(v_i) > 0$, each clique attaining the maximum weight contains $v_i$, and we may consider without loss	of generality only the cliques that contain $v_i$ even if $w(v_i) = 0$. The values $w_i , i = 1, 2, \dots , n$ can be computed recursively as follows
	
	\begin{algorithm}[!ht]
		\begin{algorithmic}[1]
			\For{$i := 1$ to $n$}
				\State $w_i := \max_{j \in W_i} w_j + w(v_i)$
			\EndFor
		\end{algorithmic}
	\end{algorithm}
	
	and clearly $\omega_w(G) = max_{i=1}^n w_i$.
\end{proof}

\begin{thm}
	\textbf{Weighted Independent Set} can be solved in polynomial time for \\ comparability graphs.
\end{thm}

\begin{proof}
	Given a transitive orientation $\overrightarrow{E}$ of $G = (V, E)$, consider the partial order $P = (V, \overrightarrow{E})$ determined by this orientation. The weighted version of Dilworth theorem yields that $\alpha_w(P)$ is equal to the minimum cost of a flow in the network $N = (V \cup \{s, t\}, E \cup \{su, ut : u \in V\})$ with vertex demands $f(u) \geq w(u)$, and this can be computed in polynomial time by network flow algorithms.
\end{proof}

\begin{cor}
	\textbf{Weighted Clique} and \textbf{Weighted Independent Set} can both be solved in polynomial time on function (= co-comparability) graphs.
\end{cor}

\section{Interval-filament graphs}

In this section we show the strongest results. Note, however, that we need the input graph to be given with an interval-filament representation (or, equivalently, with a partition of its edge set satisfying the mixing property).

\begin{thm}
	\textbf{Weighted Clique} can be solved in polynomial time for interval-filament graphs, if an interval-filament representation is provided on the input.
\end{thm}

\begin{proof}
	
\end{proof}