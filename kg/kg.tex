% Parameters of the paper.
\documentclass[12pt,a4paper]{report}
\setlength\textwidth{160mm}
\setlength\textheight{247mm}
\setlength\oddsidemargin{0mm}
\setlength\evensidemargin{0mm}
\setlength\topmargin{0mm}
\setlength\headsep{0mm}
\setlength\headheight{0mm}
\let\openright=\clearpage

%\usepackage[czech]{babel}  % Czech
\usepackage{babel}          % English
\usepackage{lmodern}        %
\usepackage[T1]{fontenc}    % Change the font
\usepackage{textcomp}       %

\usepackage[utf8]{inputenc} % Coding.

%%% Další užitečné balíčky (jsou součástí běžných distribucí LaTeXu)
\usepackage{amsmath}        % rozšíření pro sazbu matematiky
\usepackage{amsfonts}       % matematické fonty
\usepackage{amsthm}         % sazba vět, definic apod.
\usepackage{bbding}         % balíček s~nejrůznějšími symboly
\usepackage{bm}             % tučné symboly (příkaz \bm)
\usepackage{graphicx}       % vkládání obrázků
\usepackage{fancyvrb}       % vylepšené prostředí pro strojové písmo
\usepackage{indentfirst}    % zavede odsazení 1. odstavce kapitoly
\usepackage{natbib}         % zajištuje možnost odkazovat na~literaturu
\usepackage[nottoc]{tocbibind} % zajistí přidání seznamu literatury,
                            % obrázků a tabulek do~obsahu
\usepackage{icomma}         % inteligetní čárka v~matematickém módu
\usepackage{dcolumn}        % lepší zarovnání sloupců v~tabulkách
\usepackage{booktabs}       % lepší vodorovné linky v~tabulkách
\usepackage{paralist}       % lepší enumerate a itemize
\usepackage[dvipsnames]{xcolor}         % barevná sazba
                            %
\usepackage{caption}        % Packages pro subfigures.
\usepackage{subcaption}     %
\usepackage{pdfpages}       % Vkladani pdfka.
\usepackage{hyperref}       % Odkazy.
\usepackage{tikz}           % Grafy a obrázky.
\usetikzlibrary{positioning}
\usepackage{algpseudocode}  % Pseudokód
\usepackage{algorithm}
\usepackage{titling}
\usepackage{amssymb}

\overfullrule=1mm

\theoremstyle{plain}
\newtheorem{veta}{Věta}
\newtheorem{lemma}[veta]{Lemma}
\newtheorem{tvrz}[veta]{Tvrzení}

\theoremstyle{plain}
\newtheorem{definice}{Definice}
\newtheorem*{pozor}{Pozorování}
\newtheorem*{cvic}{Cvičení}
\newtheorem*{fakt}{Fakt}

\theoremstyle{remark}
\newtheorem*{dusl}{Důsledek}
\newtheorem*{pozn}{Poznámka}
\newtheorem*{prikl}{Příklad}

\theoremstyle{plain}
\newtheorem{thm}{Theorem}
\newtheorem{claim}[thm]{Claim}
\newtheorem{prop}[thm]{Proposition}
\newtheorem{conj}{Conjecture}

\theoremstyle{plain}
\newtheorem{defn}{Definition}
\newtheorem*{observ}{Observation}
\newtheorem*{notation}{Notation}
\newtheorem*{exerc}{Exercise}
\newtheorem*{fact}{Fact}
\newtheorem*{comm}{Comment}

\theoremstyle{remark}
\newtheorem*{cor}{Corollary}
\newtheorem*{cons}{Consequence}
\newtheorem*{rem}{Remark}
\newtheorem*{example}{Example}
\newtheorem*{note}{Note}

\newcommand{\R}{\mathbb{R}}
\newcommand{\N}{\mathbb{N}}
\newcommand{\Z}{\mathbb{Z}}

\DeclareMathOperator{\pr}{\textsf{P}}
\DeclareMathOperator{\E}{\textsf{E}\,}
\DeclareMathOperator{\var}{\textrm{var}}

\newcommand{\abs}[1]{\left|{#1}\right|}

\predate{\begin{center}\placetitlepicture\large}
	\postdate{\par\end{center}}

\newcommand{\titlepicture}[2][]{
	\renewcommand\placetitlepicture{
		\includegraphics[#1]{#2}\par\medskip
	}
}

\newcommand{\placetitlepicture}{}
\newcommand{\INFO}{Keep in mind there may be some mistakes. You may visit \href{https://github.com/metury/notes}{GitHub}.}
\newcommand{\ifft}{if and only if }
\newcommand{\wlogt}{without loss of generality }
\newcommand{\PES}{\faDog}
\newcommand{\OPEN}{\faDoorOpen}

\newenvironment{topic}[1]{\par\medskip\noindent\textbf{#1.}}{\par\medskip}

\newcommand{\TODO}[1]{\noindent\textcolor{WildStrawberry}{[TODO:} #1\textcolor{WildStrawberry}{]}}
\newcommand{\NOTE}[1]{\noindent\textcolor{CornflowerBlue}{[NOTE:} #1\textcolor{CornflowerBlue}{]}}
 % Use global macros.

\newcommand{\TODO}[1]{\begin{center}\textit{\textcolor{Emerald}{TODO: #1}}\end{center}}

\newcommand{\overrightsimarrow}[1]{\mathrel{\overset{\leadsto}{#1}}}

\title{Kombinatorika a grafy}
\author{Tomáš Turek}
\titlepicture[width=3in]{res/tree.pdf}
\date{\today}

\begin{document}
	\maketitle
	\tableofcontents
	\part{Kombinatorika a grafy I}
	\include{kg-i/odhady}
	\include{kg-i/vytvorujici-funkce}
	\include{kg-i/kpr}
	\chapter{Toky v sítích}

\begin{definice}
	Síť je čtveřice $(G, z, s, c)$, kde $G = (V, E)$ je orientovaný graf (tedy $V \subseteq V \times V$), $z \in V$ je \textbf{zdroj}, $s \in V$ je \textbf{stok} ($z \neq s$) a $c: E \to \mathbb{R}_{0}^{+}$. Hodnotu $c(e)$ nazýváme \textbf{kapacitou} hrany $e \in E$.
\end{definice}

\begin{definice}
	Tok v síti $(G = (V, E), z, s, c)$ je $f: E \to \mathbb{R}_{0}^{+}$ splňující následující podmínky:
	
	\begin{enumerate}
		\item $\forall r \in E : 0 \leq f(e) \leq c(e)$ (velikost toku je omezená kapacitou)
		\item \textbf{Kirchhoffův zákon} co přitéká do vrcholu, musí odtéct, neboli:
	\end{enumerate}
	
	$$
	\forall u \in V \setminus \{ z, s\}: \sum_{v:(u,v) \in E} f(u,v) - \sum_{v:(v, u) \in E} f(v,u) = 0
	$$
\end{definice}

\begin{definice}
	\textbf{Velikost toku} $f$ je $w(f) = \sum_{v:(z,v) \in E} f(z,v) - \sum_{v:(v, z) \in E} f(v,z)$.
\end{definice}

\begin{tvrz}
	Pro každou síť existuje maximální tok.
\end{tvrz}

\begin{dukaz}[Náčrt]
	Z analýzy víme, že spojitá funkce na kompaktní množině nabývá maxima. Množina $\mathcal{F} \subseteq \mathbb{R}^{|E|}$ všech toků je kompaktní funkce $w: \mathcal{F} \to \mathbb{R}$ je spojitá.
\end{dukaz}

\begin{definice}
	Řez v síti $(G, z, s, c)$ je $R \subseteq E$ taková, že každá orientovaná cesta ze zdroje $z$ do stoku $s$ používá aspoň jednu hranu z $R$.
\end{definice}

Speciálně hrany vycházející ze $z$ či hrany vycházející do $s$ tvoří řez.

\begin{definice}
	\textbf{Kapacita řezu} $R$ je $c(R) = \sum_{e \in R}c(e)$.
\end{definice}

Řezu je jen konečně mnoho $\Rightarrow$ jistě existuje řez minimální kapacity.

\begin{veta}[hlavní věta o tocích]
	Velikost maximálního toku = kapacita minimálního řezu, nebo-li, pro každou síť platí: $\max \ w(f) = \min \ c(R)$, kde $f$ je tok a $R$ řez.
\end{veta}

\begin{definice}[Elementární řez]
	Pro $A \subseteq V$, kde $z \in A$ a $s \notin A$, nazveme množinu $R_A = \{ e = (u,v) \in E: u \in A, v \notin A\}$ \textbf{elementárním řezem}. Opravdu se jedná o řez, protože pokaždé su musí nějak opustit $A$.
\end{definice}

\begin{pozor}
	Každý řez $R$ obsahuje elementární řez.
\end{pozor}

\begin{dukaz}
	Zvolme $A$ jako množinu vrcholů dosažitelných po orientované cestě ze zdroje v grafu $(V, E \setminus R)$. Potom $z \in A, s \notin A$, protože $R$ je řez $\Rightarrow R_A$ existuje $(u,v) \in R_A \Leftrightarrow u \in A, v \notin A \Rightarrow (u,v) \in R$, tedy $R_A \subseteq R$.
\end{dukaz}

\begin{pozor}
	Každý \textbf{v inkluzi minimální} řez $R$ je elementární. Nebo-li $R \setminus \{ e \}$ není řezem pro $\forall e \in R$.
\end{pozor}

\begin{dukaz}
	Z předchozího pozorování musí $R$ obsahovat elementární řez $R_A \subseteq R$ a z minimality platí $R_A = R$.
\end{dukaz}

\begin{lemma}
	Je-li $f$ tok a $R_A$ elementární řez, pak platí:
	
	$$
	w(f) = \sum_{u \in A, v \notin A, (u,v) \in E}f(u,v) - \sum_{u \in A, v \notin A, (v, u) \in E} f(v,u)
	$$
\end{lemma}

\begin{dukaz}
	Empty.
\end{dukaz}

\begin{dukaz} (Věty)
	Empty.
\end{dukaz}

\section{Fordův-Fulkersonův algoritmus}

\begin{algorithmic}[1]
	\State Nastav $f(e) = 0$ pro $\forall e \in E$
	\While {$\exists$ zlepšující cesta $P$}
	\State  vylepšuj po ní tok o $\epsilon_P$
	\EndWhile\\
	\Return Stávající tok $f$
\end{algorithmic}

\begin{veta}[o celočíselnosti]
	Jsou-li kapacity celočíselné, pak F.F. najde max. tok po konečně mnoha krocáích a navíc má takový tok celočíselnou velikost.
\end{veta}

\begin{dukaz}
	Tok se vždy zlepší o celé číslo $\epsilon_P >$ a $w(f) < \infty$.
\end{dukaz}

Existují sítě s iracionálními kapacitami, kde F.F nenajde maximální tok a nekonverguje k výsledku. V síti s celočíselnými kapacitami má F.F. alg. časovou složitost $O(w(f)(|V|+|E|))$, kde $f$ je tok. Takže je to v čase $O(|V|+|E|)$. Pokud bychom specifikovali výběr zlepšující cesty na nejkratší dostaneme \textbf{Edmondsův-Karpův algoritmus}, který má časovou složitost $0(|V|+|E|^2)$.

\section{Königova-Egerváryho věta}

\begin{definice}
	V grafu $G = (V,E)$ nazveme množinu $C \subseteq V$ \textbf{vrcholovým pokrytím}, pokud $C \cap e \neq \emptyset$ pro $\forall e \in E$.
\end{definice}

Zjistit minimální velikost vrcholové pokrytí je NP-těžká úloha.

\begin{definice}
	\textbf{Párováním} v $G$ je podgraf tvořený disjunktními hranami.
\end{definice}

\begin{veta}(Königova-Egerváryho věta)
	V bipartitiním grafu je velikost min. vrcholového pokrytí rovna velikosti maximálního párování (do počtu hran).
\end{veta}

\begin{dukaz}
	Empty.
\end{dukaz}

\section{Hallova věta}

\begin{definice}
	Mějme konečné množiny $X$ a $I$. \textbf{Množinový systém} $\mathcal{M}$ je $(M_{i}: i \in I)$, kde $M_{i} \subseteq X$. \textbf{Systém ruzných reprezentantů (SRR)} pro $\mathcal{M}$ je prosté zobrazení $f: I \to X$ takové, že $\forall i \in I : f(i) \in M_{i}$. Tedy $f$ je výběr jednoho prvku z každé $M_{i}$ takový, že žádný prvek nevybereme víckrát. \textbf{Incidenční graf} systému $\mathcal{M}$ je bipartitní graf $G_{\mathcal{M}}=(I \cup X, E)$, kde $E = \{ \{ i,x\}: i \in I, x \in X, x \in M_{i} \}$. Pokud $\mathcal{M}$ má SSR $\Leftrightarrow$ $S_{\mathcal{M}}$ obsahuje párování velikosti $|I|$.
\end{definice}

\begin{veta}[Hallova věta]
	$\mathcal{M}$ má SSR $\Leftrightarrow \forall J \subseteq I: |\cup_{j \in J}M_{j}| \geq |J|$. Pravé části se říká \textbf{Hallova podmínka}, také se věta označuje jako \textbf{Hall's marriage theorem}.
\end{veta}

\begin{definice}
	Empty.
\end{definice}


S axiomem výběru by šlo dokázat variantu s konečnými $M_{i}$ a nekonečnými $I,X$. S nekonečnými $I,X$ to platit nemusí.

\section{Rozšiřování latinských obdélníků}

\begin{dusl}
	V každém bipartitním grafu $G =  (A \cup B, E)$ s $E \neq \emptyset$ a $\mathtt{deg}_{G}(x) \geq \mathtt{deg}_{G}(y)$ pro každé $x \in A, y \in B$ existuje párování velikosti $|A|$.
\end{dusl}

\begin{dukaz}
	Empty.
\end{dukaz}

Latinský obdélník typu $k \times n$ pro $k \leq n$ je tabulka s řádky s $n$ sloupci vyplněnými symboly $1, \dots, n$ tak, že se v žádném řádku ani sloupci žádný symbol neopakuje.

\begin{veta}
	Každý latinský obdélník typu $k \times n$ lze doplnit na latinský čtverec řádu  $n$.
\end{veta}

\begin{dukaz}
	Empty.
\end{dukaz}
	\include{kg-i/souvislost}
	\include{kg-i/pocitani-dvema-zpusoby}
	\include{kg-i/ramsey}
	\include{kg-i/samoopravne-kody}
	\part{Kombinatorika a grafy II}
	\include{kg-ii/parovani}
	\include{kg-ii/minory}
	\include{kg-ii/plochy}
	\include{kg-ii/barveni}
	\include{kg-ii/extremalni-komb}
	\include{kg-ii/vytvorujici-funkce}
	\include{kg-ii/orbity}
	\part{Kombinatorika a grafy III}
	\chapter{Structural graph theory}

\begin{defn}
	$H \leq_{t} G$ means that subdivision of $H$ is a subgraph of $G$, also known as \textbf{topological minor}.
\end{defn}

\begin{defn}
	$H \leq_{m} G$ means that $H$ is a \textbf{minor} of $G$.
\end{defn}

\begin{defn}
	$H \subseteq G$ means that $H$ is a \textbf{subgraph} of $G$.
\end{defn}

\begin{defn}
	$H \sqsubseteq G$ means that $H$ is a \textbf{induced subgraph} of $G$.
\end{defn}

\begin{thm}[Kuratowski]
	$$
	K_{5}, K_{3,3} \nleq_{t} G \Leftrightarrow G \text{ planar}
	$$
	
	$$
	K_{5}, K_{3,3} \nleq_{m} G \Leftrightarrow G \text{ planar}
	$$
\end{thm}

\begin{defn}
	$\chi(G)$ means that $G$ has a coloring of size $\chi(G)$.
\end{defn}

\begin{observ}
	$C_{3}, C_{5}, C_{7}, \dots \nsubseteq G \Leftrightarrow \chi(G) \leq 2$ which holds also for $\sqsubseteq$.
\end{observ}

\begin{observ}
	$C_{3} \nleq_{m} G \Leftrightarrow G \text{ is a forest}$ also holds for $\leq_{t}$.
\end{observ}

\begin{defn}
	$Forb_{leq} (\mathcal{F}) = \{G | (\forall F \in \mathcal{F}) F \nleq G\}$
\end{defn}

We will try to show $\mathcal{G} = Forb_{\leq_{m}} (\mathcal{F})$. If $G \in \mathcal{G}$ then all minors of $G$ belong to $\mathcal{G}$.

\begin{observ}
	If $\mathcal{G} = Forb_{\leq}(\mathcal{F})$ then $\mathcal{G}$ is $\leq$-closed. Which means that $\forall G,G'$ if $G \in \mathcal{G}$ and $G' \leq G$ then $G' \in \mathcal{G}$.
\end{observ}

\begin{lemma}
	Let $\leq$ be a partial ordering of graphs. If a class $\mathcal{G}$ of graphs is $\leq$-closed, then there exist $\mathcal{F}$ s.t. $\mathcal{G} = Forb_{\leq}(\mathcal{F})$.
\end{lemma}

\begin{proof}
	$\mathcal{F} = \{F : F \nleq G\}$.
\end{proof}

\begin{defn}
	$F$ is \textbf{minimal $\leq$-obstruction} for $\mathcal{G}$ if $F \notin \mathcal{G}$ but for every $F' \lneq F$ and $F' \in \mathcal{G}$.
\end{defn}

\begin{lemma}
	Let $\leq$ be an ordering og graphs \textbf{without infinite decreasing chains}. If $\mathcal{F}$ is $\leq$-closed, then $\mathcal{G} = Forb_{\leq}(\{F : F \text{ is a minimal } \leq\text{-obstruction for } \mathcal{G}\})$.
\end{lemma}

\begin{proof}
	$G \notin \mathcal{G}$ is min $\leq$-obstruction or $\exists G' \lneq G : G \notin \mathcal{G} \Rightarrow G'$ is obstruction or we continue and because we don't have \textbf{without infinite decreasing chains} we will eventually end.
\end{proof}

If $\mathcal{G}$ is $\leq_{m}$-closed, then there exists a \textbf{finite} $\mathcal{F}$ such that $\mathcal{G} = Forb_{\leq_{m}}(\mathcal{F})$.

\begin{thm}[Robertson-Seymor]
	For every $F$ there exists an algorithm that for input graph $G$ decides whether $F \leq_{m} G$ in time $O_{F}(|G|^{3})$.
\end{thm}

\begin{defn}
	For graph $G = (V,E)$ we define $|G| = |V|$ and $||G|| = |E|$. Also for some $U \subseteq V$ $G[U]$ is a induced subgraph of $G$ that has only vertices from $U$. Then $N_{G}(v)$ stands for the neighborhood of vertex $v$ in graph $G$.
\end{defn}

\begin{defn}
	$G'$ is a \textbf{cover} of $G$ if $(\exists f : V(G') \to V(G)) \forall v \in V(G')$ for $N_{G'}(v)$ is a bijection with $N_{G}(f(v))$.
\end{defn}

\begin{example}
	We may see an example \ref{covers}:
	\begin{figure}[!h]\centering
		\begin{tikzpicture}[node distance={10mm}, thick, main/.style = {draw, circle}]
			\node[main] (1) {1};
			\node[main] (2) [above right of=1] {2};
			\node[main] (3) [below right of=2] {3};
			\node[main] (4) [below of=1] {4};
			\node[main] (5) [below of=3] {5};
			\draw (1) -- (2);
			\draw (1) -- (3);
			\draw (1) -- (5);
			\draw (2) -- (3);
			\draw (2) -- (3);
			\draw (2) -- (4);
			\draw (2) -- (5);
			\draw (3) -- (4);
			\draw (4) -- (5);
			\draw (1) -- (4);
			\draw (3) -- (5);
			\node[main] (6) [right of=3] {1};
			\node[main] (7) [below right of=6] {4};
			\node[main] (8) [below right of=7] {2};
			\node[main] (9) [below right of=8] {1};
			\node[main] (10) [below right of=9] {4};
			\node[main] (11) [below left of=8] {5};
			\node[main] (12) [below left of=11] {3};
			\node[main] (13) [above right of=8] {3};
			\node[main] (14) [above right of=13] {5};
			\node[main] (15) [above right of=6] [above left of=14] [above of=8] {2};
			\draw (6) -- (7);
			\draw (7) -- (8);
			\draw (8) -- (9);
			\draw (8) -- (11);
			\draw (8) -- (13);
			\draw (9) -- (10);
			\draw (11) -- (12);
			\draw (13) -- (14);
			\draw (7) -- (13);
			\draw (13) -- (9);
			\draw (9) -- (11);
			\draw (11) -- (7);
			\draw (6) -- (14);
			\draw (14) -- (10);
			\draw (10) -- (12);
			\draw (12) -- (6);
			\path
				(6) edge [bend left] (15)
				(14) edge [bend right] (15)
				(10) edge [bend right] (15)
				(12) edge [bend left] (15);
		\end{tikzpicture}
		\caption{Example of $G$ and $G'$ as covers.}
		\label{covers}
	\end{figure}
\end{example}

$$
\begin{array}{c}
	\{G " \exists \text{ planar } G' \text{ cover of } G \} \\
	\Updownarrow \\
	F_{1}, \dots, F_{n} \nleq_{m} G
\end{array}
$$

Contrary we take $\mathcal{G} = \{G : (\forall uv \in V(G) U \neq v, \deg(u) \geq 5, \deg(v) \geq 5) (\exists X \subseteq E(G) : |X| \leq 1) u \text{ and } v \text{ are in different component of } G - X\}$ which is $\leq_{t}$-closed. But take these graphs:

\begin{figure}[!h]\centering
	\begin{tikzpicture}[node distance={10mm}, thick, main/.style = {draw, circle}]
		\node[main] (1) {};
		\node[main] (2) [above right of=1] {};
		\node[main] (3) [below right of=2] {};
		\node[main] (4) [right of=3] {};
		\node[main] (5) [above right of=4] {};
		\node[main] (6) [below right of=5] {};
		\node[main] (7) [above right of=6] {};
		\node[main] (8) [below right of=7] {};
		\path
			(1) edge [above] (2)
			(2) edge [above] (3)
			(1) edge [below] (3);
		\path
			(4) edge (5)
			(5) edge (6)
			(6) edge (7)
			(7) edge (8)
			(4) edge (6)
			(6) edge (8);
	\end{tikzpicture}
	\label{endless}
	\caption{Obstructions.}
\end{figure}

Where each one of them is an obstruction. And we could create much more of them.

Now we take a look at some nice properties of graphs if we forbid some graphs as a minors.

$$
\begin{array}{r c l}
	K_{1} \nleq_{m} G & \Leftrightarrow & V(G) = \emptyset \\
	K_{2} \nleq_{m} G & \Leftrightarrow & E(G) = \emptyset \\
	K_{3} \nleq_{m} G & \Leftrightarrow & G \text{ is a forest } \\
	                  &                 & G \text{ is obtained from } K_{1}, K_{2} \text{ by clique sums} \\
	K_{4} \nleq_{m} G & \Leftrightarrow & G \text{ is obtained from } K_{1}, K_{2}, K_{3} \text{ by clique sums} \\
\end{array}
$$

\begin{defn}
	Graph $G$ can be obtained from $G_{1}$ and $G_{2}$ by \textbf{clique-sum} if the inter-\newline section that these graphs have in $G$ form a clique. In other way it is that we bind together two graphs by identifying their vertices and edges in the same size clique. Sometimes we may denote it as $G_{1} \bigoplus G_{2} = G$.
\end{defn}

\begin{observ}
	If $G$ is obtained from $G_{1}$ and $G_{2}$ by a clique-sum then:
	
	$$
	K_{m} \leq_{m} G \Leftrightarrow K_{m} \leq_{m} G_{1} \lor K_{m} \leq_{m} G_{2}
	$$
\end{observ}

\begin{lemma}
	If $K_{k} \leq_{m} G$ and $G$ is the clique-sum of $G_{1}$ and $G_{2}$ then $K_{k} \leq_{m} G_{1} \lor K_{k} \leq_{m} G_{2}$.
\end{lemma}

\begin{lemma}
	If $G$ is not $3$-connected then there exist $G_{1}, G_{2} \lneq_{m} G$ s.t. $G$ is a clique-sum of $G_{1}$ and $G_{2}$.
\end{lemma}

\begin{proof}
	If $G$ is not connected then it is done since it is a clique sum on $K_{0}$. If $G$ is connected, but not $2$-connected then it is a clique-sum on $K_{1}$ since there exist a articulation. If $G$ is $2$-connected then there must be two vertices which splits the graph. And these two vertices form a $K_{2}$ as a minor. That is because we split $G$ to two parts where we leave the major one side and add a edge to these two vertices, which we can do because they need to have a path between them so we contract all the edges alongside the path.
\end{proof}

\begin{defn}
	$\delta(G)$ is a minimum degree of a graph $G$.
\end{defn}

\begin{thm}
	If $G$ is $K_{4}$-minor-free then $G$ is obtained from $K_{\leq 3}$'s by clique-sums.
\end{thm}

\begin{proof}
	By induction on $|V(G)|$.
	
	\begin{enumerate}[(a)]
		\item If $G$ is not 3-connected. $G$ is a clique-sum of $G_{1},G_{2} \lneq_{m} G$. Since $K_{4} \nleq_{m} G_{1}$ and $K_{4} \nleq_{m} G_{2}$ we use induction hypothesis and we are done.
		\item If $G$ is 3-connected. If $|V(G)| \leq 3$, then $G = K_{\leq 3}$, wlog $|V(G)| \geq 4$. $\delta(G) > 1 \Rightarrow G$ contains a cycle. Let $C$ be a shortest cycle in $G$. $C$ is induced in $G$ 3-connected $\Rightarrow G \neq C$ so $\exists v \in V(G) \setminus V(C)$. By Merger's theorem there exists three paths from $v$ to $C$ intersecting only in $v$. That gives us $K_{4}$ as a minor of the graph. Which is contradiction.
	\end{enumerate}
\end{proof}

$$
\begin{array}{r c l}
	K_{5} \nleq_{m} G & \Leftrightarrow & G \text{ is obtained from planar graphs and } W_{8}  \text{ by clique sums} \\
\end{array}
$$

\begin{figure}[!h]\centering
	\begin{tikzpicture}[node distance={10mm}, thick, main/.style = {draw, circle}]
		\node[main] (1) {};
		\node[main] (2) [above right of=1] {};
		\node[main] (3) [below right of=2] {};
		\node[main] (4) [below left of=1] {};
		\node[main] (5) [below right of=3] {};
		\node[main] (6) [below left of=5] {};
		\node[main] (7) [below right of=4] {};
		\node[main] (8) [below right of=7] {};
		\draw (1) -- (2);
		\draw (4) -- (1);
		\draw (4) -- (7);
		\draw (7) -- (8);
		\draw (8) -- (6);
		\draw (6) -- (5);
		\draw (3) -- (5);
		\draw (2) -- (3);
		\draw (2) -- (8);
		\draw (1) -- (6);
		\draw (4) -- (5);
		\draw (7) -- (3);
	\end{tikzpicture}
	\caption{$W_{8}$ graph.}
	\label{w8}
\end{figure}

\begin{observ}
	If $G$ is a clique-sum of $G_{1}$ and $G_{2}$ then
	
	$$
	\chi(G) \leq \max (\chi(G_{1}), \chi(G_{2}))
	$$
\end{observ}

\begin{proof}
	We just need to match the coloring of the cliques. Other than that we don't have any problem.
\end{proof}

\section{Hadwiger's conjecture}

$K_{t}$-minor-free graphs are $(t-1)$ colorable.

$$
\begin{array}{c c c}
	K_{1} \nleq_{m} G & \chi \leq 1 & \delta \leq 0 \\
	K_{2} \nleq_{m} G & \chi \leq 2 & \delta \leq 1 \\
	K_{3} \nleq_{m} G & \chi \leq 3 & \delta \leq 2 \\
	K_{4} \nleq_{m} G & \chi \leq 4 & \delta \leq 5 \\
	K_{5} \nleq_{m} G & \chi \leq 5 & \\
\end{array}
$$

\begin{thm}
	$\exists f$ every $K_{t}$-minor-free graph $G$ has $\delta(G) \leq f(t)$.
\end{thm}

The function is somewhere near $f(t) = (1,6\dots + O(1)) t \sqrt{\log t}$. But we won't show this result. Instead we will show $f(t) = O(t^{2})$. Before we continue it is better to remind ourselves \textbf{chordal graph} and \textbf{elimination ordering} (known as PES).

\begin{defn}[Chordal decomposition of $G$]
	$V(G) = \mathcal{P}_{1} \dot{\cup} \mathcal{P}_{2} \dot{\cup} \dots \dot{\cup} \mathcal{P}_{n} \dot{\cup}$ and
	
	\begin{enumerate}
		\item $(\forall i) G[\mathcal{P}_{i}]$ is connected.
		\item "$\mathcal{P}_{i}$'s form elimination ordering" Precisely: $(\forall i \in [n])(forall j_{1},j_{2} < i)$ if $G$ has an edge between $\mathcal{P}_{i}$ and $\mathcal{P}_{j_{1}}$ and also between $\mathcal{P}_{i}$ and $\mathcal{P}_{j_{2}}$ then it also has an edge between $\mathcal{P}_{j_{1}}$ and $\mathcal{P}_{j_{2}}$.
	\end{enumerate}
\end{defn}

\begin{defn}
	Chordal partition is \textbf{geodesic} if $(\forall i) (\exists v_{i} \in \mathcal{P}_{i})$ s.t. if $v_{1}, \dots, v_{t} < i$ are the indices s.t. $G$ has an edge between $\mathcal{P}_{i}$ and $\mathcal{P}_{j_{1}}, \mathcal{P}_{j_{2}}, \dots, \mathcal{P}_{j_{t}}$ then $v_{1}, \dots, v_{t} \in \mathcal{P}_{i}$ s.t. $v_{i}$ has a neighbor in $\mathcal{P}_{j_{1}}, \mathcal{P}_{j_{2}}, \dots, \mathcal{P}_{j_{t}}$ and $G - \bigcup_{j < i} \mathcal{P}_{j}$ contains shortest paths from $v_{i}$ to $v_{1}, \dots, v_{t}$ which cover all vertices in $\mathcal{P}_{i}$.
\end{defn}

% TODO perhaps add an image

\begin{thm}
	Every graph has a geodesic chordal partition.
\end{thm}

Before we show us a proof we will take a look at a simple application. If $G$ is $K_{k}$-minor-free last part has neighbours in $t \leq k-2$ parts (otherwise it will have $K_{k}$ as a minor). Then we may take a look at a $\deg(v) \leq (k-2) + (k-2)(k-2)3 \leq 3k^{2}$. Thus getting the upper bound $\delta(G) \leq 3k^{2}$.

\begin{defn}
	Part is called \textbf{terminal} if there is no edge from any vertex in that part going to some vertex in one of the parts on the right.
\end{defn}

\begin{proof}
	Let $\mathcal{P}$ be a chordal decomposition of $G$ into parts satisfying both properties of definition of chordal decomposition (i) abd (ii) and geodesity (iii) for all non-terminal parts.
	
	This can be easily done by creating parts based on the components of connectivity. For them all properties hold, since they are all connected and "chordal" property is also satisfied since there are no edges. Also all of them are terminal (iii) doesn't have to be satisfied.
	
	Now we proof by that by choosing $\mathcal{P}$ with largest number of parts. Lets say that there is a part that does not satisfy (iii). This means that it is terminal part. Lets take vertex from the part and find the shortest paths to the vertices that are connected to some of the parts to the left. Now we put vertices to separate components and these components will make a new parts. We will also remove all these vertices from the origin part. Note that all properties are satisfied. (i) is trivial. (ii) If there are any vertices from the new parts to other parts then they are to the ones which are already connected to the origin part, which satisfied (ii) before so it is fine. Also (iii) is satisfied.
	
	The thing is that we created $\mathcal{P}$ with larger number of parts which is contradiction.
\end{proof}

\begin{observ}
	$H \leq_{t} G \Rightarrow H \leq_{m} G$
\end{observ}

\begin{observ}
	$\Delta (H) \leq 3 : H \leq_{m} G \Rightarrow H \leq_{t} G$
\end{observ}

Lets remind ourselves a table and add some new thinks.

$$
\begin{array}{ r c l}
	K_{1} \nleq_{t} G \Leftrightarrow K_{1} \nleq_{m} G & \Leftrightarrow & V(G) = \emptyset \\
	K_{2} \nleq_{t} G \Leftrightarrow K_{2} \nleq_{m} G & \Leftrightarrow & E(G) = \emptyset \\
	K_{3} \nleq_{t} G \Leftrightarrow K_{3} \nleq_{m} G & \Leftrightarrow & G \text{ is a forest } \\
	&                 & G \text{ is obtained from } K_{1}, K_{2} \text{ by clique sums} \\
	K_{4} \nleq_{t} G \Leftrightarrow K_{4} \nleq_{m} G & \Leftrightarrow & G \text{ is obtained from } K_{1}, K_{2}, K_{3} \text{ by clique sums} \\
	K_{5} \nleq_{t} G \nLeftrightarrow K_{5} \nleq_{m} G & \Leftrightarrow & G \text{ is obtained from planar graphs and } W_{8}  \text{ by clique sums} \\
\end{array}
$$

Well technically $K_{5} \nleq_{t} G \Rightarrow K_{5} \nleq_{m} G$ but the other way around is what doesn't work $K_{5} \nleq_{m} G \nRightarrow K_{5} \nleq_{t} G$. For that we can see an example \ref{minor-not-top}. We may see that $\mathcal{G} = \{G : G \text{ has } \leq 4 \text{ vertices of degree } \geq 4\}$ these graphs are so that $K_{5} \nleq_{t} G$.

\begin{figure}[!ht]\centering
	\begin{tikzpicture}[node distance={10mm}, thick, main/.style = {draw, circle}]
		\node[main] (1) {};
		\node[main] (2) [right of=1] {};
		\node[main] (3) [below of=1] {};
		\node[main] (4) [below of=2] {};
		\node[main] (5) [below of=3] {};
		\node[main] (6) [below of=4] {};
		\draw (1) -- (2);
		\draw (1) -- (3);
		\draw (4) -- (2);
		\draw (3) -- (4);
		\draw (5) -- (4);
		\draw (6) -- (4);
		\draw (6) -- (5);
		\draw (3) -- (5);
		\draw (3) -- (6);
		\path
			(1) edge [bend right] (5)
			(2) edge [bend left] (6);
	\end{tikzpicture}
	\caption{A counter example.}
	\label{minor-not-top}
\end{figure}

\section{Hájos conjecture}

If we remember Headwiger's conjecture then Hájos conjecture is the same only with topological minors. Thus it is that $K_{t} \leq_{t} G \Rightarrow \chi(G) \leq t-1$. This is actually true for $t < 4$ but it is false for $t \geq 7$ and $5,6$ are open questions.

\begin{thm}
	$\exists f_{m}(k) = O(k \sqrt{\log k})$ Every $K_{k}$-minor-free graph $G$ satisfies $\delta(k) \leq f_{m}(k)$.
\end{thm}

We won't proof this, but we will proof something similiar, that is for topological minors.

\begin{thm}
	$\exists f_{t}(k) = O(k^{2})$ Every $G$ s.t. $K_{k} \lneq_{t} G$ satisfies $\delta(G) \leq f_{t}(k)$.
\end{thm}

The corollary to this is that $\chi)G_ \leq f_{t}(k) +1$. We will proof this theorem, but to do that we need to do some steps beforehand.

Firstly imagine that the enemy gives you a graph and you need to prove that. But the enemy is kind enough to give you a graph $H$ with connectivity $>> k^2$. We could apply Merger's theorem. Though this will only give certain number of vertex disjoint paths from one vertex to another. We would more likely have this many paths between more pairs of sources and targets.

\begin{defn}
	Graph $G$ is \textbf{$k$-linked} if $|V(G)| \geq 2k$ and $\forall s_{1}, s_{2}, \dots, s_{k}, t_{1}, t_{2}, t_{k}$ distinct vertices of $G$. $G$ contains pairwise vertex-disjoint paths $P_{1}, P_{2}, \dots, P_{k}$. When $P_{i}$ has ends $s_{i}$ and $t_{i}$.
\end{defn}

We may see that there exist a graph that is 2-connected and yet not 2-linked. You may see this on the picture \ref{2-linked-2}. Also not even 3-connected graph has to be 2-linked. Which is also on the picture \ref{2-linked-3} (though we can change the vertex inside for any planar graph). We could continue and end up with that not even 5-connectivity forces 2-linked.

\begin{figure}[!ht]\centering
	\begin{subfigure}[b]{0.4\textwidth}\centering
		\begin{tikzpicture}[node distance={20mm}, thick, main/.style = {draw, circle}]
			\node[main] (1) {$s_{1}$};
			\node[main] (2) [below left of=1] {$s_{2}$};
			\node[main] (3) [below right of=1] {$t_{2}$};
			\node[main] (4) [below right of=2] {$t_{1}$};
			\draw (1) -- (2);
			\draw (2) -- (4);
			\draw (3) -- (4);
			\draw (3) -- (1);
		\end{tikzpicture}
		\caption{2-connectivity}
		\label{2-linked-2}
	\end{subfigure}
	\begin{subfigure}[b]{0.4\textwidth}\centering
		\begin{tikzpicture}[node distance={20mm}, thick, main/.style = {draw, circle}]
			\node[main] (1) {$s_{1}$};
			\node[main] (2) [below left of=1] {$s_{2}$};
			\node[main] (3) [below right of=1] {$t_{2}$};
			\node[main] (4) [below right of=2] {$t_{1}$};
			\node[main] (5) [below of=1] {};
			\draw (1) -- (2);
			\draw (2) -- (4);
			\draw (3) -- (4);
			\draw (3) -- (1);
			\draw (1) -- (5);
			\draw (2) -- (5);
			\draw (3) -- (5);
			\draw (4) -- (5);
		\end{tikzpicture}
		\caption{3-connectivity}
		\label{2-linked-3}
	\end{subfigure}
	\caption{A counter example to 2-linked graphs.}
	\label{2-linked}
\end{figure}

\begin{observ}
	Every $k$-linked graph is $(2k-1)$ connected.
\end{observ}

\begin{proof}
	That is simply because we put all the $s_{i}, t_{i}$ for $i \in [k-1]$ to the edge cut and then choose $s_{k}$ in the left part and $t_{k}$ in the right part then we can see that it is indeed $(2k-1)$-connected.
\end{proof}

\begin{thm}\label{2-linked-thm}
	If $G$ is $2k$-connected, $K_{4k} \leq_{m} G$ then $G$ is $k$-linked.
\end{thm}

We won't prove this directly. Instead we will later on introduce another theorem that is actually pretty much the same and prove that.

\begin{cor}
	If $G$ is $\max(2k, f_{m}(4k)+1)$-connected then $G$ is $K$-linked.
\end{cor}

\begin{proof}
	We use the theorem to get that $\delta > f_{m}(4k)$ thus $K_{4k} \leq_{m} G$.
\end{proof}

Also we can say $\exists f_{l}(k) = O(k \sqrt{\log k})$. If $G$ is $f_{l}(k)$-connected then $G$ is $k$-linked.

\begin{cor}
	If $G$ is $f_{l}\left( \frac{k (k-1)}{2}\right)$-connected then $K_{k} \leq_{t} G$.
\end{cor}

\begin{proof}
	To see this we choose $k$ vertices and for every one of them $k-1$ neighbors. Then we give $s_{i}$ and $t_{i}$ to every single one of these vertex so that every neighborhood has pair with all others. Then we find such paths between them.
\end{proof}

\begin{lemma}
	If $\bar{d}(G) \geq 4d$ then $G$ contains a $(d+1)$-connected subgraph $H$ of minimum degree $2d+1$.
\end{lemma}

\begin{proof}
	Let $H$ be a minimal subgraph of $G$ s.t. $|V(H)| \geq 2d$ and $|E(H)| > 2d (|V(H)| - d)$. We may see that $|V(H)| > 2d$ that is if it has $2d$ vertices then
	
	$$
	\frac{2d^{2} - d}{2} = \binom{2d}{2} > |E(H)| > 2d^{2}
	$$
	
	which is a contradiction.
	
	Then we also have that $\delta (H) \geq 2d+1$. If we have $\delta(H) \leq 2d$ we may remove the certain vertex. But we need to show that given properties still hold. We will split the graph to two parts $|A|, |B| \geq 2d+2 > 2d$. Then
	
	$$
	\begin{array}{r l}
		|E(G)| & \leq |E(A)| + |E(B)| \\
		(1) & \leq 2d(|V(A)| - d) + 2d(|V(A)| - d) \\
		& = 2d(|V(A)| + |V(B)| - 2d) \\
		& = 2d(|V(H)| - |V(A \cap B)| - 2d) \\
		|E(G)| & > 2d(|V(H)| - d)
	\end{array}
	$$
	
	Where $(1)$ is due to the minimality of $H$. The thing is with the last two lines we get that $|A \cap B| > d$.
\end{proof}

\begin{proof}
	This actually is enough for the theorem to be proven since the enemy doesn't have to be kind anymore.
\end{proof}

\begin{defn}
	A \textbf{model} of $K_{m}$ in $G$ is $M_{1}, M_{2}, \dots, M_{m} \subseteq V(G)$ pairwise distinct and $\forall i : G[M_{i}]$ is connected and $(\forall i \neq j) \exists uv \in E(G): u \in M_{i}, v\in M_{j}$.
\end{defn}

We may take a look at an example of model of $K_{4}$ which is in picture \ref{k4-model}.

\begin{figure}[!ht]\centering
	\begin{tikzpicture}[node distance={10mm}, thick, main/.style = {draw, circle, fill}]
		\node[main] (1) [color=cyan] {};
		\node[main] (2) [above of=1] [color=cyan] {};
		\node[main] (3) [below left of=1] [color=cyan] {};
		\node[main] (4) [below right of=1] [color=cyan] {};
		\node[main] (5) [left of=1] [color=cyan] {};
		\node[main] (6) [right of=1] [color=cyan] {};
		\node[main] (7) [right of=6] [color=cyan] {};
		\node[main] (8) [below right of=6] [below right of=4] [color=orange] {};
		\node[main] (9) [below left of=5] [below left of=3] [color=blue] {};
		\node[main] (10) [left of=9] [color=blue] {};
		\node[main] (11) [below of=10] [color=blue] {};
		\node[main] (12) [below of=9] [color=blue] {};
		\node[main] (13) [below right = 1.8cm of 12] [color=purple] {};
		\node[main] (14) [below left of=13] [color=purple] {};
		\node[main] (15) [below right of=13] [color=purple] {};
		\path[color=blue] (9) edge (10)
		(9) edge (12)
		(10) edge (11)
		(11) edge (12);
		\path[color=cyan] (1) edge (2)
		(1) edge (3)
		(1) edge (4)
		(1) edge (5)
		(1) edge (6)
		(6) edge (7);
		\path[color=purple] (13) edge (14)
		(13) edge (15)
		(14) edge (15);
		\path (8) edge (15)
		(8) edge (6)
		(8) edge (12)
		(13) edge (4)
		(14) edge (12)
		(10) edge (5);
	\end{tikzpicture}
	\caption{Example of $K_{4}$ model.}
	\label{k4-model}
\end{figure}

\begin{observ}
	$K_{m} \leq_{m} G \Leftrightarrow \text{there is a mode of } K_{m} \text{ in } G$.
\end{observ}

\begin{defn}
	\textbf{Separation} in $G$ is $(A,B)$ where $A,B \subseteq V(G), A \cup B = V(G)$, no edge between $A \setminus B$ and $B \setminus A$.
\end{defn}

On picture \ref{separation} we may see an example of $(A,B)$-separation. Where the \textcolor{orange}{orange} points are both in $A$ and $B$ and then the rest is either only in one or second part, which are set by their connectivity.

\begin{figure}[!ht]\centering
	\begin{tikzpicture}[node distance={10mm}, thick, main/.style = {draw, circle, fill}]
		\node[main] (1) [color=orange] {};
		\node[main] (2) [color=orange] [below of=1] {};
		\node[main] (3) [right of=1] [color=cyan] {};
		\node[main] (4) [right of=2] [color=cyan] {};
		\node[main] (5) [left of=2] [color=cyan] {};
		\node[main] (6) [below left of=5] [color=cyan] {};
		\node[main] (7) [above left of=5] [color=cyan] {};
		\path [color=cyan, thick] (1) edge (2)
		(1) edge (3)
		(1) edge (4)
		(1) edge (7)
		(2) edge (3)
		(2) edge (4)
		(2) edge (6)
		(2) edge (5)
		(5) edge (6)
		(5) edge (7)
		(6) edge (7);
	\end{tikzpicture}
	\caption{Example of separation.}
	\label{separation}
\end{figure}

\begin{defn}
	The \textbf{order} of the separation is $|A \cap B|$.
\end{defn}

\begin{defn}
	$S$ is \textbf{well-linked} to a model $M_{1}, M_{2}, \dots, M_{m}$ if every separation $(A,B)$ with $S \subseteq A_{i}$ $(\exists i) M_{i} \subseteq B \setminus A$ has order $\geq |S|$.
\end{defn}

\begin{thm}
	$\forall G_{i}, S = \{s_{1}, s_{2}, \dots, s_{k}, t_{1}, t_{2}, \dots, t_{k}\} \subseteq V(G)$ and $M_{1}, \dots, M_{4k}$ mdel of $K_{4m}$ in $G$. If $S$  is well-linked to $M_{1}, M_{2}, \dots, M_{4k}$ then $G$ contains distinct paths from $s_{i}$ to $t_{i}$ for all $i \in [k]$.
\end{thm}

We can see that this is somewhat refolmulation of theorem before (\ref{2-linked-thm}). Thus if we prove this we will also prove the previous theorem. But we will introduce another similiar theorem which will imply this theorem.

\begin{defn}
	$G, S\ subseteq V(G)$ and $M_{1}, M_{2}, \dots, M_{m} \subseteq V(G)$ pairwise distinct is an \textbf{$S$-relaxed model} of $K_{m}$ in $G$ if
	
	\begin{enumerate}
		\item $(\forall i) G[M_{i}]$ is connected \underline{or} every componnet of $G[M_{i}]$ intersects $S$.
		\item $(\forall i \neq j) \exists uv \in E(G)$ s.t. $u \in M_{i}, v \in M_{j}$ \underline{or} $M_{i} \cap S \neq \emptyset \neq M_{j} \cap S$.
	\end{enumerate}
\end{defn}

\begin{thm}[\textcolor{blue}{Slightly changed}]
	$\forall G_{i}, S = \{s_{1}, s_{2}, \dots, s_{k}, t_{1}, t_{2}, \dots, t_{k}\} \subseteq V(G)$ and $M_{1}, \dots, M_{4k}$ \textcolor{blue}{$S$-relaxed mdel} of $K_{4m}$ in $G$. If $S$  is well-linked to $M_{1}, M_{2}, \dots, M_{4k}$ then $G$ contains distinct paths from $s_{i}$ to $t_{i}$ for all $i \in [k]$.
\end{thm}

\begin{proof}
	We will prove this theorem by induction on $|V(G)|$. We will separate it to some distinct cases.
	
	\begin{enumerate}[(1)]
		\item Suppose there exisits a separation $(A,B)$ of order $2k$ (which is $= |S|$) s.t. $S \subsetneq A$ and $(\exists i) M_{i} \subseteq B \setminus A$. Then by Menger's theorem there exists $2k$ disjoint paths from $S$ to $A \cap B$ (since $S$ is well-linked to $M_{1}, \dots, M_{4k}$).\textit{ We want: $G[B]$ disjoint paths from $s_{i}'$ to $t_{i}'$ for all $i \in [k]$, where $s_{i}'$ and $t_{i}'$ are the ends from the paths labeled same as the beginings.} We apply induction hypothesis on $G[B]$ $S' = \{s_{i}', t_{i}' | \forall i \in [k]\}$ $M_{1} \cap B, M_{2} \cap B, \dots, M_{4k} \cap B$.
		
		First we need to prove that the properties still holds. Such as that it is still $S'$-relaxed and $S'$ is well-linked. Consider $M_{i} \cap S = \emptyset$ so $\forall j \neq i$ there is at least one vertex in $B$. Then $M_{1} \cap B$ components do not intersect $S'$ so it didn't intersect $S$. Therefore it had to be connected and thus still is. That is the first property and the second is \textit{left out as exercise}. So $S'$ is relaxed model.
		
		We now take a look at if $S'$ wouldn't be well-linked. Then there would be a separation with order $< 2k$. But this separation would be present even before so it cannot be there.
		
		Now WLOG: Every separation $(A,B)$ s.t. $S \subsetneq A, (\exists i) M_{i} \subseteq B \setminus A$ has order $> 2k$.
		
		\item Suppose $\exists v \in V(G) \setminus (S \cup \bigcup_{i}^{4k} M_{i})$ apply I.H. on $G -v$. We need to show that it is well-linked. Suppose we have a separation with order $<2k$. We put $v$ in the intersection of the separation ("cut") and get a separation of $G$ with order $\leq 2k$. That can't happen since we assumed the orrder is $>2k$.
		
		\item Suppose $(\exists i) \exists uv \in E(G[M_{i}])$ s.t. $v \notin S$. Aplly I.H> to $G / uv$ (contract the edge $uv$). We may see that $S$ is relaxed model and well-linked with the similiar arguments as in the point before.
	\end{enumerate}
	
	With this induction we end up with $S \subseteq V(G)$ and $M_{1}, M_{2}, \dots, M_{4k}$. We know $(\forall i) M_{i} \cap S = \emptyset \Rightarrow |M_{i}| =1$ \underline{or} $M_{i} \subseteq S$. Also all single $M_{i}$ forms a clique. We would like to find if there exist a mathcing between $S$ and $V(G) \setminus S$ covering $S$. For that we may recall Hall's theorem and thus we need $\forall X \subseteq S: |N(X)| \geq |X|$ where $N(X)$ are the neighbours of $X$. Lets take a look at one $M_{k} = X$ and its $N(X)$. There are not necessarily edges to $M_{1}, \dots, M_{g}$. Lets put $A = S \cup N(X)$ and $B = \text{ clique on } M_{i}\text{s} \cup (S \setminus X)$. By that we get that $X = A \setminus B$ and $V(G) \setminus (S \cup N(X)) = B \setminus A$. So $(S \setminus X) \cup N(X) = A \cap B$ which can't be smaller than $2k$. So $|S \setminus X| + |N(X)| \geq 2k$ where $S \setminus X| = |S| - |X|$ and $|S| = 2k$ this means that $|N(x)| \geq |X|$.
	
	Therefore we find the matching between $S$ and clique. Thus we take for each $i$ the edge in matching from $s_{i}$ to $s_{i}'$, then path from clique from $s_{i}'$ to $t_{i}'$ and next from matching $t_{i}'$ to $t_{i}$.
\end{proof}
	\include{kg-iii/tree-decomposition}
	\chapter{Polynomials for graph theory}

We will introduce polynomials representing the graphs. Then we will look at what properties these polynoms have. But firstly we will look at some basics. Let $p(x_{1}, \dots, x_{n})$ be a polynomial on $n$ variables. One term $ax_{i}^{l} x_{j}^k \dots$ is called a \textbf{monomial}.

\begin{defn}
	The \textbf{total degree} of a monomial $x_{1}^{d_{!}} x_{2}^{d_{2}} \cdots x_{n}^{d_{n}}$ is the sum $d_{1} + d_{2} + \dots + d_{n}$.
\end{defn}

\begin{defn}
	Total degree of polynomial is the maximal total degree of its monomials.
\end{defn}

Also we will denote $[x_{1}^{d_{1}} \dots x_{n}^{d_{n}}]p$ the coefficients of $x_{1}^{d_{1}} \dots x_{n}^{d_{n}}$.

\begin{thm}[Chevalley–Warning, \textit{no proof}]
	Let $p$ be a prime number and $f_{1}, \dots, f_{k}$ polynomials over $\Z_{p}$ in $n$ variables and $\sum_{i = 1}^{k} \text{total degree of } f_{i} < n$, then the number of $a_{1}, \dots, a_{n} \in \Z_{p}$ such that for all $i$ $f_{i}(a_{1}, \dots, a_{n}) = 0$ is divisible by $p$.
\end{thm}

Now lets see an example of this. Let $f_{1} = x^2 + y^2 + z^2 + u =0$ and $f_{2} = x - y + z - u = 0$ over $\Z_{3}$. The total degree of $f_{1} + f_{2} = 2 + 1 < 4$ which is the number of variables. So the solutions are $x = y = z = u =0$ where the number of them has to be divisible by $3$. Also we have a solution $x = 1, z = 1, y = 2, u =0$.

\begin{thm}[Combinatorial Nullstellensatz, \textit{no proof}]
	Let $f$ be a polynomial in $n$ variables $x_{1}, \dots, x_{n}$, $f \not\equiv 0$. Suppose $S_{1}, S_{2}, \dots, S_{n} \subseteq \R$ such that $(\forall i) |S_{i}| > \deg_{x_{i}}(f)$. Then $\exists a_{1} \in S_{1}, \dots, a_{n} \in S_{n}$ such that $f(a_{1}, \dots, a_{n}) \neq 0$.
\end{thm}

Where $\deg_{x_{i}}(f)$ is the largest degree of $x_{i}$ in $f$. Example of application would be to ask if graph $G$ has a $3$-regular graphs as a subgraph, e.g. if it is true that $\delta(G) \geq 10^{10} \Rightarrow G$ has a $3$-regular subgraph? Generally NO.

\begin{thm}
	Suppose $\delta(G) \geq 4, \Delta(G) \leq 5$ and $G$ is not 4-regular. Then $G$ has a $3$-regular subgraph.
\end{thm}

\begin{proof}
	We will consider the following polynomials over $\Z_{3}$, for $v \in V(G)$ we define $f_{v} = \sum_{e \ni v} x_{e}^{2}$. Now we will take a look at such system of equations. We have this many variables: $|E(G)| = \frac{\sum_{v} \deg(v)}{2} > 2|V(G)|$. (Remember $G$ is not 4-regular.) And $\sum_{v} \text{total degree}(f_v) = 2|V(G)| < |E(G)|$. This implies that we can use the first theorem. Therefore $|\{a_{e} \in \Z_{3} : e \in E(G)\}|$ such that $f_{v}(\overrightarrow{a}) =0$ for all $v \ in V(G)$ is divisible by 3. There exists at least one solution (since $x_{e} = 0$ for $e \in E(G)$) which means there exists another solution $\{a_{e} : e \in E(G)\}$ such that $\exists e : a_{e} \neq 0$.
	
	Now we create a subgraph $H$ as follows. $E(H) = \{e \in E(G) : a_{e} \neq 0\}$ and $V(H)$ are all vertices indices to $E(H)$. If we look at vertex $v$ then $f_{v}(\overrightarrow{a}) = \sum_{e \ni_{G} v} a_{e}^{2} = \sum_{e \ni_{H} v} a_{e}^{2}$ which is equivalent to $\deg_{H}(v) \mod 3$. Also $f_{v} = 0$. So every vertex has a degree 3.
\end{proof}

Now we will use polynomials for coloring, but not the usual one we may find. We will be talking about a coloring which is obtained by choosing one color from a list assigned to each vertex. Then it must have the same property as a normal coloring. This is called \textbf{list coloring}. When all vertices have the same size lists $k$, than it can be also called that it is \textbf{$k$-choosable}.

\begin{defn}
	List chromatic number $\chi_{l}(G) = \text{smallest } k : G$ can be colored from any assignment of list of size $\geq k$.
\end{defn}

We will be trying to obtain the result that every bipartite plane graph is $3$ list colorable. But before we do so we will build another theory. Also it is possible to create such bipartite graphs that the $\chi_{l}(G)$ is arbitrarily large. For a graph $G$ we define $\overrightarrow{G}$ as an arbitrary orientation of $G$. And also we will define a polynomial:

$$
P_{\overrightarrow{G}} (x_{1}, \dots, x_{n}) = \prod_{(v_{i},v_{j}) \in E(\overrightarrow{G})} x_{i} - x_{j}
$$

To get a better understanding lets see a simple example of a graph $G$ which can be seen on a picture \ref{pol-ex}. This the polynomial would be $P_{\overrightarrow{G}} = (x_{2} - x_{1}) (x_{3} - x_{2}) (x_{1} - x_{3}) = x_1 x_2 x_3 - x_2 x_3 - x_1 x_2^2 + x_2^2 x_3 - x_1^2 x_3 + x_1 x_3^2 + x_1^2 x_2 - x_1x_2x_3$.

\begin{figure}[!ht]\centering
	\begin{tikzpicture}[node distance={20mm}, thick, main/.style = {draw, circle}]
		\node[main] (2) {$v_{2}$};
		\node[main] (1) [below left of=2] {$v_{1}$};
		\node[main] (3) [below right of=2] {$v_{3}$};
		\path[->] (1) edge (2)
				  (2) edge (3)
				  (3) edge (1);
	\end{tikzpicture}
	\caption{Example for a polynomial for a graph $\overrightarrow{G}$.}
	\label{pol-ex}
\end{figure}

Now if we take $V(G) = \{v_{1}, \dots, v_{n}\}$ which map each one of them $v_{i} \to x_{i}$. Then for a assignment $\overrightarrow{c}$ as for each $i$ $v_i \to c_i$ we know that $G$ has a proper coloring iff $P_{\overrightarrow{G}}(\overrightarrow{c}) \neq 0$. That is if $S_{1}, \dots, S_{n}$ are lists of allowed colors for $v_{1}, \dots, v_{n}$ then $G$ is colorable from $S_{1}, \dots, S_{n}$ iff $(\exists c_{1} \in S_{1}, \dots, c_{n} \in S_{n}) P_{\overrightarrow{G}}(c_{1}, \dots, c_{n}) \neq 0$. To prove this we could use some theorem, but we will need a stronger version.

\begin{thm}[Combinatorial Nullstellensatz 2nd version]
	Let $f$ be a polynomial in $x_{1}, \dots, x_{n}$ and $S_{1}, \dots, S_{n} \subseteq \R$. If $(\exists d_{1}, \dots, d_{n})$ total degree $(f) \leq d_{1} + \dots + d_{n}$ and $[x_{1}^{d_{1}}\dots x_{n}^{d_{n}}]f \neq 0$ and $(\forall i) |S_{i}| > d_{i}$. Then $\exists c_{1} \in S_{1}, \dots, c_{n} \in S_{n} : f(c_{1}, \dots, c_{n}) \neq 0$.
\end{thm}

But why does it hold? We can assume $|S_{i}| = d_{i} + 1$. For the values $S_{i} = \{a_{1}, \dots, a_{d_{i}+1}\}$ (as colors). We know that if $x_{i} \in S$ then $(x_{i} - a_{1}) (x_{i} - a_{2}) \cdots (x_{i} - a_{d_{i}+1}) = 0$. Then $x_{i}^{d+1} - (a_{1} + \dots + a_{d+1}) x_{i}^{d_{i}} + \dots$ where we denote $b_{d_{i}} = (a_{1} + \dots + a_{d+1})$ so that leads to:

$$
x_{i}^{d_{i}+1} = b_{d_{i}}x_{i}^{d_{i}} + \dots + b_{1}x_{1} + b_{0}
$$

Now we may take a polynomial $f$ from the 2nd theorem. The polynomial $f'$ has the same values on $S_{1}, \dots, S_{n}$ but $\deg_{v_{i}}(f') \leq d_{i}$. So now the condition $(\forall i) |S_{i}| > \deg_{v_{i}} (f')$ holds. Only think is to see that $f' \not\equiv 0$.

We have already shown this theorem.

\begin{thm}
	If $[x_{1}^{d_1} \dots x_{n}^{d_n}]P \neq 0$, $d_1 + \dots + d_n = $ total degree of $P$, $S_1, \dots, S_{n}, (\forall i) |S_i| > d_i$ then $\Rightarrow (\exists c_{1} \in S_{1}, \dots, c_{n} \in S_{n}) P(c_{1}, \dots c_n) \neq 0$.
\end{thm}

\begin{observ}
	$P_{\overrightarrow{G}} (c_1, \dots, c_n) \neq 0 \Leftrightarrow c_1, \dots, c_n$ is a proper coloring of $G$.
\end{observ}

By combining these two results we may get a following corollary.

\begin{cor}
	If $[x_{1}^{d_1} \dots x_{n}^{d_n}]P_{\overrightarrow{G}} \neq 0$ and $L$ is a list assignment such that $(\forall i) |L(v_i)| > d$ then $G$ is $L$-colorable.
\end{cor}

Now we will see other variants of this polynomial and how it can be computed in other ways. Here we see an example on picture \ref{orientation}. First we may see the original graph and its orientation $\overrightarrow{G}$ on picture \ref{orientation-original}. Then one monomial in a corresponding polynomial is a choice of one $\pm x_{i}$ in the bracket. This choice induces a new orientation $\overrightsimarrow{G}$ which both can be seen on picture \ref{orientation-choice}. Thus the polynomial can be computed as follows:

$$
\sum_{\overrightsimarrow{G} \text{ orientation of }G} (-1)^{\text{\# of different direction in } \overrightsimarrow{G} \text{ from } \overrightarrow{G}} \prod_{i} x_{i}^{\deg^{-}_{\overrightsimarrow{G}}(v_{i})}
$$

\begin{figure}[!ht]\centering
	\begin{subfigure}{0.45\textwidth}\centering
		\begin{tikzpicture}[node distance={35mm}, thick, main/.style = {draw, circle}]
			\node[main] (1) {$v_{1}$};
			\node[main] (2) [above right of=1] {$v_{2}$};
			\node[main] (3) [below right of=1] {$v_{3}$};
			\node[main] (4) [above right of=3] {$v_{4}$};
			\draw[->] (1) -- (2) node[midway, above left] {$(x_{2} - x_{1})$};
			\draw[->] (3) -- (1) node[midway, below left] {$(x_{1} - x_{3})$};
			\draw[->] (2) -- (3) node[midway, left] {$(x_{3} - x_{2})$};
			\draw[->] (2) -- (4) node[midway, above right] {$(x_{4} - x_{2})$};
			\draw[->] (4) -- (3) node[midway, below right] {$(x_{3} - x_{4})$};
		\end{tikzpicture}
		\caption{The original graph $\overrightarrow{G}$.}
		\label{orientation-original}
	\end{subfigure}
	\begin{subfigure}{0.45\textwidth}\centering
		\begin{tikzpicture}[node distance={35mm}, thick, main/.style = {draw, circle}]
			\node[main] (1) {$v_{1}$};
			\node[main] (2) [above right of=1] {$v_{2}$};
			\node[main] (3) [below right of=1] {$v_{3}$};
			\node[main] (4) [above right of=3] {$v_{4}$};
			\draw[->] (1) -- (2) node[midway, above left] {$(x_{2} -$\textcolor{red}{$x_{1}$}$)$};
			\draw[->] (3) -- (1) node[midway, below left] {$(x_{1} -$\textcolor{red}{$x_{3}$}$)$};
			\draw[->] (2) -- (3) node[midway, left] {$($\textcolor{red}{$x_{3}$}$ - x_{2})$};
			\draw[->] (2) -- (4) node[midway, above right] {$($\textcolor{red}{$x_{4}$}$ - x_{2})$};
			\draw[->] (4) -- (3) node[midway, below right] {$($\textcolor{red}{$x_{3}$}$ - x_{4})$};
			\draw[->, bend left=20, color=cyan] (2) edge (1);
			\draw[->, bend right=20, color=cyan] (2) edge (4);
			\draw[->, bend right=20, color=cyan] (4) edge (3);
			\draw[->, bend left=20, color=cyan] (2) edge (3);
			\draw[->, bend left=20, color=cyan] (1) edge (3);
		\end{tikzpicture}
		\caption{The \textcolor{red}{choice} and \textcolor{cyan}{new orientation $\overrightsimarrow{G}$}.}
		\label{orientation-choice}
	\end{subfigure}
	\caption{Example for computing polynomial by orientation, the original graph $\overrightarrow{G}$.}
	\label{orientation}
\end{figure}

Thus altogether we get this result:

$$
[x_{1}^{d_{1}} x_{2}^{d_{2}} \dots x_{n}^{d_{n}}]P_{\overrightarrow{G}} = \sum_{\overrightsimarrow{G} \text{ orientation of }G : \deg^{-}_{\overrightsimarrow{G}}(v_{i}) = d_{i} \forall i} (-1)^{\text{\# of different direction in } \overrightsimarrow{G} \text{ from } \overrightarrow{G}}
$$

When we show another example where we have a vertex with in degree 3. See picture \ref{eulerian-orientation}. For the indegree to be same it must be that for all changed edge to the vertex there must be some changed edge out of the vertex.

\begin{figure}[!ht]\centering
	\begin{tikzpicture}[node distance={15mm}, thick, main/.style = {draw, circle}]
		\node (R1) {};
		\node [below of=R1] (R2) {};
		\node [below of=R2] (R3) {};
		\node [below of=R3] (R4) {};
		\node[main, fill] [below left of=R2] [above left of=R3] (1) {};
		\node [above left of=1] (L1) {};
		\node [left of=1] (L2) {};
		\node [below left of=1] (L3) {};
		\draw[->] (L1) edge (1);
		\draw[->] (L2) edge (1);
		\draw[->] (L3) edge (1);
		\draw[->] (1) edge (R1);
		\draw[->] (1) edge (R2);
		\draw[->] (1) edge (R3);
		\draw[->] (1) edge (R4);
		\draw[->, bend left=15, color=cyan] (L1) edge (1);
		\draw[->, bend right=15, color=cyan] (1) edge (L2);
		\draw[->, bend right=15, color=cyan] (1) edge (L3);
		\draw[->, bend left=15, color=cyan] (R1) edge (1);
		\draw[->, bend left=15, color=cyan] (R2) edge (1);
		\draw[->, bend right=15, color=cyan] (1) edge (R3);
		\draw[->, bend right=15, color=cyan] (1) edge (R4);
		\draw[->, bend left=10, color=red] (1) edge (R1);
		\draw[->, bend left=10, color=red] (1) edge (R2);
		\draw[->, bend right=10, color=red] (L3) edge (1);
		\draw[->, bend right=10, color=red] (L2) edge (1);
	\end{tikzpicture}
	\caption{Example for Eulerian subgraph orientations.}
	\label{eulerian-orientation}
\end{figure}


\begin{observ}
	If $\overrightarrow{G}$ and $\overrightsimarrow{G}$ are orientations with the same indegrees, the $\overrightarrow{H}$ subgraph of $\overrightarrow{G}$ with edges oriented differently, then $\forall v \in V$ $\deg_{\overrightarrow{H}}^{+} (v) = \deg_{\overrightarrow{H}}^{-} (v)$ or in other words: $\overrightarrow{H}$ is Eulerian subgraph of $\overrightarrow{G}$.
\end{observ}

\begin{lemma}
	Suppose $\overrightarrow{G}$ is an orientation of $G$ such that $\forall i$ $d_{i} = \deg^{-}(v_{i})$ in $\overrightarrow{G}$. Then
	
	$$
	[x_{1}^{d_{1}} x_{2}^{d_{2}} \dots x_{n}^{d_{n}}] P_{\overrightarrow{G}} = \sum_{\overrightarrow{H} \text{ Eulerian subgraph of } \overrightarrow{G}} (-1)^{\left| E \left( \overrightarrow{H} \right) \right|}
	$$
	
	which is also equivalent to the \textbf{number of Eulerian paths of $\overrightarrow{G}$ with even number of edges} minus \textbf{number of Eulerian paths of $\overrightarrow{G}$ with odd number of edges}.
\end{lemma}

\begin{cor}
	Suppose $\overrightarrow{G}$ is an orientation of $G$ where $(\forall i) \deg^{-}(v_{i}) = d_{i}$. Let $L$ be a list assignment for $G$ s.t. $(\forall i) |L(v_{i})| > d_{i}$. If $\overrightarrow{G}$ has different number of Eulerian subgraphs with even and odd number of edges then it is $L$-colorable.
\end{cor}

Lets see this result on an example on pictures \ref{euler-list}. Where the sizes of lists are written in the nodes. We may see there are visualized two Eulerian paths with \textcolor{blue}{3} edges and \textcolor{orange}{4} edges. Also there is an empty Eulerian subgraph. Altogether we have two even subgraphs and one odd, so that means there exists $L$-coloring. But this theory only show the existence, not the construction.

\begin{figure}[!ht]\centering
	\begin{subfigure}{0.30\textwidth}\centering
		\begin{tikzpicture}[node distance={25mm}, thick, main/.style = {draw, circle}]
			\node[main] (1) {2};
			\node[main] (2) [above right of=1] {3};
			\node[main] (3) [below right of=1] {2};
			\node[main] (4) [above right of=3] {2};
			\draw (1) -- (2);
			\draw (3) -- (1);
			\draw (2) -- (3);
			\draw (2) -- (4);
			\draw (4) -- (3);
		\end{tikzpicture}
		\caption{The original graph $G$.}
	\end{subfigure}
	\begin{subfigure}{0.30\textwidth}\centering
		\begin{tikzpicture}[node distance={25mm}, thick, main/.style = {draw, circle}]
			\node[main] (1) {2};
			\node[main] (2) [above right of=1] {3};
			\node[main] (3) [below right of=1] {2};
			\node[main] (4) [above right of=3] {2};
			\draw[->, color=cyan] (1) -- (2);
			\draw[->, color=cyan] (3) -- (1);
			\draw[->, color=cyan] (3) -- (2);
			\draw[->, color=cyan] (2) -- (4);
			\draw[->, color=cyan] (4) -- (3);
		\end{tikzpicture}
		\caption{The orientation $\overrightarrow{G}$.}
	\end{subfigure}
	\begin{subfigure}{0.30\textwidth}\centering
		\begin{tikzpicture}[node distance={25mm}, thick, main/.style = {draw, circle}]
			\node[main] (1) {2};
			\node[main] (2) [above right of=1] {3};
			\node[main] (3) [below right of=1] {2};
			\node[main] (4) [above right of=3] {2};
			\draw[->, color=cyan] (1) -- (2);
			\draw[->, color=cyan] (3) -- (1);
			\draw[->, color=cyan] (3) -- (2);
			\draw[->, color=cyan] (2) -- (4);
			\draw[->, color=cyan] (4) -- (3);
			% First eulerian path.
			\draw[->, bend left=20, color=orange] (1) edge (2);
			\draw[->, bend left=20, color=orange] (3) edge (1);
			\draw[->, bend left=20, color=orange] (2) edge (4);
			\draw[->, bend left=20, color=orange] (4) edge (3);
			% Eulerian path second.
			\draw[->, bend left=20, color=blue] (3) edge (2);
			\draw[->, bend right=20, color=blue] (2) edge (4);
			\draw[->, bend right=20, color=blue] (4) edge (3);
		\end{tikzpicture}
		\caption{Euler subgraphs.}
	\end{subfigure}
	\caption{Usage of shown corollary for a list assignment.}
	\label{euler-list}
\end{figure}

Now lets take $G$ a bipartite graph. We may see an observation that every Eulerian subgraph is edge-disjoint union of cycles. Therefore if $G$ is bipertite, then it has even number of edges in all Eulerian subgraphs.

\begin{cor}
	If $G$ is bipartite, $\overrightarrow{G}$ has indegrees $d_{1}, \dots, d_{n}$ then $G$ is $L$-colorable for every list assignments $L$ s.t. $(\forall i) |L(v_{i})| > d_{i}$.
\end{cor}

\begin{observ}
	Every planar bipartite graph has an orientation with maximal indegree $\leq 2$ therefore it has list chromatic number at most 3.
\end{observ}

But it does not apply for all bipartite graphs since it is possible to create such bipartite graph such that $\chi_{l}$ is arbitrarily large.
	\chapter{VC - dimension}

VC stands for the names Varnik--Cherrorenkis. We will be considering a set systems $\mathcal{F}$. And then for a set $X$ we denote $X \cap \mathcal{F}$ as $\{X \cap A : A \in \mathcal{F}\}$.

\section{Set systems}

\begin{defn}
	$\mathcal{F}$ \textbf{breaks} $X$ if $X \cap \mathcal{F} = 2^{X}$.
\end{defn}

Lets see for ourselves an example, which is on picture \ref{broken-x}. For $\mathcal{F}$ we have all half planes in $\R^2$. Then the set $X = \{a,b,c\}$. Then if we draw all lines and choose both planes it will generate $X \cap \mathcal{F} = \{ \emptyset, \{a\}, \{b\}, \{c\}, \{a,b\}, \{b,c\}, \{a,c\}, \{a,b,c\}\}$ which are all subsets of $X$. Hence it breaks the $X$.

\begin{figure}[!ht]\centering
	\begin{tikzpicture}[node distance={15mm}, thick, main/.style = {draw, circle}]
		\node (p1) {};
		\node[main] (b) [above right of=p1] {b};
		\node (p2) [below right of=b] {};
		\node[main] (a) [below of=p1] {a};
		\node (p3) [right of=a] {};
		\node[main] (c) [right of=p3] {c};
		\node (1) [left of=p1] {};
		\node (2) [right of=p2] {};
		\node (3) [above right of=p2] {};
		\node (4) [below left of=p3] {};
		\node (5) [above left of=p1] {};
		\node (6) [below right of=p3] {};
		\node (7) [left of=4] {};
		\node (8) [right of=6] {};
		\draw[color=black] (7) -- (8);
		\draw[color=blue] (1) -- (2);
		\draw[color=orange] (5) -- (6);
		\draw[color=cyan] (3) -- (4);
	\end{tikzpicture}
	\caption{Example of $\mathcal{F}$ and set $X$ which is broken.}
	\label{broken-x}
\end{figure}

\begin{defn}
	We say that $VC-\dim(\mathcal{F})$ is $\max \{|X| : \mathcal{F} \text{ breaks } X\}$.
\end{defn}

In our previous example we see that $VC-\dim(\mathcal{F})$ is at least 3. But we may see that it is exactly 3. Because if we take 4 points they are either all in convex hull thus it is impossible to take the opposite points or one point is in the convex hull of other three of them, for which it is impossible to take only the points forming the convex hull except the middle one. Other example can be $\mathcal{F}$ as all intervals in $\R$ for which the $VC-\dim(\mathcal{F})$ is 2.

All these examples are geometrical ones. Lets see some for graph theory. Lets have $G$ graph s.t. $K_{n} \nleq_{t} G$. Then $\mathcal{F} = \{N_{G}[v] : v \in V(G)\}$. Where the \textbf{closed neighborhood} is defined as $N_{G}[v] = \{v\} \cup \{u \mid \{u,v\} \in E(G)\}$. We may see that $VC-\dim(\mathcal{F}) \leq n-1$. For contradiction $X \subseteq V$ and $|X| = n$. For every 2-element subset there is a vertex adjacent to them, this would give me a $K_{n}$ as a topological subgraph.

\section{Subsystems}

Now we will be considering sub-systems. That is $\mathcal{F} \subseteq 2^{Y}$ for some $Y$. There will be defined a measurement $\mu$ on $Y$. As an example can be the half-planes but only inside the square of size 1.

\begin{defn}
	$N$ is an $\epsilon$-net if
	
	$$
	(\forall A \in \mathcal{F}) : \mu(A) \geq \epsilon \mu(Y) \land N \cap A \neq \emptyset
	$$
\end{defn}

For an example consider $\mathcal{F} = \{\text{axis aligned rectangles in }Y\}$ and $\mu$ be the area of the rectangle. Then the $\epsilon$-net must hit all rectangles. We can easily create a grid of points which are $\epsilon$ apart. This will create $\frac{1}{\epsilon^2}$ points in net.

\begin{thm}
	There $\exists c$ s.t. if $\mathcal{F} \subseteq 2^Y$ has $VC-\dim(\mathcal{F})$ $k$ then
	
	$$
	(\forall \epsilon > 0) c \frac{k}{\epsilon} \log \left( \frac{k}{\epsilon} \right)
	$$
	
	independently at random chosen elements of $Y$ form an $\epsilon$-net with probability $\geq 1/2$. This is all with the probability for $A \subseteq Y$ $\Pr [p \in A] = \frac{\mu(A)}{\mu(Y)}$.
\end{thm}

\begin{defn}
	$\tau (\mathcal{F}) = \min \left( |Z| : (\forall A \in \mathcal{F}) Z \cap A \neq \emptyset \right)$.
\end{defn}

As an example we may see that for $\mathcal{F}$ closed neighborhood in $G$ the $\tau(\mathcal{F})$ is for the minimal size of dominating set in the given graph $G$.
\end{document}

