\chapter{Extremální kombinatorika}

\begin{definice}
	Pro $n \in \mathbb{N}$ a graf $F$ definujeme $\text{ex}(n,F) :=$ největší počet hran v grafu na $n$ vrcholech, který neobsahuje $F$ jako podgraf. Nebo-li:
	
	$$
	\text{ex}(n,F) = \max\{|E|; G = (V,E): |V| = n, F \subseteq G\}
	$$
\end{definice}

\begin{definice}
	\textbf{Turanův graf $T(n,r)$} je úplný $r$-partitní graf na $n$ vrcholech, jehož všechny partity mají velikost $\left\lfloor \frac{n}{r} \right\rfloor$ anebo $\left\lceil \frac{n}{r} \right\rceil$. Potom $t(n,r):=$ počet hran $T(n,r)$.
\end{definice}

\begin{veta}[Turán]
	$\forall n,r \in \mathbb{N}: \text{ex}(n,K_{r+1}) = t(n,r)$
\end{veta}

\begin{proof}
	Pozorování: $T(n,r)$ neobsahuje $K_{r+1}$, tedy $\text{ex}(n,K_{r+1}) \geq t(n,r)$. Stačí dokázat: $\text{ex}(n,K_{r+1})\leq t(n,r)$. Nechť $G = (V,E)$ je graf na $n$ vrcholech, $K_{r+1} \nsubseteq G$ a $|E| = \text{ex}(n,K_{r+1})$. Tvrzení 1: Každé 2 nesousedící vrcholy $x,y$ mají v $G$ stejný stupeň. Sporem kdyby $\deg(x) > \deg(y)$ tak $y$ odstraním sousedy a přidám mu sousedy $x$. Ten má ale více hran a protože $\{x,y\} \notin E$ a s $x$ nebyla klika, tak teď také žádná klika nevznikla s $y$. "Nebo-li $y$ nahradím kopií $x$." Tvrzení 2: Definujeme relaci $R := \{(x,y) \in V \times V: \{x,y\} \notin E\}$. Potom $R$ je ekvivalence. Jistě je $R$ reflexivní, také symetrické. Pro spor předpokládejme, že $R$ není tranzitivní: $\exists x,y,z: (x,y) \in R, (y,z) \in R \land (x,z) \notin R$. Dle Tvrzení 1: $\deg_{G}(x) = \deg_{G}(y) = \deg_{G}(z)$. Potom "nahradím $x$ a $z$ kopiemi $y$". A platí $|E(G')| > |E|$. A $G'$ neobsahuje $K_{r+1}$ obdobným argumentem jako u Tvrzení 1. Nyní nechť $P_{1}, P_{2}, \dots, P_{k}$ jsou třídy ekvivalence $R$. Tvrzení 3: $k=r$ (pokud $n \geq r$). $k > r:$ tak $K_{r+1} \subseteq G$ a to je spor. $k < r:$ tak lze partitu s $\geq 2$ vrcholy rozdělit na dvě menší partity a přidáme hrany mezi nimi a dostaneme $G'$, který $K_{r+1} \nsubseteq G'$ \& $|E(G')| > |E|$ opět spor. Tvrzení 4: BÚNO: $|P_{1}| \leq |P_{2}| \leq \dots \leq |P_{r}|$. Tvrdíme, že $|P_{1}| \leq |P_{r}| + 1$. Kdyby nějaké dvě partity byly odlišné $\geq 2$. Potom vezmeme půlku přebytečných vrcholů a přehodíme je do předchozí partity. Následně spojíme hranami. Dostanu $G'$ kde $K_{r+1} \nsubseteq G$ a $|E(G')| > |E|$. \textit{Poznámka: $(l+2) l < (l+1)(l+1)$.} Shrnutí: $G$ je úplný $r$-partitní graf, kde všechny partity jsou skoro stejné $\Rightarrow G \cong T(n,r)$.
\end{proof}

\begin{definice}
	\textbf{Hypergraf} je dvojice $(V,E)$, kde prvky $E$ \textit{("hyperhrany")} jsou podmnožiny $V$.
\end{definice}

\begin{definice}
	Hypergraf je \textbf{$k$-uniformní}, pokud všechny jeho hyperhrany mají $k$ vrcholů.
\end{definice}

\begin{definice}
	$f(n,k):=$ max počet hyperhran v $k$-uniformním hypergrafu na $n$ vrcholech, v němž žádné dvě hyperhrany nejsou disjunktní.
\end{definice}

\begin{pozor}
	Pro $n < k: f(n,k) = 0$.
\end{pozor}

\begin{pozor}
	Pro $k \leq n < 2k: f(n,k) = \binom{n}{k}$.
\end{pozor}

\begin{pozor}
	Pro $n \geq 2k: f(n,k) \geq \binom{n-1}{k-1}$. (Vybereme předem jeden vrchol.)
\end{pozor}

\begin{definice}
	Označme $V= \{1,2,3, \dots ,n\}$, na $V$ uvažujme sčítání modulo $n$. \textbf{Interval} je podmnožina $V$ tvaru $\{i,i+1,i+2, \dots, i+k\}$.
\end{definice}

\begin{pozor}
	Pro $n \geq 2k$ máme na $V$ přesně $n$ intervalů.
\end{pozor}

\begin{lemma}
	Nechť $V = \{1,2,3,\dots,n\}, n \geq 2k$ a $G = (V,E)$ je $k$-uniformní hypergraf jehož každá hyperhrana je interval a každé dvě hyperhrany se protínají. Potom $|E| \leq k$.
\end{lemma}

\begin{proof}
	BÚNO: $I = \{1,2,3,\dots,k\} \in E$. Označme $I_{j}^{-}:= \{j,j-1,j-2, \dots, j-k+1\}$ a $I_{j}^{+}:= \{j+1,j+2, \dots, j+k\}$. $I$ je protnutí $I_{1}^{-}, I_{2}^{-}, \dots, I_{k-1}^{-}$ \& $I_{1}^{+}, I_{2}^{+}, \dots, I_{k-1}^{+}$. Navíc z každé dvojice $I_{j}^{-},I_{j}^{+}$ nejvýše jeden patří do $E$, protože $I_{j}^{-} \cap I_{j}^{+} = \emptyset$. Tudíž $|E| \leq k$.
\end{proof}

\begin{veta}[Erdös-Ko-Rado]
	Pro libovolné $k \in \mathbb{N}$ a $n \geq 2k$ platí $f(n,k) = \binom{n-1}{k-1}$.
\end{veta}

\begin{proof}
	\textbf{Myšlenka:} $G = (V,E)$ je $k$-uniformní hypergraf na $n$ vrcholech, každé dvě hyperhrany se protínají $\to |E| \leq \binom{n-1}{k-1}$. Ekvivalentně: \textbf{(1)} -- $\frac{|E|}{\binom{n}{k}} \leq \frac{\binom{n-1}{k-1}}{\binom{n}{k}} = \frac{k}{n}$. Lemma: Když každá hyperhrana je interval $\frac{|E|}{n} \leq \frac{k}{n}$ -- \textbf{(2)}. Tyto dva zlomky jsou vlastně pravděpodobnosti. Takže náhodně očíslujeme vrcholy a mám stejnou pravděpodobnost v obou případech. \textbf{Důkaz:} Mějme $n \geq 2k$. Nechť $G= (V,E)$ je $k$-uniformní hypergraf v němž aždé 2 hyperhrany se protínají a $|E|$ je co největší. Chceme dokázat $|E| \leq \binom{n-1}{k-1}$. Nechť $X$ je počet dvojic $(e,\pi)$ t.ž. $e \in E$ a $\pi: V \to \{1,2,\dots,n\}$ taková, že $\pi$ zobrazí $e$ na intervalu. Potom pomocí počítání dvěma způsoby:
	
	\begin{enumerate}
		\item $X \leq n! \cdot k$ (dle lemma)
		\item $X = |E| \cdot n \cdot k! \cdot (n-k)!$
	\end{enumerate}
	
	$|E| \cdot n \cdot k! \cdot (n-k)! \leq n! \cdot k$ a $|E| \leq \frac{k}{n} \binom{n}{k} = \binom{n-1}{k-1}$.
\end{proof}

\begin{definice}
	Slunečnice (nebo $\Delta$-systém) se středem $S$ a $l$ lístky je $l$-tice množin $L_{1},\dots,L_{l}$ taková, že $\forall i \neq j: L_{i} \cap L_{j} = S$.
\end{definice}

\begin{definice}
	$s(k,l) := \sup \{|E|; G = (V,E)$ je $k$-uniformní hypergraf neobsahující žádnou slunečnici s  $l$ $\}$.
\end{definice}

\begin{veta}["lemma o slunečnici", Erdös-Rado]
	$\forall k,l \in \mathbb{R}: s(k,l) < + \infty$
\end{veta}

\begin{proof}
	Indukcí dle $k$. $k=1: s(k,l) = l -1$. $k > 1:$ Nechť $G= (V,E)$ je $k$-uniformní hypergraf neobsahující slunečnici s $l$ lístky. Nechť $D \subseteq E$ je co největší množina po dvou disjunktních hyperhran v $G$. Jistě $|D| \leq l - 1$, jinak máme slunečnici s $|D| \geq l$ lístky. Označme $W := \bigcup_{d \in D} d \subseteq V, |W| = k \cdot |D| \leq k \cdot (l - 1)$. Jistě každá $e \in E$ obsahuje aspoň jeden vrchol $W$. Tedy existuje $x \in W$, který je obsažen v aspoň $\frac{|E|}{|W|} = \frac{|E|}{k \cdot(l-1)}$ hyperhranách z $E$. Označme $E_{x} := \{ e \in E, x \in e\}$ pak $E_{x}^{-} := \{ e \setminus \{x\}, e \in E_{x}\}$ a $G_{x}^{-} := (V,E_{x}^{-})$. $G_{x}^{-}$ je $(k-1)$-uniformní hypergraf, který neobsahuje slunečnici s $l$ lístky: kdyby $e_{1},e_{2}, \dots ,e_{l}$ byla slunečnice v $G_{x}^{-}$, tak $e_{1} \cup \{x\}, e_{2} \cup \{x\}, \dots, e_{l} \cup \{x\}$ je slunečnice v $G$. Tedy dle IP: $|E_{x}^{-}| = s(k-1,l) < + \infty$. Navíc $|E_{x}^{-}| = |E_{x}| \geq \frac{|E|}{k \cdot (l-1)}$, tedy $|E| \leq k \cdot (l-1) \cdot s(k-1,l)$. Tedy $s(k,l) \leq k \cdot (l-1) \cdot s(k-1,l)$.
\end{proof}

\begin{pozn}
	Důkaz nám dává odhad $s(k,l) \leq k!(l-1)^{k}$.
\end{pozn}

\textit{Hypotéza}: $(\forall l)(\exists c_{l}): s(k,l) \leq c_{l}^{k}$.

\begin{definice}
	\textbf{Hamiltonovská kružnice} v grafu $G = (V,E)$ je kružnice v $G$ obsahující všechny vrcholy $G$.
\end{definice}

\begin{definice}
	Pro $n \geq 3$ označme $h(n) := \max \{d \in \mathbb{N}_{0}, \exists$ graf na $n$ vrcholech s min stupněm $\geq d$, který neobsahuje hamiltonovskou kružnici.$\}$.
\end{definice}

\begin{veta}[Bondy-Chvátal]
	Nechť $G = (V,E)$ je graf s $n \geq 3$ vrcholy, nechť $x,y \in V$ jsou nesousedící vrcholy $G$ takové, že $\deg_{G}(x) + \deg_{G}(y) \geq n$. Nechť $G^{+}:= (V,E\cup\{xy\})$. Potom $G$ je hamiltonovský iff $G^{+}$ je hamiltonovský.
\end{veta}

\begin{proof}
	"$\Rightarrow$" je triviální. "$\Leftarrow$" Označme $e_{0} = \{xy\}$. Nechť $G^{+}$ obsahuje hamiltonovskou kružnici $C$. Pokud $e_{0} \notin C$, tak $C$ je hamiltonovská kružnice v $G$. Předpoklad $e_{0} \in C$ jinak triviálně. Očíslujeme vrcholy a hrany $C$ takto: $x = x_{1}, x_{2}, \dots, x_{n-1}, x_{n} = y$ a $e_{0}, e_{1}, e_{2}, \dots , e_{n}, e_{0}$. Cíl je najít $i \in \{1,2,3, \dots, n-1\}$ tak, že $x$ sousedí s $x_{i+1}$ a $y$ sousedí s $x_{i}$ v grafu $G$. Označme $S_{x} := \{i \in \{1,2,3,\dots,n-1\}, \{xx_{i+1}\} \in E\}$ z toho plyne, že $|S_{x}| = \deg_{G}(x)$ a taky $S_{y} := \{i \in \{1,2,3,\dots,n-1\}, \{yx_{i}\} \in E\}$ pak $|S_{y}| = \deg_{G}(y)$. Tedy $|S_{x}| + |S_{y}| \geq n, |S_{x} \cup S_{y}| \leq |\{1,2,3, \dots, n-1\}| \leq n-1$, tudíž $\exists i \in S_{x} \cap S_{y}$. $(C \setminus \{e_{0},e_{i}\}) \cup \{\{xx_{i+1}\},\{yx_{i}\}\}$ je hamiltonovská kružnice v $G$.
\end{proof}

\begin{dusl}[Dirac]
	Každý graf na $n \geq 3$ vrcholech s min stupněm $\geq \frac{n}{2}$ je hamiltonovský. (Nebo $h(n) < \frac{n}{2}$.)
\end{dusl}

\begin{dusl}
	$\forall x \neq y \in V: \deg_{G}(x) + \deg_{G}(y) \geq n$. Pokud $G$ je úplný, tak hotovo. Jinak můžeme postupně přidávat hrany a vytvořit úplný graf. Pak pomocí Bondy-Chvátalovy věty jsou všechny tyto grafy v posloupnosti hamiltonovské.
\end{dusl}

\begin{definice}
	\textbf{Multigraf} je jako graf, ale můžu mít více hran mezi stejnou dvojicí vrcholů a můžu mít i smyčky. \textit{Formálně:} Multigraf je dvojice množin $(V,E)$ spolu s incidenční funkcí $f: E \to \binom{V}{2} \cup \binom{V}{1}$, kde $V$ jsou vrcholy a $E$ hrany.
\end{definice}

\begin{definice}
	\textbf{Incidenční matice} multigrafu $G = (V,E)$ je matice $I_{G} \in \{0,1,2\}^{|V| \times |E|}$, kde v řádku odpoví-\newline dajícímu vrcholu $x \in V$ a sloupci odpovídající hraně $e \in E$ je hodnota 2, pokud $e$ je smyčka u $x$, 1 pokud $x$ je jedna ze dvou konců $e$, 0 jinak.
\end{definice}

\begin{definice}
	Mějme multigraf $G = (V,E)$ s maticí incidence $I_{G}$.
	
	\begin{enumerate}
		\item Označme: $k(G) = k(V,E)$ počet komponent souvislosti $G$.
		\item Označme: $r(G) = r(VE)$ hodnost $I_{G}$. (nad $\mathbb{Z}_{2}$)
		\item Označme: $n(G) = n(V,E)$ dimenze jádra $\text{Ker}(I_{G})$ matice $I_{G}$, kde $\text{Ker}(I_{G}) = \{x \in (\mathbb{Z}_{2})^{|E|}: I_{g}x = 0\}$. Také se $n(G)$ nazývá nulita $G$.
	\end{enumerate}
\end{definice}

\begin{pozor}
	$r(V,E) = |V| - k(V,E)$
\end{pozor}

\begin{pozor}
	$n(V,E) = |E| - r(V,E)$
\end{pozor}

\begin{definice}
	$\text{Ker}(I_{G})$ \textbf{prostor cyklů} $G= (V,E)$.
\end{definice}

\begin{definice}
	$G = (V,E)$ multigraf $e \in E$. Pak:
	
	\begin{itemize}
		\item $G-e := (V, E \setminus \{e\})$
		\item $G / e$ (kontrakce hrany $y$) $:= G - e$, pokud $e$ je smyčka, jinak nový vrchol $v_{e}$ všechny hrany se projeví na novém vrcholu (protože máme multigraf).
	\end{itemize}
\end{definice}

\begin{pozor}
	$G-e$ i $G/e$ má vždy o jednu hranu méně než $G$.
\end{pozor}


$r(G) = |V| - k(G) = |F|$, kde $F \subseteq E$ je největší podmnožina $E$ neobsahující kružnici.

$n(G) = |E| - r(G) = |F|$, kde $F \subseteq E$ je největší podmnožina $E$ taková, že $k(G-F) = k(G)$

$$
r(G-e) = 
\left\{
\begin{array}{ll}
	r(G)-1 & e \text{ je most v } G \\
	r(G) & \text{jinak}
\end{array}
\right.
$$

$$
n(G-e) = 
\left\{
\begin{array}{ll}
	n(G) & e \text{ je most v } G \\
	n(G) - 1 & \text{jinak}
\end{array}
\right.
$$

$$
r(G/e) = 
\left\{
\begin{array}{ll}
	r(G) & e \text{ je smyčka v } G \\
	r(G)-1 & \text{jinak}
\end{array}
\right.
$$

$$
n(G/e) = 
\left\{
\begin{array}{ll}
	r(G)-1 & e \text{ je smyčka v } G \\
	r(G) & \text{jinak}
\end{array}
\right.
$$

\begin{definice}
	\textbf{Tutteův polynom} multigrafu $G=(V,E)$, značený $T_{G}(x,y)$ je definován:
	
	$$
	T_{G} = \sum_{F \subseteq E} (x-1)^{r(V,E)-r(V,F)} \cdot (y-1)^{n(V,F)}
	$$
\end{definice}

\begin{pozn}
	$x^{0}$ je konstantní funkce $\equiv 1$
\end{pozn}

\begin{pozor}
	$T_{G}(1,1) =$ \# počet koster v souvislém grafu $G$.
\end{pozor}

\begin{tvrz}
	Nechť $G_{1} =(V_{1},E_{1})$ a $G_{2} = (V_{2},E_{2})$ jsou multigrafy, kde $E_{1} \cap E_{2} = \emptyset$ a $|V_{1} \cap V_{2}| \leq 1$. Nechť $G = (V = V_{1} \cup V_{2}, E = E_{1} \cup E_{2})$. Potom $T_{G}(x,y) = T_{G_{1}}(x,y)T_{G_{2}}(x,y)$.
\end{tvrz}

\begin{proof}
	Nechť $V_{1} \cap V_{2} = \emptyset$ (situace $|V_{1} \cap V_{2}| = 1$ je obdobná).
	
	$T_{G}(x,y) = \sum_{F_{1} \subseteq E_{1}} \sum_{F_{2} \subseteq E_{2}} (x-1)^{r(V,E) - r(V, F_{1} \cup F_{2}} \cdot (y-1)^{n(V,F_{1} \cup F_{2}} =$ \textbf{(1)}. $r(V, F_{1} \cup F_{2}) = r(V,F_{1}) + r(V,F_{2})$ stejně tak i pro $n(G)$.
	
	$$
	(1) = \sum_{F_{1} \subseteq E_{1}} \sum_{F_{2}} (x-1)^{r(E_{1})+r(E_{2}) - (r(F_{1}) + r(F_{2}))} \cdot (y-1)^{n(F_{1}) + n(F_{2})} =
	$$
	
	$$
	\left( \sum_{F_{1}\subseteq E_{1}} (x-1)^{r(E_{1}) - r(F_{1})} \cdot (y-1)^{n(F_{1})}\right) \left( \sum_{F_{2}\subseteq E_{2}} (x-1)^{r(E_{2}) - r(F_{2})} \cdot (y-1)^{n(F_{2})} \right) =
	$$
	
	$$
	=T_{G_{1}}(x,y) T_{G_{2}}(x,y)
	$$
\end{proof}

\begin{dusl}
	$e$ je most v $G+(V,E)$, tak $T_{G-e}(x,y) = T_{G/e}(x,y)$.
\end{dusl}

\begin{pozor}
	$e$ je smyčka v $G$, potom $T_{G-e}(x,y) = T_{G/e}(x,y)$, protože $G-e = G/e$.
\end{pozor}

\begin{veta}
	Nechť $G = (V,E)$ je multigraf. Potom:
	
	\begin{enumerate}
		\item pokud $E = \emptyset$, tak $T_{G}(x,y) = 1$
		\item pokud $e \in E$, tak
		\begin{enumerate}
			\item pokud $e$ je smyčka, tak $T_{G}(x,y) = y \cdot T_{G-e}(x,y) = y \cdot T_{G/e}(x,y)$
			\item pokud $e$ je most, tak $T_{G}(x,y) = x \cdot T_{G-e}(x,y) = x \cdot T_{G /e}(x,y)$
			\item jinak $T_{G}(x,y) = T_{G-e}(x,y) + T_{G/e}(x,y)$.
		\end{enumerate}
	\end{enumerate}
\end{veta}

\begin{proof}
	1. Plyne z definice. 2. Volme $e \in E$ potom:
	
	$$
	T_{G}(x,y) = \sum_{F \subseteq E; e \notin F} \dots + \sum_{F \subseteq E; e \in F} \dots = S_{1} + S_{2}
	$$
	
	$$
	S_{1} = \sum_{F \subseteq E; e \notin F}(x-1)^{r(E \setminus \{e\}) - r(F)} \cdot (y-1)^{F}
	$$
	
	$$
	S_{2} = \sum_{F \subseteq E; e \in F}(x-1)^{r(E \setminus \{e\}) - r(F)} \cdot (y-1)^{F}
	$$
	
	$$
	T_{G-e}(x,y) = \sum_{F \subseteq E \setminus \{e\}}(x-1)^{r(E \setminus \{e\}) - r(F)} \cdot (y-1)^{F}
	$$
	
	$$
	T_{G/e}(x,y) = \sum_{F \subseteq E \setminus \{e\}}(x-1)^{r(E \setminus \{e\}) - r(F)} \cdot (y-1)^{F}
	$$
	
	Pokud $e$ není most v $G$ tak $r(E) = r(E \setminus \{e\})$, tedy $S_{1} = T_{G-e}(x,y)$. Pokud $e$ je most v $G$, tak $r(E) = r(E \setminus \{e\}) + 1$ a tedy $S_{1} = (x-1) \cdot T_{G-e}(x,y)$. Pokud $e$ je smyčka, tak $S_{2} = (y-1) \cdot T_{G/e}(x,y)$. Pokud $e$ není smyčka, tak $S_{2} = T_{G/e}(x,y)$. Takže pak celkově podle toho co je $e$:
	
	$$
	T_{G}(x,y) = S_{1} + S_{2} =
	\left\{
	\begin{array}{ll}
		\text{most} & (x-1)T_{G-e} + T_{G/e} = x \cdot T_{G/e} = x \cdot T_{G-e} \\
		\text{smyčka} & (y-1)T_{G/e} + T_{G-e} = y \cdot T_{G-e} = y \cdot T_{G/e} \\
		\text{jinak} & T_{G-e} + T_{G/e}
	\end{array}
	\right.
	$$
\end{proof}

\begin{definice}
	\textbf{Obarvení multigrafu} $G=(V,E)$ pomocí $b$ barev je funkce $f: V \to \{1,2,3, \dots ,b\}$ taková, že žádná hrana $e \in E$ nemá oba konce zbarvené na stejnou barvu. Pokud $G$ obsahuje smyčku, tak $G$ nemá žádné obarvení.
\end{definice}

\begin{definice}
	\textbf{Chromatický polynom} $G= (V,E)$ je funkce $\chi_{G}(z): \mathbb{N}_{0} \to \mathbb{N}_{0}$, kde $\chi_{G}(z)$ je počet obarvení $G$ pomocí $z$ barev.
\end{definice}

\begin{cvic}
	$G = K_{n}$ tak $\chi_{G}(z) = \binom{z}{n}n! = z \cdot (z-1) \cdot \dots \cdot (z - n +1)$ a $H = \bar{K_{n}}$ tak $\chi_{H}(z) = z^{n}$.
\end{cvic}

\begin{tvrz}
	Nechť $G = (V,E)$ je multigraf, $z \in \mathbb{N}_{0}$. Potom:
	
	\begin{enumerate}
		\item pokud $E = \emptyset$, tak $\chi_{G}(z) = z^{|V|}$
		\item pokud $e \in E$, tak:
		\begin{enumerate}
			\item pokud $e$ je smyčka tak $\chi_{G}(z) = 0$
			\item jinak $\chi_{G}(z) = \chi_{G-e}(z) - \chi_{G/e}(z)$.
		\end{enumerate}
	\end{enumerate}
\end{tvrz}

\begin{proof}
	1. Triviálně. 2. 1. Plyne z definice. 2. 2. jsou dvě možnosti: Více hran: pak musí být stejné obarvení a to také platí, protože $\chi_{G/e}(z) = 0$ kvůli smyčce. Jen jedna hrana, tak musíme odebrat obarvení, které dají oboum vrcholům stejnou barvu a to je přesně $\chi_{G/e}(z)$.
\end{proof}

\begin{tvrz}
	$\forall$ multigraf $G$:
	
	$$
	\chi_{G}(z) = (-1)^{|V| - k(G)} \cdot z^{k(G)} \cdot T_{G}(1-z,0)
	$$
\end{tvrz}

\begin{proof}
	Druhou stranu výrazu si označíme jako $PS_{G}(z)$. Pak jsou dva možné postupy.
	
	\begin{enumerate}
		\item Opraví se $PS_{G}(z)$ a zjistí se, že $PS_{G}(z) = \sum_{F \subseteq E}(-1)^{|F|} \cdot z^{k(V,F)}$. Pak pomocí **principu inkluze a exkluze** (*PIE*) ze zdůvodnní, že ten výraz je roven $\chi_{G}(z)$.
		\item Zkontroluje se, že $\chi_{G}(z)$ splní stejné podmínky rekurze jako $PS_{G}(z)$.
	\end{enumerate}
	
	V tomto případě volíme první možnost. Označme $\bar{\chi_{G}}(z) := \sum_{F \subseteq E}(-1)^{|F|} \cdot z^{k(V,F)}$ \textbf{(1)}. Pozorování: Pokud $G$ obsahuje smyčku $e$, tak $\bar{\chi_{G}}(z) = 0$, protože:
	
	$$
	\bar{\chi_{G}}(z) = \sum_{F \subseteq E \setminus \{e\}}((-1)^{|F|} \cdot z^{k(V,F)} + (-1)^{|F \cup \{e\}} \cdot z^{k(V,F)})
	$$
	
	\textbf{(1)} Předpokládejme, že $G$ neobsahuje smyčku. Označme si $\mathcal{F}:=$ množina všech funkcí $|V| \to \{1,2,\dots, z\}$ $|\mathcal{F}|= z^{|V|}$. Pro hranu $e = \{xy\} \in E$ označím $\hat{S_{e}}:= \{f \in \mathcal{F}; f(x) = f(y)\}$.
	
	$$
	\chi_{G}(z) = |\mathcal{F} \setminus \bigcup_{e \in E}\hat{S_{e}}| = |\mathcal{F}| - |\bigcup_{e \in E}\hat{S_{e}}| =^{\text{PIE}}
	$$
	
	$$
	=^{\text{PIE}} |\mathcal{F}| - (\sum_{\emptyset \neq F \subseteq E} (-1)^{|F|}+1 |\bigcap_{e \in E} \hat{S_{e}}|) = z^{|V|} + \sum_{\emptyset \neq F \subseteq E}(-1)^{|F|}|\bigcap_{e \in F}\hat{S_{e}}| =(1)
	$$
	
	Obecně $|\bigcup_{e \in E} \hat{S_{e}}| = z^{k(V,E)}$, protože v komponentě musí být jedna barva.
	
	$$
	(1) = \sum_{F \subseteq E}(-1)^{F} z^{k(V,F)} = \bar{\chi_{G}}(z)
	$$
\end{proof}
