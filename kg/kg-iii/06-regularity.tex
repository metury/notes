\chapter{Regularity of graphs}

As one can already met the random graphs which are denoted as $G = (n,p)$ where we have $n$ vertices and every edge between each pair of a vertices is there with a probability $p$. This notation si usually used for proving some properties that graphs tend to have for more vertices.

Example could be computing the number of triangles in the graph. To be exact we want to compute the expected number, so $\E [\text{number of triangle}] = p^3 \binom{n}{3}$. And also then we may use Chernoff's bound to get that $\Pr \left[|\text{number of triangles } - p^3 \binom{n}{3}| > \epsilon n^3\right] = e^{-\Theta(n^3)}$. In other words the probability that the number being way different is small.

In the next example and for further reading we will denote $e(A,B)$ as the number of edges between two subset of vertices $A, B \subseteq V$. In the probability case it is somewhat similiar to $p \cdot |A| \cdot |B|$. Which leads to the fact that

$$
\Pr \left[\left|e(A,B) - p \cdot |A| \cdot |B|\right| \geq \epsilon \cdot |A|\cdot |B| \right] < e^{-\Theta(|A||B|)}.
$$

If we would take $A, B$ such that $|A|, |B| \geq \delta n$ then there may be $\leq 4^n$ number of subsets.

Well the real graphs we may encounter do not have this nice property. So we will build a theory around that in fact it is not that bad and can be somewhat similiar to the probability case.

\section{Regular pairs}

\begin{defn}
	Lets denote $d(A,B) = \frac{e(A,B)}{|A| \cdot |B|}$ or so called \textbf{density} of a graph. Sometimes we will denote $d(A,B)$ as $p$, because it will behave somewhat similarly as the probability.
\end{defn}

\begin{defn}
	For some $\delta, \epsilon > 0$. Let $A, B \subseteq V$ for some graph $G = (V,E)$. Then we say that $(A,B)$ is $(\delta, \epsilon)$-\textbf{regular pair} if following holds
	
	$$
	\forall A' \subseteq A : |A'| \geq \delta |A|, \forall B' \subseteq B : |B'| \geq \delta |B| : |d(A'B') - d(A,B)| \leq \epsilon.
	$$
	
	Sometimes we will shorten the notation to just $\epsilon$-regular pair. For $\delta = \epsilon$.
\end{defn}

Now we will look at how many neighbors can be for some $v \in A$. To be exact we will denote $B_{0}$ the neighbors and we want to count these vertices for which $|B_{0}| \geq \delta |B|$. But we may see there are some degenerate vertices, but we will show it is not that many.

\begin{lemma}
	The number of vertices in $A$ such that $\deg_{B_{0}}v > (p + \epsilon |B_{0}|)$ is at most $\delta |A|$. And also the number of vertices in $A$ such that $\deg_{B_{0}}v < (p - \epsilon |B_{0}|)$ is at most $\delta |A|$. And also vice versa for $B$.
\end{lemma}

\begin{proof}
	For contradiction denote $A_{0}$ as all such vertices in $A$ which have low number of neighbors. For the contradiction suppose $|A_{0}| \geq \delta |A|$. This result in the following.
	
	$$
	\begin{aligned}
		|d(A_{0}, B_{0}) - p| &\leq \epsilon \\
		 &= \left| \frac{e(A_{0},B_{0})}{|A_{0}| |B_{0}|} - p \right| \\
		 &= \left| \frac{\sum_{v \in A_{0}} \deg_{B_{0}}(v)}{|A_{0}| |B_{0}|} - p \right| \\
		 &= \left| \frac{|A_{0}| \cdot (p+ \epsilon |B_{0}|)}{|A_{0}| |B_{0}|} - p \right| \\
		 &= \frac{|A_{0}| \cdot (p+ \epsilon |B_{0}|)}{|A_{0}| |B_{0}|} - p = \epsilon
	\end{aligned}
	$$
	
	Which is a contradiction.
\end{proof}

Now we will denote $q(A,B)$ as the number of vertices $(v_1, v_2, v_3, v_4)$ forming a 4-cycle and $v_1 \neq v_3 \in A$ and $v_2 \neq v_4 \in B$. In the probability case we see that 

$$
\E [q(A,B)] = p^4 \binom{|A|}{2} \binom{|B|}{2} \cdot 4 \approx p^4 |A|^2 |B|^2 - O(n^3)
$$

One can be confused by the $4$ in there. That is because the way we defined it we count every 4-cycle four times.

\begin{lemma}
	For a graph $G$ and $A, B \subseteq V$ as an $(A, B)$ $\epsilon$-regular pair and $|A| = |B| = n$ we have following inequalities.
	
	$$
	(p^4 - 6 \epsilon)n^4 - 2n^3 \leq q(A,B) \leq (p^4 + 16 \epsilon) n^4
	$$
\end{lemma}

The exact constants are not that necessary, so they are picked somewhat randomly. We will prove only the lower bound. The other one is similiar.

\begin{proof}
	Lets put $A_{0} = \{ v \in A : \deg_{B} v < (p - \epsilon) n\}$. Which by the lemma we get $|A_{0}| \leq \epsilon n$. Now lets take $v_1 \in A \setminus A_{0}$. Now lets consider such $B_{v_1}$ (neighbors of $v_1$) for which $|B_{v_1}| \geq (p - \epsilon) n$. WLOG we may assume it is $\ge \epsilon n$, otherwise the lower bound is negative. Denote $A_{v_1} = \{v \in A  : \deg_{B_{v_1}} v < (p - \epsilon) |B_{v_1}|\}$. Also by the lemma $|A_{v_1}| \leq \epsilon n$. Now we choose $v_3 \in A \setminus A_{v_1}$.
	
	Now we may say that the number of 4-cycles is $\geq (1-\epsilon)n (1 - \epsilon)n ((p - \epsilon) |B_{v_1}|)^2 - 2n^3$. Where the first $(1-\epsilon)n$ is for $v_1$ and the second for $v_3$, we multiply it by the choices from $B$ and subtract all wrong choices (choosing the same vertex more than once). Now this can be calculated.
	
	$$
	\begin{aligned}
		\text{number of 4-cycles} & = (1- \epsilon)^2 (p - \epsilon)^4 n^4 - 2n^3 \\
		&\geq (1-2\epsilon) (p^4 - 4\epsilon) n^4 - 2n^3 \\
		&\geq (p^4 - 6\epsilon) n^4 - 2n^3
	\end{aligned}
	$$
\end{proof}

Now we will try to use this even more for computing the number of triangles from $A,B,C$ which are all $\epsilon$-regular pairs. Also denote the number as $t(A,B,C)$ then the densities will be denoted as $p_{AB}, p_{BC}, p_{AC}$. Also we are interested only in those triangles $v_1, v_2, v_3$ where $v_1 \in A, v_2 \in B, v_3 \in C$. In the random graph it would be $p_{AB} p_{BC} p_{AC} n^3$. But we will show it is also similiar with regular pairs.

\begin{lemma}
	For $\epsilon$-regular pairs $(A,B), (B,C), (A,C)$ which are the same size $|A| = |B| = |C| = n$ where the densities $d(X,Y)$ are denoted by $p_{XY}$ (as before). The following bounds hold.
	
	$$
	p_{AB} p_{BC} p_{AC} n^3 - 10 \epsilon n^3 \leq t(A,B,C) \leq p_{AB} p_{BC} p_{AC} n^3 + 10 \epsilon n^3
	$$
\end{lemma}

\begin{proof}
	Now we will show us the upper bound. Lets consider these sets:
	
	$$
	\begin{aligned}
		A_1 &= \{ v \in A : \deg_B v \geq (p_{AB} + \epsilon) n\} \quad |A_1| \leq \epsilon n \\
		A_2 &= \{ v \in A : \deg_C v \geq (p_{AC} + \epsilon) n\} \quad |A_2| \leq \epsilon n
	\end{aligned}
	$$
	
	Now we consider three separate cases for which set is used for the $v_1 \in A$.
	
	\begin{enumerate}[(i)]
		\item $v_1 \in A_1$ then $\epsilon |A_1| n^2 \leq \epsilon n^3$.
		\item $v_1 \in A_2$ similarly $\epsilon |A_2| n^2 \leq \epsilon n^3$.
		\item Last case is if $v_1 \in A \setminus (A_1 \cup A_2)$. We denote as $B_{v_1}$ and $C_{v_1}$ the neighbors of $v_1$ in $B$ and $C$ respectively. By the definition $|B_{v_1}| \leq (p_{AB} + \epsilon) n$ and  $|C_{v_1}| \leq (p_{AC} + \epsilon) n$. Therefore altogether $e(B_{v_1}, C_{v_1}) \leq (p_{BC} +\epsilon) \cdot |B_{v_1}| \cdot |C_{v_1}| \leq (p_{AB} p_{BC} p_{AC} + 7\epsilon) n^2$. So we may have at most $\epsilon n^2$ choices for $B_{v_1}$ and $C_{v_1}$.
	\end{enumerate}
	
	If we sum it all then the $t(A,B,C) \leq 2\epsilon n^3 + n (\epsilon n^2 + \epsilon n^2 + (p  + 7\epsilon)n^2) = (p + 9 \epsilon)n^3$.
\end{proof}

Now with all of that we would like to see if we are able to partition the vertices of graph $G$ into such sets which are $\epsilon$-regular. At least we would like to try bounding the number of partitions for some $m$ and larger $M_\epsilon$ which will be bounded to $\epsilon$. Well we may see that it won't work this way, like for the graph known as \textbf{Half graph} which is shown in the picture \ref{half-graph}.

\begin{figure}[!ht] \centering
	\begin{tikzpicture}[node distance={16mm}, main/.style = {draw, circle, fill, inner sep=2pt}]
		\node[main, color=mygreen] (1) {};
		\node[main, right of = 1, color=mygreen] (2) {};
		\node[main, right of = 2, color=mygreen] (3) {};
		\node[main, right of = 3, color=mygreen] (4) {};
		\node[main, right of = 4, color=mygreen] (5) {};
		\node[main, right of = 5, color=mygreen] (6) {};
		\node[main, right of = 6, color=mygreen] (7) {};
		\node[main, right of = 7, color=mygreen] (8) {};
		\node[main, right of = 8, color=mygreen] (9) {};
		
		\node[below of = 1] (phantom) {};
		
		\node[main, below of = phantom, color=myblue] (11) {};
		\node[main, right of = 11, color=myblue] (12) {};
		\node[main, right of = 12, color=myblue] (13) {};
		\node[main, right of = 13, color=myblue] (14) {};
		\node[main, right of = 14, color=myblue] (15) {};
		\node[main, right of = 15, color=myblue] (16) {};
		\node[main, right of = 16, color=myblue] (17) {};
		\node[main, right of = 17, color=myblue] (18) {};
		\node[main, right of = 18, color=myblue] (19) {};
		
		\draw (1) edge (11) (1) edge (12)
			(1) edge (13) (1) edge (14)
			(1) edge (15) (1) edge (17)
			(1) edge (16) (1) edge (19)
			(1) edge (18);
			
		\draw (2) edge (12) (2) edge (13)
		    (2) edge (14) (2) edge (15)
		    (2) edge (17) (2) edge (16)
			(2) edge (19) (2) edge (18);
		
		\draw (3) edge (13) (3) edge (14)
			(3) edge (15) (3) edge (17)
			(3) edge (16) (3) edge (19)
			(3) edge (18);
			
		\draw (4) edge (14) (4) edge (15)
			(4) edge (17) (4) edge (16)
			(4) edge (19) (4) edge (18);
			
		\draw (5) edge (15) (5) edge (17)
			(5) edge (16) (5) edge (19)
			(5) edge (18);
			
		\draw (6) edge (17) (6) edge (16)
			(6) edge (19) (6) edge (18);
			
		\draw (1) (7) edge (17) (7) edge (19) (7) edge (18);
			
		\draw (8) edge (18) (8) edge (19);
			
		\draw  (9) edge (19);
	\end{tikzpicture}
	\caption{Counterexample of a graph $G$ as a half graph with $18$ vertices.}
	\label{half-graph}
\end{figure}

\section{Regular partition}

\begin{defn}
	We will denote $\epsilon$-regular partition of graph $G$ with $n = |V(G)|$ such disjoint parts $\mathcal{P} = \{A_{1}, A_{2}, \dots, A_{m}\}$ of $V(G)$ for which all properties are satisfied.
	
	\begin{enumerate}
		\item $|A_1| = |A_2| = \dots = |A_m|$,
		\item $|A_0| \leq \epsilon n$ and
		\item for $1 \leq i \leq j \leq m$, $(A_i, A_j)$ are $\epsilon$-regular with at most $\leq \epsilon m^2$ exceptions.
	\end{enumerate}
	
	Also $m$ is the \textbf{order} of partition $\mathcal{P}$.
\end{defn}

\begin{thm}[Szemerédi, Regularity lemma (RL)]
	$\forall \epsilon > 0, \forall m_0 \geq 1$ there exists $M$ such that $\forall G : |V(G)| \geq m_0$ there exists $\epsilon$-regular partitioning of $G$ $A_0, \dots, A_m$ that $m_0 \leq m \leq M$.
\end{thm}

This theorem may actually be useful in some other theoretical results but as we will show us later on the number $M$ is approximately

$$
2^{2{^{2^{2^{\reflectbox{$\ddots$}^{m_0}}}}}} \quad \text{where there is } \frac{1}{\epsilon^4} \text{ number of twos.}
$$

\section{Applications}

\begin{lemma}[Removal lemma for triangle]
	$(\forall \alpha >0) \ (\exists n_0, \beta > 0) \ (\forall G : |V(G)| = n \geq n_0)$ then either
	
	\begin{itemize}
		\item $G$ contains $\beta n^3$ triangles, or
		\item $(\exists X \subseteq E(G): |X| \leq \alpha n^2)$ and $G-X$ is triangle free.
	\end{itemize}
\end{lemma}

This application and many others may be used for huge graphs that typically cannot fit into RAM or even is way to large to have the date (like a friendship graph for the whole world). Thus we use only sampling for some portion of the data and then we can be somewhat certain of the correctness.

\begin{proof}
	Firstly we will transform $\alpha$ into $\epsilon$ and $m_0$ (which will be $m_0 = 1/\epsilon$). Then we will use regularity lemma to obtain $M$ and lastly transform this into $n_0$ and $\beta$ where $n_0 = m_0$.
	
	Given graph $G$ we use RL which gives us $\epsilon$-regular partition $V_0, \dots, V_m$ for $m_0 \leq m \leq M$. Now lets denote
	
	$$
	\begin{aligned}
		X_1 &= \text{all edges incident to vertices in } V_0 \\
		X_2 &= \bigcup_{i \in 1, \dots, m} E(G[V_{i}]) \quad (\text{all edges connected in one part})\\
		X_3 &= \text{edges between } V_i \text{ and } V_j \text{, where } (V_i, V_j) \text{ is not } \epsilon-\text{regular} \\
		X_4 &= \text{edges between } V_i \text{ and } V_j \text{, where } d(V_i, V_j) \leq p \\
		X &= X_1 \cup X_2 \cup X_3 \cup X_4
	\end{aligned}
	$$
	
	Now lets compute the sizes of all sets.
	
	$$
	\begin{aligned}
		|X_1| &\leq |V_0| \cdot n \leq \epsilon n^2 \\
		|X_2| &\leq m \left( \frac{n}{m} \right)^2 \leq \frac{n^2}{m} \leq \frac{n^2}{m_0} = \epsilon n^2 \\
		|X_3| &\leq \epsilon m^2 \left( \frac{n}{m} \right)^2 \leq \epsilon n^2 \\
		|X_4| &\leq m^2 \cdot p \left( \frac{n}{m} \right)^2 = p n^2\\
		|X| &\leq (p + 3\epsilon) n^2
	\end{aligned}
	$$
	
	And we want $(p + 3\epsilon) n^2$ to be at most $\alpha n^2$. Thus $(p + 3\epsilon)$ to be at most $\alpha$. We will consider few cases.
	
	\begin{enumerate}[(a)]
		\item $G-X$ is triangle-free, therefore we are done.
		\item $G-X$ contains a triangle. None of the vertices are from $V_0$. All vertices are from different parts and also all of these parts are $\epsilon$-regular with density $> p$. This all comes from the definition of $X$. And now we may use the triangle lemma stated before. Hence the number of triangles in $G$ is $\geq (p^3 - 10\epsilon) t^3$ for $t$ as a number of vertices in those parts.
		
		$$
		t = \frac{n - |V_0|}{m} \geq \frac{(1-\epsilon)n}{m} \geq \frac{(1 - \epsilon)n}{M}
		$$
		
		So the number of triangles is
		
		$$
		\geq (p^3 - 10\epsilon) \left( \frac{1-\epsilon}{M} \right)^3 n^3
		$$
		
		Therefore $\beta = (p^3 - 10\epsilon) \left( \frac{1-\epsilon}{M} \right)^3$. Because $\beta$ has to be positive so $p \geq \sqrt[3]{10 \epsilon}$ so we put $p = \sqrt[3]{11\epsilon}$. We need to choose $\epsilon$ so that $\sqrt[3]{11\epsilon} + 3\epsilon = p + 3\epsilon \leq \alpha$.
	\end{enumerate}
\end{proof}

\begin{lemma}
	Let $H$ be a graph with $\chi(G) = k$ and $(\forall p > 0) \ (\exists \epsilon > 0, n_1)$ such that if in $G$ there are disjoint sets $V_1, \dots, V_k$ s.t.
	
	\begin{enumerate}
		\item $|V_1| = \dots = |V_{k}| = n$
		\item $\forall 1 \leq i < j \leq k$ $(V_i, V_j)$ is an $\epsilon$-regular and $d(V_i, V_j) \geq p$
	\end{enumerate}
	
	then $H \subseteq G$.
\end{lemma}

\begin{proof}
	Actually this proof is fairly similiar to the one before. We may divide the vertices to sets with the same color. Then we fix some vertices in $G$ for a color in the same part. After that we must choose huge subsets of other parts and fix other vertices from these. Because they are $\epsilon$-regular such subsets exists. Further proof is left as an exercise.
\end{proof}

\begin{thm}[Túrán]
	For a graph $G$ which has $|V(G)| = n$ and $|E(G)| > \left(1 - \frac{1}{k-1}\right) \frac{n^2}{2}$ then $K_k \subseteq G$.
\end{thm}

This won't be proven since it is more of a reminder. Also note that this is tight since we can create a graph with this many vertices and edges which does not contain $K_k$. That is the Túrán's graph. Now we can ask ourselves a question. If we have a graph $H$ what is the maximum number of edges of an $n$-vertex graph that does not contain $H$ as a subgraph. This is typical problem for extremal theory, so we can denote this number as a function $f_H(n)$.

\begin{thm}[Erdos-Stone]
	Suppose $k = \chi(H) \geq 2$ then $(\forall \beta > 0) \ (\exists n_0)$ if $|V(G)| = n \geq n_0$ and $|E(G)| \geq \left(1 - \frac{1}{k-1} + \beta \right) \frac{n^2}{2}$, then $H \subseteq G$.
\end{thm}

This theorem actually somewhat answers our question. Because $\lim_{n \to \infty} \frac{f_H (n)}{\binom{n}{2}} = 1 - \frac{1}{k-1}$. But this does not actually work that well for bipartite graphs, because the content in the parentheses goes to 0. So it gives only rough estimate.

\begin{proof}
	Similarly as before we will from $\beta$ get to $p = \beta/8$ and by the use of lemma for subgraph $H$ we get $0 < \epsilon < p$ and $n_1$. Lets put $m_0 = 1/\epsilon$ and use RL to get $M$ and then $n_0 = \max \left(m_0, \frac{M}{1-\epsilon} n_1 \right)$.
	
	For given graph $G$ create partition $V_0, V_1, \dots, V_m$ for $m_0 \leq m \leq M$ by the RL. We define $X_1,X_2,X_3,X_4$ and $X$ as before. So
	
	$$
	|X| \leq (p + 3\epsilon) n^2 < 4pn^2 = \frac{\beta}{2}n^2
	$$
	
	Therefore we may compute the size of edges as:
	
	$$
	|E(G-X)| > |E(G)| - \beta \frac{n^2}{2} \geq \left( 1 - \frac{1}{k-1} + \beta \right) \frac{n^2}{2} - \beta \frac{n^2}{2} \geq \left( 1 - \frac{1}{k-1} \right) \frac{n^2}{2}
	$$
	
	Lets define an auxiliary graph $F$. Which has $V(F) = \{1, \dots, m\}$ and $ij \in E(F)$ iff $(V_i,V_j)$ is an $\epsilon$-regular pair of density $\geq p$ iff $G-X$ has an edge between $V_i$ and $V_j$.
	
	$$
	\left( 1 - \frac{1}{k-1} \right) \frac{n^2}{2} < |E(G-X)| \leq |E(F)| \left(\frac{n}{m}\right)^2
	$$
	
	Hence altogether $|E(F)| > \left(1-\frac{1}{k-1}\right) \frac{m^2}{2}$ and by Túrán $K_k \subseteq F$ and therefore $k$ parts are $\epsilon$-regular.Each part has $\frac{n(1-\epsilon)}{m} \geq \frac{n(1-\epsilon)}{M} \geq n_1$ so by the lemma mentioned before we get that $H \subseteq G$.
\end{proof}

\subsection{Arithmetic Ramsey theory}

Firstly we will show a theorem that one could already encounter.

\begin{thm}[Van der Warden]
	$(\forall c) (\forall l) \forall \text{ coloring } \varphi : \N \to \{1, \dots, c\}$ there exists $k \in \{1, \dots, c\}, a, d > 0$ such that $\varphi(a) = \varphi(a + d) = \dots = \varphi(a + d(l -1))$.
\end{thm}

This is called a arithmetic progression or AP for short. Now the question may be if we could somewhat limit it and not use all numbers from $\N$. In particular if it is true that $(\forall c \geq 1) (\forall l) (\exists N_0) (\forall N \geq N_0) (\forall B \subseteq \{1, \dots, N\} : |B| \geq \frac{N}{c}) (\exists a, d> 0)$ such that $a, a+d, \dots, a+d(l-1) \in B$.

Actually Róth showed that it is indeed true for $l = 3$ and later on also Szemerédi showed that it holds for arbitrary $l$. Now we will show us the Róth theorem.

\begin{defn}
	Given $A \subseteq \{1, \dots, N\}^2$ the \textbf{corner} in $A$ is a triplet $(x,y), (x+d,y), (x, y+d) \in A$ for $d \neq 0$.
\end{defn}

\begin{lemma}[Corner lemma]
	$(\forall \delta > 0) (\exists N_1)(\forall N \geq N_1) (\forall A \subseteq \{1, \dots, N\}^2)$ if $|A| \geq \delta N^2$, then $A$ contains a corner.
\end{lemma}

\begin{proof}
	That is we obtain $\delta$ and by setting $\alpha = \delta/40$ we may use the triangle removal lemma and get $n_0, \beta$. After that we would get our $N_1$. (This will be $N_1 = \max \{\frac{n_0}{6}, \frac{1}{\beta}\}$.)
	
	Firstly we will create 3-partite graph. Three parts will be $X,Y$ and $Z$. Now we define when there are edges.
	
	\begin{itemize}
		\item $x \in X, y \in Y$ forms an edge $xy$ if $(x,y) \in A$,
		\item $x \in X, z \in Z$ forms an edge $xz$ if $(x,z-x) \in A$ and
		\item $z \in Z, y \in Y$ forms an edge $zy$ if $(z-y,y) \in A$.
	\end{itemize}
	
	One can already see that the triangles in this graph corresponds to corners. That is for a triplet $(x,y), (x, y +d), (x+d,y) \in A$ then $x,y,z=x+y+d$ forms a triangle. And the other way we have $xyz$ triangle. Then $(x,y), (x, y+d), (x+d,y) \in A$ for $d = z -x-y$.
	
	Now we can observe that $\forall (x,y) \in A$, $x,y, x+y$ is also a triangle and we will call these the \textbf{bad} triangles. We may see that $\delta N^2 \leq$ the number of bad triangles $= |A| \leq N^2$. Now we have two cases by the removing triangles lemma.
	
	\begin{enumerate}
		\item Either there is $\geq \beta (6N)^3$ triangles and we need to show that this is $> N^2 \geq$ bad triangles. So $216 \beta N > 1$ that is for the choice of $N_1$ previously mentioned.
		\item Or there $\exists X \in E: X \leq \alpha(6N)^2$ intersecting all triangles. $X = \frac{36}{40} \delta N^2$, but we may see because of the estimate on the number of bad triangles and also by the observation that bad triangles are edge-disjoint that $X$ is not possible.
	\end{enumerate}
\end{proof}


\begin{thm}[Róth]
	$(\forall c \geq 1) (\exists N_0) (\forall N \geq N_0) (\forall B \subseteq \{1, \dots, N\})$ if $|B| \geq \frac{N}{c}$, then $B$ contains 3-AP.
\end{thm}

\begin{proof}
	The scheme is: we obtain $c$ and create $\delta$ with which we use the corner lemma and get $N_1$ and create $N_0$ from it.
	
	Let $A = \{(x,y) : y-x \in B\}$ and $x,y \in \{1, \dots, 2N\}$. So for $k \in B$ there is $(1,k+1), (2,k+2), \dots, (N, k+N)$. $|A| = N \cdot |B| \geq \frac{N^2}{c} - \frac{(2N)^2}{4c}$ so we put $\delta = \frac{1}{4c}$ and $N_0 = N_1$. By the corner lemma $\exists x,y,d \neq 0$ s.t. $(x,y), (x,y+d), (x+d,y) \in A$. Now it means that $y-x \in B, y-x -d \in B$ and $y-x+d \in B$ which forms the 3-arithmetic progression.
\end{proof}

Now for arbitrary $l$ we could rewrite corner lemma for $l-1$ dimension case then use hypergraphs with $l-1$ tuples and so on. But the thing is that this is not that easy, since the regularity lemma wouldn't be enough. Everything would have to be redefined and proved again. %Example with K_4.

Now we will show that $\exists B \subseteq \{0, \dots, N-1\}$ and $|B| \geq \frac{N}{16^{\sqrt{\log_e N}}}$ without 3-AP. Firstly we will put $S \subseteq \Z^n$ without 3-AP. We will set $d = 2^{n-1}$ and $N = (2d)^n$ so actually $S \subseteq \{0, \dots, d-1\}^n$ then let $S_k = \{(x_1, \dots, x_n) \in \{0, \dots, d-1\}^n, x_1^2 + \dots + x_n^2 = k\}$ which is a sphere. This does not contain 3-AP. Because if we intersect two points on the sphere then the line does not intersect any other point. We will use this for $0 \leq k leq (d-1)^2n$ and also $\exists k: |S_k| \geq \frac{d^n}{(d-1)^2n+1} \geq \frac{d^n}{d^2n} = \frac{d^{n-2}}{n}$ and lets fix such $k$. Now we ap $(x_1, \dots, x_n)$ to the $f = \sum_{i = 1}^n x_i (2d)^{i-1}$. (Or in other words the number in base $2d$.) Let $B = f(S_k)$. And also $0 \leq f \leq (2d)^n - 1 = N +1$. $f$ is injective and $|B|$ is large enough. If we sum two digits then there will be no carries because we used base $2d$. If there would be 3-AP then $f(x) + f(y) = 2f(z)$ but $f(x) + f(y) = f(x + y)$ and $2f(z) = f(2z)$. And $x + y = 2z$ would mean there was 3-AP in $S_k$, which is a contradiction.

\section{Proof of regularity lemma}

To make the proof easier we will slightly change the definition of $\epsilon$-regular partition. So that it will be $\mathcal{P} = \{A_1, \dots, A_m, A_{m+1}, \dots, A_{m+t}\}$ where we have replaced $A_0$ by singleton parts. So $t \leq \epsilon n$ and $|V_{m+1}| = \dots = |V_{m+t}|$.

The overall sketch of the proof is that we will create a partition $\mathcal{P}_{0}$ and then step by step we will replace the partition by $\mathcal{P}_{1}, \dots, \mathcal{P}_{k}$ where $\mathcal{P}_{k}$ will be $\epsilon$-regular. We will also introduce a "quality" function $q$.

\begin{enumerate}
	\item In each iteration, the order of the partition increases at most exponentially.
	\item $(\forall i) q(\mathcal{P}_{i}) \geq q(\mathcal{P}_{i-1}) + \frac{\epsilon^5}{4}n^2$.
\end{enumerate}

Therefore $k \leq \frac{4}{\epsilon^5}$ and ahence $M = \exp(\exp(\dots \exp(m_0)))$ which is iterated $k$ times. That is all to the outline of the proof.

\begin{defn}
	The quality function $q : \text{ partition of } V(G) \to [0, n^2]$ is defined firstly by $q : A, B \to [0,n^2]$ for $A,B \subseteq V(G)$ disjoint subsets.
	
	$$
	q(A,B) = |A| \cdot |B| \cdot d^2(A,B) = \frac{e^2(A,B)}{|A| \cdot |B|}
	$$
	
	Then for partition $\mathcal{P} = \{V_1, \dots, V_p\}$ is defined as
	
	$$
	q(\mathcal{P}) = \sum_{1 \leq i < j \leq p} q(V_i, V_j).
	$$
\end{defn}

Firstly we will show us that if we subdivide the parts or refine them the quality can only rise.

\begin{lemma}
	If $\mathcal{A} = \{A_1, \dots, A_s\}$ is a partition of $A$ and $\mathcal{B} = \{B_1, \dots, B_t\}$ is a partition of $B$. Then
	
	$$
	\sum_{i = 1}^{s} \sum_{j = 1}^{t} q(A_i, B_j) \geq q(A,B).
	$$
\end{lemma}

\begin{proof}
	For the proof we will use \href{https://en.wikipedia.org/wiki/Cauchy%E2%80%93Schwarz_inequality}{Cauchy-Schwarz inequality}. That is $\left( \sum_i u_i^2 \right) \left( \sum_i v_i^2 \right) \geq \left| \sum_i u_i v_i \right|$. Lets define $u$ and $v$ and then sum it.
	
	$$
	\begin{aligned}
		u_{ij} &= \sqrt{|A_i| \cdot |B_j|} d(A_i, B_j) \\
		v_{ij} &= \sqrt{|A_i| \cdot |B_j|} \\
		\sum_{i,j} u_{ij}^2 &= \sum_{i,j} |A_i| \cdot |B_j| d^2(A_i, B_j) = \sum_{i,j} q(A_i, B_j) \\
		\sum_{i,j} v_{ij}^2 &= \sum_{i,j} |A_i| \cdot |B_i| = \left( \sum_i |A_i| \right) \left( \sum_i |B_i| \right) = |A| \cdot |B| \\
		\sum_{i,j} u_{ij} v_{ij} &= |A_i| \cdot |B_j| d(A_i, B_j) = \sum_{i,j} e(A_i, B_j) = e(A,B)
	\end{aligned}
	$$
	
	So witch Cauchy Schwarz inequality we get.
	
	$$
	\begin{aligned}
		\sum_{i,j} q(A_i, B_j) |A| \cdot |B| & \geq e(A,B)^2 \\
		\sum_{i,j} q(A_i, B_j) \geq \frac{e(A,B)^2}{|A| \cdot |B|} &= q(A,B)
	\end{aligned}
	$$
\end{proof}

\begin{lemma}
	If $\mathcal{P}'$ is a refinement of $\mathcal{P}$ then $q(\mathcal{P}') \geq q(\mathcal{P})$.
\end{lemma}

\begin{proof}
	Let $\mathcal{P} = \{V_1, \dots, V_m\}$ then subdivision of each part is $\mathcal{P}' = \mathcal{V}_1 \dot{\cup} \mathcal{V}_2 \dot{\cup} \dots \dot{\cup} \mathcal{V}_m$. To be precise $\mathcal{V}_i$ is a partition of $V_i$.
	
	$$
	\begin{aligned}
		q(\mathcal{P}') &\geq \sum_{1 \leq i < j \leq m} \sum_{P \in \mathcal{V}_i} \sum_{Q \in \mathcal{V}_j} q(P,Q) \\
		 &\geq \sum_{1 \leq i < j \leq m} q(V_i, V_j) \quad \text{(by the lemma)} \\
		 &= q(\mathcal{P})
	\end{aligned}
	$$
\end{proof}

\begin{lemma}
	Suppose $(A,B)$ is not $\epsilon$-regular. This means that $\exists A_1 \subseteq A, B_1 \subseteq B : |A_1| \geq \epsilon |A|, |B_1| \geq \epsilon |B|$ such that $|d(A_1, B_1) - d(A,B)| \geq \epsilon$. Lets denote $A_2 = A \setminus A_1, B_2 = B \setminus B_1$ then
	
	$$
	\sum_{i,j = 1}^2 q(A_i,B_j) \geq q(A,B) + \epsilon^4 |A| \cdot |B|.
	$$
\end{lemma}

\begin{proof}
	Firstly some simplification for $u \in A_i, v \in B_j$ $d(u,v) = d(A_i, B_j)$ also $d = d(A,B)$ and $\delta (u,v) = d(u,v) - d$. Now we will show us that the sum of deltas is zero.
	
	$$
	\begin{aligned}
		d \cdot |A| \cdot |B| &= e(A,B) = \sum_{i,j = 1}^2 e(A_i, B_j) = \sum_{i,j = 1}^2 d(A_i, B_j) |A_i| \cdot |B_j| \\
		&= \sum_{i,j = 1}^2 \sum_{u \in A_i} \sum_{v \in B_j} d(u,v) = \sum_{u \in A} \sum_{v \in B} d(u,v) \\
		&= \sum_{u \in A} \sum_{v \in B} \delta(u,v) + d = d \cdot |A| \cdot |B| + \sum_{u \in A} \sum_{v \in B} \delta(u,v)\\
		\sum_{u \in A} \sum_{v \in B} \delta(u,v) &= 0
	\end{aligned}
	$$
	
	With that and similiar process we will prove the lemma.
	
	$$
	\begin{aligned}
		\sum_{i,j = 1}^2 q(A_i, B_j) &= \sum_{i,j = 1}^2 |A_i| \cdot |B_i| \cdot d^2(A_i, B_j) \\
		&= \sum_{i,j = 1}^2 \sum_{u \in A_i} \sum_{v \in B_j} d^2(u,v) \\
		&= \sum_{u \in A} \sum_{v \in B} d^2(u,v)\\
		&= \sum_{u \in A} \sum_{v \in B} (d + \delta(u,v))^2\\
		&= \sum_{u \in A} \sum_{v \in B} (d^2 + 2d \delta(u,v) + \delta^2(u,v))\\
		&= |A| \cdot |B| \cdot d^2 + 2d \sum_{u \in A, v \in B} \delta(u,v) + \sum_{u \in A, v \in B} \delta^2(u,v)\\
		&\geq q(A,B) + 0 + \sum_{u \in A_1, v \in B_1} \delta^2(u,v)
	\end{aligned}
	$$
	
	Now because $|d(A_1, B_1) - d(A,B)| > \epsilon$ and $\delta^2(u,v) > \epsilon^2$ we get that
	
	$$
	\begin{aligned}
		\sum_{i,j = 1}^2 q(A_i, B_j) &> q(A,B) + \epsilon^2 |A_1| \cdot |B_1|\\
		&\geq q(A,B) + \epsilon^4 |A| \cdot |B|
	\end{aligned}
	$$
\end{proof}

Now suppose $\mathcal{P} = \{V_1, \dots, V_s, V_{s+1}, \dots, V_{s+t}\}$ where $V_{s+1}, \dots, V_{s+t}$ are singletons. Suppose it is not $\epsilon$-regular and $N = |V_1| = \dots = |V_s|$ and also $t \leq \epsilon n$. So lets denote $X = \{(i,j) : i < j \leq s, (V_i, V_j) \text{ is not } \epsilon\text{-regular}\}$. It means that $|X| \geq \epsilon s^2$. Look at $(i,j)$ if it is not $\epsilon$-regular than we can use the very last lemma and get four parts which we will denote $V_{ij1}, V_{ij2}$ and $V_{ji1}, V_{ji2}$. We get that $\sum_{a,b = 1}^2 q(V_{ija}, V_{jib}) > q(V_i, V_j) + \epsilon^4 N^2$. This will be for all pairs. These partitions will be denoted as $\mathcal{P}_{ij}$.

Altogether we will create new partitions by these refinements and putting it together with $\mathcal{P}$. So formally $\mathcal{P}'$ will ve common subdivision of $\mathcal{P}$ and $\mathcal{P}_{ij}$ for $(i,j) \in X$.

Now the number of non-singleton parts in $\mathcal{P}'$ is at most $s2^s$ since we split up to $s$ parts and each one $2^s$ times (doubling every time). Now we can compute the following.

$$
q(\mathcal{P}') \geq \sum_{1 \leq i < j \leq s + t} \sum_{P \text{ subpart of } V_i} \sum_{Q \text{ subpart of } V_j} q(P,Q)\\
$$

Now the part $\sum_{P \text{ subpart of } V_i} \sum_{Q \text{ subpart of } V_j} q(P,Q)$ is either $\geq q(V_i, V_j)$ by the first lemma or if $(i,j) \in X$ then $\geq q(V_i, V_j) + \epsilon^4 N^2$ by the first and last lemma. Therefore we may continue in the computation.

$$
\begin{aligned}
	q(\mathcal{P}') &\geq \sum_{1 \leq i < j \leq s + t} \sum_{P \text{ subpart of } V_i} \sum_{Q \text{ subpart of } V_j} q(P,Q)\\
	&\geq \sum_{1 \leq i < j \leq s +t} q(V_i, V_j) + |X| \epsilon^4 N^2\\
	&\geq q(\mathcal{P}) + \epsilon^5 (sN)^2\\
	&\geq q(\mathcal{P}) + \epsilon^5 ((1-\epsilon)n)^2\\
	&\geq q(\mathcal{P}) + \epsilon^5 (n/2)^2\\
	&\geq q(\mathcal{P}) + \frac{\epsilon^5}{4} n^2\\
\end{aligned}
$$

Which is the increase of quality we were aiming for. Therefore we will repeat this process. But we are not guaranteed to have same-sized parts. Thus we transform $\mathcal{P}'$ and we will try to round it to $N' = \left\lceil \frac{N}{4^2} \right\rceil$. We will greedily split those parts to obtain $N'$ parts. All the leftovers will proceed to be as a singletons. Hence we obtain $\mathcal{P}''$.

\begin{itemize}
	\item Non-singleton parts have size $N'$.
	\item $q(\mathcal{P}'') \geq q(\mathcal{P}') \geq q(\mathcal{P}) + \frac{\epsilon^5 n^2}{4}$.
	\item Number of non-singleton parts is at most $\frac{Ns}{N/4^s} = s4^s$.
	\item Number of new singleton parts is at most $s2^s(N' - 1) \leq s2^s \frac{N}{4^s} = s \frac{N}{2^s} \leq \frac{n}{2^s}$ for large $s$ it is small.
	\item Number of non-singleton parts is at least $S_0 / 2$.
\end{itemize}

Lastly we need to set up the parameters.

$$
\begin{aligned}
	s_0 &:= \max \left( 2m_0, \left\lceil 2\log_2 \frac{8}{\epsilon^6} \right\rceil \right)\\
	M &:= \max \left( \left\lceil \frac{2}{\epsilon} s_0 \right\rceil, f(f(f(\dots f(s_0)))) \right) \quad \text{(repeated } \lceil 4/\epsilon^5 \rceil \text{ times)}
\end{aligned}
$$

Lastly it is necessary to show that it actually works and that there is not too many of singletons. The exact details are omitted.