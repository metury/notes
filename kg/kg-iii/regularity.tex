\chapter{Regularity of graphs and Ramsey's extremal theory}

As one can already met the random graphs which are denoted as $G = (n,p)$ where we have $n$ vertices and every edge between each pair of a vertices is there with a probability $p$. This notation si usually used for proving some properties that graphs tend to have for more vertices.

Example could be computing the number of triangles in the graph. To be exact we want to compute the expected number, so $\E [\text{number of triangle}] = p^3 \binom{n}{3}$. And also then we may use Chernoff's bound to get that $\Pr [|\text{number of triangles } - p^3 \binom{n}{3}| > \epsilon n^3] = e^{-\Theta(n^3)}$. In other words the probability that the number being way different is small.

In the next example and for further reading we will denote $e(A,B)$ as the number of edges between two subset of vertices $A, B \subseteq V$. In the probability case it is somewhat similiar to $p |A| |B|$. Which leads to the fact that $\Pr [|e(A,B) - p|A||B|| \geq \epsilon |A||B|] < e^{-\Theta(|A||B|)}$. If we would take $A, B$ such that $|A|, |B| \geq \delta n$ then there may be $\leq 4^n$ number of subsets.

Well the real graphs we may encounter do not have this nice property. So we will build a theory around that in fact it is not that bad and can be somewhat similiar to the probability case.

\begin{defn}
	Lets denote $d(A,B) = \frac{e(A,B)}{|A| \cdot |B|}$ or so called density of a graph. Sometimes we will denote $d(A,B)$ as $p$, because it will behave somewhat similarly as the probability.
\end{defn}

\begin{defn}
	For some $\delta, \epsilon > 0$. Let $A, B \subseteq V$ for some graph $G = (V,E)$. Then we say that $(A,B)$ is $(\delta, \epsilon)$-\textbf{regular pair} if following holds
	
	\begin{itemize}
		\item $\forall A' \subseteq A : |A'| \geq \delta |A|$,
		\item $\forall B' \subseteq B : |B'| \geq \delta |B|$ and
		\item $|d(A'B') - d(A,B)| \leq \epsilon$.
	\end{itemize}
	
	Sometimes we will shorten the notation to just $\epsilon$-regular pair.
\end{defn}

Now we will look at how many neighbors can be for some $v \in A$. To be exact we will denote $B_{0}$ the neighbors and we want to count these vertices for which $|B_{0}| \geq \delta |B|$. But we may see there are some degenerate vertices, but we will show it is not that many.

\begin{lemma}
	The number of vertices in $A$ such that $\deg_{B_{0}}v > (p + \epsilon |B_{0}|)$ is at most $\delta |A|$. And also the number of vertices in $B$ such that $\deg_{A_{0}}v < (p + \epsilon |A_{0}|)$ is at most $\delta |B|$.
\end{lemma}

\begin{proof}
	For contradiction denote $A_{0}$ as all such vertices in $A$ which have low number of neighbors. For the contradiction suppose $|A_{0}| \geq \delta |A|$. This result in the following.
	
	$$
	\begin{aligned}
		|d(A_{0}, B_{0}) - p| &\leq \epsilon \\
		 &= \left| \frac{e(A_{0},B_{0})}{|A_{0}| |B_{0}|} - p \right| \\
		 &= \left| \frac{\sum_{v \in A_{0}} \deg_{b_{0}}(v)}{|A_{0}| |B_{0}|} - p \right| \\
		 &= \left| \frac{|A_{0} (p+ \epsilon |B_{0}|)}{|A_{0}| |B_{0}|} - p \right| \\
		 &= \frac{|A_{0} (p+ \epsilon |B_{0}|)}{|A_{0}| |B_{0}|} - p = \epsilon
	\end{aligned}
	$$
	
	Which is a contradiction.
\end{proof}

Now we will denote $q(A,B)$ as the number of vertices $(v_1, v_2, v_3, v_4)$ forming a 4-cycle and $v_1 \neq v_3 \in A$ and $v_2 \neq v_4 \in B$. In the probability case we see that 

$$
\E [q(A,B)] = p^4 \binom{|A|}{2} \binom{|B|}{2} \cdot 4 \approx p^4 |A|^2 |B|^2 - O(n^3)
$$

One can be confused by the $4$ in there. That is because the way we defined it we count every 4-cycle four times.

\TODO{The rest of the pages.}