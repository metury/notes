\chapter{Hales-Jewett theorem}

We will now jump to another topic and mainly the Hales-Jewett theorem. This was actually motivated by famous Van der Waerden theorem \ref{van-der-waerden}.

\begin{thm}
	For every $n > 0$ and every finite coloring of integers there exists monochromatic arithmetic progression of length $n$, i.e. $a, a + b, a + 2b, \dots, a + (n-1)b$ for $b > 0$.
	\label{van-der-waerden}
\end{thm}

But the main issue is that this theorem is rather more algebraic and also used in combinatorics. For example we could create an auxiliary hypergraph $H = (\N, \text{all pregressions of length }n)$ so that we are looking at edges.

\begin{topic}{Multidimensional Tic-Tac-Toe}
	Lets consider $\Sigma$ finite alphabet (set). Then $\Sigma^n$ is set of functions $[n] \to \Sigma$ (usually called words). Lets define a hypergraph by the following set. $L \subseteq \Sigma^n$ is combinatorial line if $M \subseteq \{1, 2, \dots, n\}$ nonempty and $f: [n] \to \Sigma$ such that
	
	$$
	L = \left\{g | \exists c \in \Sigma \ g(i) = \left\{
	\begin{array}{ll}
		c & \text{if } i \in M \\
		f(i) & \text{if } i \notin M
	\end{array}
	\right.\right\}.
	$$
	
	\begin{example}
		Lets have $w = (\{\lambda\} \cup \Sigma)^n$ then lets have $A\lambda\lambda B$ which can either be $AAAB$ or $ABBB$. So formally $M = \{2,3\}$ and $f = AAAB$.
	\end{example}

	\begin{figure}[!ht]\centering
		\begin{tikzpicture}[n/.style = {draw, circle, fill, inner sep=1.5pt},
			l/.style = {line width=2.5pt},
			o/.style = {myorange},
			g/.style = {mygreen},
			b/.style = {myblue}]
			\foreach \j in {1,...,3} {
				\foreach \i in {1,...,3} {
					\node[n] (\j-\i) at (2*\j,2*\i) {};
				}
			}
			\foreach \j in {1,...,3}{
				\node (l1-\j) at (1.5, 8 - 2*\j) {\j};
			}
			\foreach \j in {1,...,3}{
				\node (l2-\j) at (2*\j, 6.5) {\j};
			}
			\foreach \j in {1,...,3}{
				\draw[o,l] (\j-1) to (\j-2) to (\j-3);
				\node[o] at (2*\j, 1.5) {$\lambda$\j};
				\draw[g,l] (1-\j) to (2-\j) to (3-\j);
				\node[g] at (6.5, 2*\j) {\j$\lambda$};
			}
			\draw[b,l] (1-3) to (2-2) to (3-1);
			\node[b] at (6.5,1.5) {$\lambda\lambda$};	
		\end{tikzpicture}
		\caption{Combinatorial representation of combinatorial lines for $\Sigma = \{1,2,3\}$ and $n =2$. Which we can see we obtain a hypergraph $H = (\Sigma^n, \text{Lines})$.}
	\end{figure}
\end{topic}

\begin{thm}{Hales-Jewett, 1964}
	For every finite $\Sigma$ and finite $n > 0$ $\exists N > 0$ denoted as $\HJ (|\Sigma|, r)$ such that the chromatic number of $H(\Sigma^n, \text{Lines})$ is at least $n$, or alternatively if $\Sigma^n$ is $r$-colored then there exist monochromatic combinatorial line (which is a subspace of dimension 1).
\end{thm}

\begin{thm}{Van der Warden, 1927}
	For $r$ number of colors and $l$ length of progression is known then there is finite version $N = \VW(r,k)$. And $N$ is not primitive recursion.
\end{thm}

Actually in the terms of Van der Warden theorem Shelah showed that $\VW$ has upper bound via a primitive recursion which is of a kind

$$
2^{2^{r^{2^{2^{k+q}}}}}.
$$

\newpage

\begin{topic}{Proof of Hales Jewett theorem}
	\begin{defn}
		Parameter spaces are for $\Sigma$ finite alphabet and its $n$-dimensional parameter $\Sigma^n$ of functions $[r] \to \Sigma$ space.
	\end{defn}

	So the $d$-dimensional subspace is a subset of $\Sigma^n$ described by a word in $\Sigma \cup \{\lambda_1, \lambda_2, \dots, \lambda_d\}$ where every $\lambda_i$ appears at least once and is a set $S = \{w(u) | u \in \Sigma^d\}$.
	
	\begin{example}
		$\Sigma = \{a,b\}$ let $W = a \lambda_1 \lambda_2 b \lambda_1$, which by $(aa)$ is replaced to $aaaba$.
	\end{example}

	Now we will proceed with improvement of the Hales Jewett theorem.
	
	\begin{thm}{Hales Jewett improved}
		$\forall$ finite $\Sigma$ $\forall r > 0, d > 0$ there exists $N = \HJ(|\Sigma, r, d)$ such that if $\Sigma^N$ is $r$-colored then there exist monochromatic $d$-dimensional subspace.
		\label{hj-imrpov}
	\end{thm}

	\begin{observ}
		$\forall s, d: \ \HJ (s, 1, d) = d$ and also $\forall r, d: \ \HJ(1, r, d) = d$. Lets see that $\HJ(2,2,1) = 2$, or generally that $\HJ(2,r,1) = r$ which can be viewed by pigeonhole principle, so that there is $k$ zeroes followed by $r - k$ ones, then every pair makes a line. In $\Sigma^r$ are $r + 1$ step functions.
	\end{observ}
\end{topic}