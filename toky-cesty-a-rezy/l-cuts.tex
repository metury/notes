\chapter{$L$-bounded cuts}

In this chapter we will consider a new problem which is length bounded cuts. This problem is NP-hard.

\begin{itemize}[]
	\item \textbf{INPUT} $G = (V,E)$, $s,t \in V$, $L \in \N$.
	\item \textbf{OUTPUT} $F \subseteq E$ such that $d_{G \setminus F}(s,t) > L$.
	\item \textbf{OBJECTIVE} $\min (|F|)$.
\end{itemize}

Lets see an easy example of a graph $G$ as shown on the picture \ref{l-bounded cut} and for $L = 4$. There can actually be two minimal $L$-bounded cuts. The \textcolor{black}{first} one is actually not a "real" cut, but the \textcolor{black}{second} one is.

\begin{figure}[!ht]\centering
	\caption{Global scheme of the 3-SAT.}
	\label{l-bounded cut}
\end{figure}


\section{$L$-bounded flow}

For $L$-bounded cut there is also the opposite problem which is in P and it is the $L$-bounded flow.

\begin{itemize}[]
	\item \textbf{INPUT} $G = (V,E)$, $s,t \in V$, $L \in \N$.
	\item \textbf{OUTPUT} Flow between $s-t$ that can be decomposed into paths of length $\leq L$.
	\item \textbf{OBJECTIVE} $\max$ the flow.
\end{itemize}

We will also show us an example for a graph $G$ depicted on the picture \ref{l-bounded flow} for $L = 3k$. We may see that $L$-cut is $k+1$ since we may delete \textcolor{black}{these edges} but also \textcolor{black}{the bottom ones}. On the other hand $L$-flow is at most 2.

\begin{figure}[!ht]\centering
	\caption{Global scheme of the 3-SAT.}
	\label{l-bounded flow}
\end{figure}

\begin{observ}
	Every $L$-bounded $s-t$ path uses at least $k$-edges from the bottom so max $L$-flow is at most 2.
\end{observ}

Therefore the difference between $L$-cut and $L$-flow can be at least $\sqrt{n}$.

\section{Approximation for $L$-cut}

Consider the following LP relaxation. Alternatively the ILP will surely solve the problem. Lets denote $\mathcal{P}_L$ as the set of all $L$-bounded $s-t$ paths.

$$
\begin{aligned}
	\min \sum_{e \in E} x_e & \\
	\sum_{x \in p} x_e \geq 1 & \quad \forall p \in \mathcal{P}_L \\
	x_e \geq 0 & \quad \forall e \in E
\end{aligned}
$$

We may ask ourselves what is the integrality gap? We can create one simple algorithm for solving such problem.

\begin{algorithm}
	\caption{$L$-bounded approximation}
	\begin{algorithmic}[1]
		\Require $G = (V,E)$
		\Ensure $L$-bounded cut.
		\While{$\exists L$-bounded $s-t$ path in $G$}
			\State Remove all edges of $p$.
		\EndWhile
	\end{algorithmic}
\end{algorithm}

\begin{observ}
	While the OPT $\geq k$ therefore it is $L$-approximation. Since the $k$ is the number of $L$-paths.
\end{observ}