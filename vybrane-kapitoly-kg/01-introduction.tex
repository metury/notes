\chapter{Introduction to Steiner systems}

\section{Simple hypergraphs}

\begin{defn}
	Hypergraph is a tuple $(X, \M)$ where $\M \subseteq \mathcal{P}(X)$. Or generally just a set system.
\end{defn}

\begin{defn}
	A simple hypergraph (linear, $k$-graph) is $(X,\M)$ if $M_1 \neq M_2 \in \M \Rightarrow |M_1 \cap M_2| \leq 1$.
\end{defn}

\begin{example}
	Lets see some of the easier examples of simple hypergraphs.
	
	\begin{itemize}
		\item Graphs themselves are simple hypergraphs. Where $(X, \M)$ and $\M \subseteq \binom{X}{2}$.
		\item Or generally $k$-graphs, where $\M \subseteq \binom{X}{k}$.
		\item A well known Fano plane, see picture \ref{fano-plane}.
		\item Lets have a set of points $A$ and define $X = \binom{A}{2}$; which are edges in $A$ and $\M = \{\binom{T}{2} | |T| = 3, T \subseteq A\}$; which are triangles in $A$. This is also simple.
	\end{itemize}
\end{example}

\TODO{Finish the first two lectures.}